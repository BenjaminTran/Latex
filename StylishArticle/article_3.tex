%%%%%%%%%%%%%%%%%%%%%%%%%%%%%%%%%%%%%%%%%
% Stylish Article
% LaTeX Template
% Version 2.1 (1/10/15)
%
% This template has been downloaded from:
% http://www.LaTeXTemplates.com
%
% Original author:
% Mathias Legrand (legrand.mathias@gmail.com) 
% With extensive modifications by:
% Vel (vel@latextemplates.com)
%
% License:
% CC BY-NC-SA 3.0 (http://creativecommons.org/licenses/by-nc-sa/3.0/)
%
%%%%%%%%%%%%%%%%%%%%%%%%%%%%%%%%%%%%%%%%%

%----------------------------------------------------------------------------------------
%	PACKAGES AND OTHER DOCUMENT CONFIGURATIONS
%----------------------------------------------------------------------------------------

\documentclass[fleqn,10pt]{SelfArx} % Document font size and equations flushed left

\usepackage[english]{babel} % Specify a different language here - english by default

\usepackage{lipsum} % Required to insert dummy text. To be removed otherwise

%----------------------------------------------------------------------------------------
%	COLUMNS
%----------------------------------------------------------------------------------------

\setlength{\columnsep}{0.55cm} % Distance between the two columns of text
\setlength{\fboxrule}{0.75pt} % Width of the border around the abstract

%----------------------------------------------------------------------------------------
%	COLORS
%----------------------------------------------------------------------------------------

\definecolor{color1}{RGB}{0,0,90} % Color of the article title and sections
\definecolor{color2}{RGB}{0,20,20} % Color of the boxes behind the abstract and headings

%----------------------------------------------------------------------------------------
%	HYPERLINKS
%----------------------------------------------------------------------------------------

\usepackage{hyperref} % Required for hyperlinks
\hypersetup{hidelinks,colorlinks,breaklinks=true,urlcolor=color2,citecolor=color1,linkcolor=color1,bookmarksopen=false,pdftitle={Title},pdfauthor={Author}}

%----------------------------------------------------------------------------------------
%	ARTICLE INFORMATION
%----------------------------------------------------------------------------------------

\JournalInfo{Journal, Vol. XXI, No. 1, 1-5, 2013} % Journal information
\Archive{Additional note} % Additional notes (e.g. copyright, DOI, review/research article)

\PaperTitle{Article Title} % Article title

\Authors{John Smith\textsuperscript{1}*, James Smith\textsuperscript{2}} % Authors
\affiliation{\textsuperscript{1}\textit{Department of Biology, University of Examples, London, United Kingdom}} % Author affiliation
\affiliation{\textsuperscript{2}\textit{Department of Chemistry, University of Examples, London, United Kingdom}} % Author affiliation
\affiliation{*\textbf{Corresponding author}: john@smith.com} % Corresponding author

\Keywords{Keyword1 --- Keyword2 --- Keyword3} % Keywords - if you don't want any simply remove all the text between the curly brackets
\newcommand{\keywordname}{Keywords} % Defines the keywords heading name

%----------------------------------------------------------------------------------------
%	ABSTRACT
%----------------------------------------------------------------------------------------

\Abstract{\lipsum[1]~}

%----------------------------------------------------------------------------------------

\begin{document}

\flushbottom % Makes all text pages the same height

\maketitle % Print the title and abstract box

\tableofcontents % Print the contents section

\thispagestyle{empty} % Removes page numbering from the first page

%----------------------------------------------------------------------------------------
%	ARTICLE CONTENTS
%----------------------------------------------------------------------------------------

\section*{Introduction} % The \section*{} command stops section numbering

\addcontentsline{toc}{section}{Introduction} % Adds this section to the table of contents

DNA polymerases are an essential tool for researchers in the biological sciences
particularly if intensive PCR procedures are required in an experimental
protocol. Phusion is able to consistently generate a DNA product with great
accuracy, robustness, speed, and impressive yield. In fact, it has a greater
processivity than both \textit{Taq} and \textit{Pyroccocus
furiosus}\cite{ref:brochure}. For these reasons it is a frequently used DNA
polymerase and therefore quite expensive. Thus, there is motivation to find a
cheaper alternative to Phusion. Over the past several years, biosciences
laboratories at Rice have been searching for means to produce such an
alternative and have been successful.\par
By the procedure in Wang et al.\cite{ref:Wang}, a novel method to engineer DNA
polymerases with enhanced processivity and performance, biosciences laboratories
at Rice have developed a reliable procedure to produce this alternative. The
approach calls for the covalent linking of the polymerase domain to a sequence
non-specific dsDNA binding protein. The binding protein used is Sso7d from
\textit{Sulfolobus solfataricus}. This procedure was applied to Pfu polymerase,
commonly used for high fidelity PCR applications, with success. The resulting
product is PFU-DBD.\par
Previous Bioc 211 students have refined the process of purifying PFU-DBD, but
the question of what is the optimal method of storing the product remains
unanswered. Experiments performed by other students concurrently with our work
have shown that PFU-DBD tends to fare better in buffers with lower percentage
glycerol content. This is a somewhat unexpected result as usually glycerol is
used as a stabilizing agent for long term storage of proteins. We are
investigating if TEN buffer solution with a 0\% glycerol content would have an
impact on the shelf-life of PFU-DBD. Similarly, we tested if storage
temperatures would have an affect as well. Given that other experiments were run
with variations of potential storage parameters, a meta-analysis of the
aggregate data will hopefully reveal key insights into the optimal conditions
for the storage of PFU-DBD.

%------------------------------------------------

\section{Methods}

\begin{figure*}[ht]\centering % Using \begin{figure*} makes the figure take up the entire width of the page
\includegraphics[width=\linewidth]{view}
\caption{Wide Picture}
\label{fig:view}
\end{figure*}

\lipsum[4] % Dummy text

\begin{equation}
\cos^3 \theta =\frac{1}{4}\cos\theta+\frac{3}{4}\cos 3\theta
\label{eq:refname2}
\end{equation}

\lipsum[5] % Dummy text

\begin{enumerate}[noitemsep] % [noitemsep] removes whitespace between the items for a compact look
\item First item in a list
\item Second item in a list
\item Third item in a list
\end{enumerate}

\subsection{Subsection}

\lipsum[6] % Dummy text

\paragraph{Paragraph} \lipsum[7] % Dummy text
\paragraph{Paragraph} \lipsum[8] % Dummy text

\subsection{Subsection}

\lipsum[9] % Dummy text

\begin{figure}[ht]\centering
\includegraphics[width=\linewidth]{results}
\caption{In-text Picture}
\label{fig:results}
\end{figure}

Reference to Figure \ref{fig:results}.

%------------------------------------------------

\section{Results and Discussion}

\lipsum[10] % Dummy text

\subsection{Subsection}

\lipsum[11] % Dummy text

\begin{table}[hbt]
\caption{Table of Grades}
\centering
\begin{tabular}{llr}
\toprule
\multicolumn{2}{c}{Name} \\
\cmidrule(r){1-2}
First name & Last Name & Grade \\
\midrule
John & Doe & $7.5$ \\
Richard & Miles & $2$ \\
\bottomrule
\end{tabular}
\label{tab:label}
\end{table}

\subsubsection{Subsubsection}

\lipsum[12] % Dummy text

\begin{description}
\item[Word] Definition
\item[Concept] Explanation
\item[Idea] Text
\end{description}

\subsubsection{Subsubsection}

\lipsum[13] % Dummy text

\begin{itemize}[noitemsep] % [noitemsep] removes whitespace between the items for a compact look
\item First item in a list
\item Second item in a list
\item Third item in a list
\end{itemize}

\subsubsection{Subsubsection}

\lipsum[14] % Dummy text

\subsection{Subsection}

\lipsum[15-23] % Dummy text

%------------------------------------------------
\phantomsection
\section*{Acknowledgments} % The \section*{} command stops section numbering

\addcontentsline{toc}{section}{Acknowledgments} % Adds this section to the table of contents

So long and thanks for all the fish \cite{Figueredo:2009dg}.

%----------------------------------------------------------------------------------------
%	REFERENCE LIST
%----------------------------------------------------------------------------------------
\phantomsection
\bibliographystyle{unsrt}
\bibliography{sample}

%----------------------------------------------------------------------------------------

\end{document}
