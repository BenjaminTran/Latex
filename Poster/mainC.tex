%%%%%%%%%%%%%%%%%%%%%%%%%%%%%%%%%%%%%%%%%
% Jacobs Landscape Poster
% LaTeX Template
% Version 1.1 (14/06/14)
%
% Created by:
% Computational Physics and Biophysics Group, Jacobs University
% https://teamwork.jacobs-university.de:8443/confluence/display/CoPandBiG/LaTeX+Poster
% 
% Further modified by:
% Nathaniel Johnston (nathaniel@njohnston.ca)
%
% This template has been downloaded from:
% http://www.LaTeXTemplates.com
%
% License:
% CC BY-NC-SA 3.0 (http://creativecommons.org/licenses/by-nc-sa/3.0/)
%
%%%%%%%%%%%%%%%%%%%%%%%%%%%%%%%%%%%%%%%%%

%----------------------------------------------------------------------------------------
%	PACKAGES AND OTHER DOCUMENT CONFIGURATIONS
%----------------------------------------------------------------------------------------

\documentclass[final]{beamer}

\usepackage[scale=1.22]{beamerposter} % Use the beamerposter package for laying out the poster
\usepackage{eso-pic}
\usepackage{commath}
\usepackage{fancybox}
\usepackage{tikz}
\usepackage[labelfont=bf,justification=justified,format=plain]{caption} 
\usepackage[labelformat = empty,position=top]{subcaption}
\usepackage[export]{adjustbox}
\usepackage{verbatim}
\usetikzlibrary{positioning}
\usetikzlibrary{shapes, arrows}
\usetikzlibrary{backgrounds}
\usetikzlibrary{calc}


\usetheme{confposter} % Use the confposter theme supplied with this template

\definecolor{fu-blue}{RGB}{0,51,102}
\setbeamercolor{block title}{fg=fu-blue,bg=white} % Colors of the block titles
%\setbeamertemplate{background canvas}[vertical shading][bottom=white,top=structure.fg!25]
\setbeamercolor{block body}{fg=black,bg=white} % Colors of the body of blocks
\setbeamercolor{block alerted title}{fg=white,bg=dblue!70} % Colors of the highlighted block titles
\setbeamercolor{block alerted body}{fg=black,bg=dblue!10} % Colors of the body of highlighted blocks
% Many more colors are available for use in beamerthemeconfposter.sty

\newcommand\AtPagemyUpperLeft[1]{\AtPageLowerLeft{%
\put(\LenToUnit{0.02\paperwidth},\LenToUnit{0.885\paperheight}){#1}}}
\newcommand\AtPagemyUpperRight[1]{\AtPageLowerLeft{%
\put(\LenToUnit{0.9\paperwidth},\LenToUnit{0.885\paperheight}){#1}}}
\AddToShipoutPictureFG{
  \AtPagemyUpperLeft{{\includegraphics[height=9.55cm,width=9.58cm,keepaspectratio]{Rice.png}}}
}%
\AddToShipoutPictureFG{
  \AtPagemyUpperRight{{\includegraphics[height=9.5cm,width=9.53cm,keepaspectratio]{CMS.png}}}
}%


%-----------------------------------------------------------
% Define the column widths and overall poster size
% To set effective sepwid, onecolwid and twocolwid values, first choose how many columns you want and how much separation you want between columns
% In this template, the separation width chosen is 0.024 of the paper width and a 4-column layout
% onecolwid should therefore be (1-(# of columns+1)*sepwid)/# of columns e.g. (1-(4+1)*0.024)/4 = 0.22
% Set twocolwid to be (2*onecolwid)+sepwid = 0.464
% Set threecolwid to be (3*onecolwid)+2*sepwid = 0.708

\newlength{\sepwid}
\newlength{\onecolwid}
\newlength{\twocolwid}
\newlength{\threecolwid}
\setlength{\paperwidth}{48in} % A0 width: 46.8in
\setlength{\paperheight}{36in} % A0 height: 33.1in
\setlength{\sepwid}{0.024\paperwidth} % Separation width (white space) between columns
\setlength{\onecolwid}{0.22\paperwidth} % Width of one column
\setlength{\twocolwid}{0.464\paperwidth} % Width of two columns
\setlength{\threecolwid}{0.708\paperwidth} % Width of three columns
\setlength{\topmargin}{-0.5in} % Reduce the top margin size
%-----------------------------------------------------------

\usepackage{graphicx}  % Required for including images

\usepackage{booktabs} % Top and bottom rules for tables

%----------------------------------------------------------------------------------------
%	TITLE SECTION 
%----------------------------------------------------------------------------------------

\title{Extraction of $ \boldsymbol{v_{2}} $ Anisotropy Harmonic of $
\boldsymbol{\Xi} $ Baryons\\ Produced at LHC Energies} % Poster title

\author{Benjamin Tran and Wei Li} % Author(s)

\institute{Rice University Bonner Nuclear Laboratories} % Institution(s)

%----------------------------------------------------------------------------------------

\begin{document}

\addtobeamertemplate{block end}{}{\vspace*{2ex}} % White space under blocks
\addtobeamertemplate{block alerted end}{}{\vspace*{2ex}} % White space under highlighted (alert) blocks

\setlength{\belowcaptionskip}{2ex} % White space under figures
\setlength\belowdisplayshortskip{2ex} % White space under equations

\begin{frame}[t] % The whole poster is enclosed in one beamer frame

\begin{columns}[t] % The whole poster consists of three major columns, the second of which is split into two columns twice - the [t] option aligns each column's content to the top

\begin{column}{\sepwid}\end{column} % Empty spacer column

\begin{column}{\onecolwid} % The first column


%----------------------------------------------------------------------------------------
%	INTRODUCTION
%----------------------------------------------------------------------------------------

\begin{block}{Introduction}

It is believed that shortly after the Big Bang the universe was in a quark-gluon
plasma (QGP) state. This state exists at extremely high temperatures $T \sim
$ 175 MeV and energy-densities and is characterized by asymptotically free
quarks and gluons. QGP is believed to be produced at high energy
particle accelerators such as the LHC where colliding particles form a strongly
interacting system modeled by hydrodynamics. To study the properties of the QGP,
one can analyze the elliptic flow, $ v_{2} $, of different particle species.
Elliptic flow is a measure of the momentum asymmetry of the evolving system
after a collision. This asymmetry 

This statement requires citation \cite{Smith:2012qr}.

\end{block}

%------------------------------------------------

\begin{figure}
\includegraphics[width=1\linewidth]{Collision.png}
\caption{Left: Two heavy-ions just before a non-central collision. Right:
Spectator nucleons continue down beam pipe while participants form strongly
interacting system.}
\end{figure}

%----------------------------------------------------------------------------------------

%----------------------------------------------------------------------------------------
%	OBJECTIVES
%----------------------------------------------------------------------------------------

\begin{alertblock}{Objectives}

\begin{itemize}
\item Extract $ v_{2} $ of cascade particles.
\item Compare $ v_{2} $ as a function of transverse momentum to $
\mathrm{K}_{\mathrm{S}}^{0} $ and $ \Lambda/\overline{\Lambda} $.
\item Explore particle species dependence of $ v_{2} $ at high multiplicity and
low transverse momentum.
\item Apply analysis to $ \sqrt{S_{NN}} = 8 $ TeV.
\end{itemize}

\end{alertblock}
\end{column} % End of the first column

\begin{column}{\sepwid}\end{column} % Empty spacer column

\begin{column}{\twocolwid} % Begin a column which is two columns wide (column 2)

\begin{columns}[t,totalwidth=\twocolwid] % Split up the two columns wide column

\begin{column}{\onecolwid}\vspace{-.6in} % The first column within column 2 (column 2.1)

%----------------------------------------------------------------------------------------
%	MATERIALS
%----------------------------------------------------------------------------------------

\begin{block}{The CMS Detector}
\begin{figure}[H]
  \centering
  \includegraphics[scale=0.26]{cmsDetector.png}
  \caption{The Compact Muon Solenoid Detector}
\end{figure}


\end{block}

%----------------------------------------------------------------------------------------

\end{column} % End of column 2.1

\begin{column}{\onecolwid}\vspace{-.6in} % The second column within column 2 (column 2.2)

%----------------------------------------------------------------------------------------
%	METHODS
%----------------------------------------------------------------------------------------

\begin{block}{Analysis Workflow}

\tikzstyle{block} = [rectangle, draw=fu-blue!80, fill=llgreen, text width
=4.5in, rounded corners, line width=0.2cm, text centered,minimum height = 1.25in]
\tikzstyle{line} = [draw, -latex', line width=0.2cm]

\begin{center}
\begin{tikzpicture}[node distance=2in, auto]
\node[block] (init) {56M pPb collisions};
\node[block, below of=init] (cuts) {Kinematic cuts};
\node[block, below of=cuts] (correlation) {Particle correlations};
\node[block, below of=correlation] (Fourier) {Fourier fit};
\node[block, below of=Fourier] (extract) {Extract $ v_{2} $};

%\node[block, right of=init,xshift=3in] (peak) {Mass peak};
%\node[block, right of=cuts,xshift=3in] (Fourier) {Fourier Fit};
%\node[block, right of=correlation,xshift=3in] (Extract) {Extract $ v_{2} $};

% Draw Arrows
\path[line] (init) -- (cuts);
\path[line] (cuts) -- (correlation);
\path[line] (correlation) -- (Fourier);
\path[line] (Fourier) -- (extract);
\end{tikzpicture}
\end{center} 

%\begin{itemize}
%\item 5 TeV proton-lead (pPb) collisions (56M events).
%\item Kinematic cuts are determined to eliminate background and retrieve cascade
%candidates.
%\item Invariant mass peak is used to appropriately perform background
%subtraction of $ v_{2} $.
%\item Cascade candidates from the peak and sideband regions are correlated with
%charged hadrons and resulting 1D azimuthal correlation function is fitted with a
%Fourier expansion.
%\item Calculate $ v_{2}^{sig} $.
%\end{itemize}
\end{block}

%----------------------------------------------------------------------------------------

\end{column} % End of column 2.2

\end{columns} % End of the split of column 2 - any content after this will now take up 2 columns width

%----------------------------------------------------------------------------------------
%	IMPORTANT RESULT
%----------------------------------------------------------------------------------------

%\begin{alertblock}{Important Result}
%
%Lorem ipsum dolor \textbf{sit amet}, consectetur adipiscing elit. Sed commodo molestie porta. Sed ultrices scelerisque sapien ac commodo. Donec ut volutpat elit.
%
%\end{alertblock} 

%----------------------------------------------------------------------------------------

\begin{columns}[t,totalwidth=\twocolwid] % Split up the two columns wide column again

\begin{column}{\twocolwid} % The first column within column 2 (column 2.1)

%----------------------------------------------------------------------------------------
%	MATHEMATICAL SECTION
%----------------------------------------------------------------------------------------

\begin{block}{Method}

\begin{minipage}[t]{0.5\textwidth}
The $ v_{2} $ is determined from the Fourier fit of the 1D azimuthal correlation
function. The Fourier fit contains the first three terms:\vspace{0.5in}
\begin{equation}
\frac{1}{N_{\mathrm{trig}}}\dod{N^{\mathrm{pair}}}{\Delta\phi} =
\frac{N_{\mathrm{assoc}}}{2\pi}\left[ 1 +
\sum_{n}^{}2V_{n\Delta}\cos(n\Delta\phi) \right]
\end{equation}\vspace{0.3in}

The $ V_{2\Delta} $ factorizes into the product of single-particle anisotropies,

\begin{equation}
V_{2\Delta}(p_{\mathrm{T}}^{\mathrm{trig}},p_{\mathrm{T}}^{\mathrm{assoc}}) =
v_{2}(p_{\mathrm{T}}^{\mathrm{trig}})\times
v_{2}(p_{\mathrm{T}}^{\mathrm{assoc}})
\end{equation}

Therefore,

\begin{equation}
\boxed{
v_{2}(p_{\mathrm{T}}^{\Xi}) =
\frac{V_{2\Delta}(p_{\mathrm{T}}^{\Xi},p_{\mathrm{T}}^{\mathrm{ref}})}{\sqrt{V_{2\Delta}p_{\mathrm{T}}^{\mathrm{ref}},p_{\mathrm{T}}^{\mathrm{ref}}}}
}
\end{equation}



\begin{equation}
\kappa =\frac{\xi}{E_{\mathrm{max}}} %\mathbb{ZNR}
\end{equation}

\end{minipage}
\end{block}

%----------------------------------------------------------------------------------------

\end{column} % End of column 2.1

\begin{column}{\onecolwid} % The second column within column 2 (column 2.2)

%----------------------------------------------------------------------------------------
%	RESULTS
%----------------------------------------------------------------------------------------

\begin{block}{Results}
\begin{figure}[H]
  \centering
  \begin{subfigure}[t]{0.45\textwidth}
    \includegraphics[scale=0.46, valign=t]{PubInvMassNew.png}
  \end{subfigure}
  \begin{subfigure}[t]{0.45\textwidth}
    \includegraphics[scale=0.46, valign=t]{PubXiHadFourier.png}
  \end{subfigure}
  \caption{
  Left: Invariant mass peak for $ 185 \leq
  \mathrm{N}_{\mathrm{trk}}^{\mathrm{offline}} < 220 $ fitted with a double
  gaussian with common mean 1.3222 GeV. Right: Fourier fit of 1D azimuthal
  correlation function of cascade-hadron pairings.
  }
\end{figure}

%\begin{figure}
%\includegraphics[width=0.8\linewidth]{PubInvMassNew.png}
%\caption{Figure caption}
%\end{figure}

Nunc tempus venenatis facilisis. Curabitur suscipit consequat eros non porttitor. Sed a massa dolor, id ornare enim:

\begin{table}
\vspace{2ex}
\begin{tabular}{l l l}
\toprule
\textbf{Treatments} & \textbf{Response 1} & \textbf{Response 2}\\
\midrule
Treatment 1 & 0.0003262 & 0.562 \\
Treatment 2 & 0.0015681 & 0.910 \\
Treatment 3 & 0.0009271 & 0.296 \\
\bottomrule
\end{tabular}
\caption{Table caption}
\end{table}

\end{block}

%----------------------------------------------------------------------------------------

\end{column} % End of column 2.2

\end{columns} % End of the split of column 2

\end{column} % End of the second column

\begin{column}{\sepwid}\end{column} % Empty spacer column

\begin{column}{\onecolwid} % The third column

%----------------------------------------------------------------------------------------
%	CONCLUSION
%----------------------------------------------------------------------------------------

\begin{block}{Conclusion}

Nunc tempus venenatis facilisis. \textbf{Curabitur suscipit} consequat eros non porttitor. Sed a massa dolor, id ornare enim. Fusce quis massa dictum tortor \textbf{tincidunt mattis}. Donec quam est, lobortis quis pretium at, laoreet scelerisque lacus. Nam quis odio enim, in molestie libero. Vivamus cursus mi at \textit{nulla elementum sollicitudin}.

\end{block}

%----------------------------------------------------------------------------------------
%	ADDITIONAL INFORMATION
%----------------------------------------------------------------------------------------

\begin{block}{Additional Information}

Maecenas ultricies feugiat velit non mattis. Fusce tempus arcu id ligula varius dictum. 
\begin{itemize}
\item Curabitur pellentesque dignissim
\item Eu facilisis est tempus quis
\item Duis porta consequat lorem
\end{itemize}

\end{block}

%----------------------------------------------------------------------------------------
%	REFERENCES
%----------------------------------------------------------------------------------------

\begin{block}{References}

\nocite{*} % Insert publications even if they are not cited in the poster
\small{\bibliographystyle{unsrt}
\bibliography{sample}\vspace{0.75in}}

\end{block}

%----------------------------------------------------------------------------------------
%	ACKNOWLEDGEMENTS
%----------------------------------------------------------------------------------------

\setbeamercolor{block title}{fg=red,bg=white} % Change the block title color

\begin{block}{Acknowledgements}

\small{\rmfamily{Nam mollis tristique neque eu luctus. Suspendisse rutrum congue nisi sed convallis. Aenean id neque dolor. Pellentesque habitant morbi tristique senectus et netus et malesuada fames ac turpis egestas.}} \\

\end{block}

%----------------------------------------------------------------------------------------
%	CONTACT INFORMATION
%----------------------------------------------------------------------------------------

\setbeamercolor{block alerted title}{fg=black,bg=norange} % Change the alert block title colors
\setbeamercolor{block alerted body}{fg=black,bg=white} % Change the alert block body colors

\begin{alertblock}{Contact Information}

\begin{itemize}
\item Web: \href{http://www.university.edu/smithlab}{http://www.university.edu/smithlab}
\item Email: \href{mailto:john@smith.com}{john@smith.com}
\item Phone: +1 (000) 111 1111
\end{itemize}

\end{alertblock}

\begin{center}
\begin{tabular}{ccc}
\includegraphics[width=0.4\linewidth]{logo.png} & \hfill & \includegraphics[width=0.4\linewidth]{logo.png}
\end{tabular}
\end{center}

%----------------------------------------------------------------------------------------

\end{column} % End of the third column

\end{columns} % End of all the columns in the poster

\end{frame} % End of the enclosing frame

\end{document}
