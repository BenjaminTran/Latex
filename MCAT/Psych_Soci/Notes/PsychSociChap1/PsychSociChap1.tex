% !TeX root = PsychSociChap1.tex
\documentclass[../Psych_Soci_review.tex]{subfiles}

\begin{document}

\chapter{Biological Foundations of Behavior}

\section{Neuronal Structure and Function}

Structural components:
\begin{enumerate}
  \item \vocab{Soma}: The central cell body; contains nucleus and where most of
    the biosynthetic activity of the cell takes place
  \item \vocab{Axons and dendrites}: have only one axon(sends signals) but can
    have many dendrites(receives signals). One(many) dendrite =
    \emph{bipolar(multipolar)}.
  \item \vocab{Synaptic knobs}: axons terminate here, these form connections
    with target cells.
  \item \vocab{Synaptic cleft}: releases chemical messengers when receives
    action potential.
\end{enumerate}
Full structure shown below:
\begin{figure}[h]
  \centering
  \includegraphics[scale=0.1,frame]{neuron.jpg}
\end{figure}

\section{The Action Potential}
\subsection{Resting Membrane Potential}

Potential of $ \approx $ -70 mV (interior negative wrt exterior). Two proteins
to create potential: \chemfig{Na^{+}/K^{+}} ATPase and potassium leak channels.
The ATPase pumps three sodium out for two potassium in and hydrolyzes one ATP.
Leak channels allow for ions to flow down gradient at all times. There are very
few sodium leak channels.\\
Neurons are polarized. The firing of an action potential (AP) causes
depolarization down the axon. After an action potential has fired the cell
re-polarizes.

\subsection{Depolarization}

\vocab{Voltage-gated sodium channels} are important for AP propagation. Responds
to change in membrane potential (when it becomes $ \approx $ -50 mV); allows
sodiums to flood in causing depolarization to about +35 mV. Slightly depolarizes
next section and process carries on.

\subsection{Repolarization}

To produce this effect the following occur:
\begin{enumerate}
  \item Voltage-gated sodium channels inactivate very quickly
  \item Voltage-gated potassium channels open for longer and close only when
    potential drops to $ \approx -90 $ mV
  \item Potassium leak channels and ATPase bring potential back to resting
\end{enumerate}

\subsection{Saltatory Conduction}

\vocab{Myelin} insulates axon. \vocab{Schwann cells} are glial cells that
produce the myelin. Since no charges may flow through the myelin sheaths there
are gaps in the covering called \vocab{nodes of Ranvier} where all of the
channels are concentrated.

\subsection{Glial Cells}

Glial cells provide structural and metabolic support to neurons. Maintain
resting membrane potential but do not generate AP. Examples below:

\begin{figure}[h]
  \centering
  \includegraphics[scale=0.15,frame]{Glial.jpg}
\end{figure}

\subsection{Equilibrium Potentials}

\vocab{Equilibrium potential} is the membrane potential at which the gradient
driving movement of ions does not exist. It is specific for particular ions and
can be determined via the Nernst Equation:
\[ E_{ion} = \dfrac{RT}{zF}\ln
  \dfrac{\left[X\right]_{outside}}{\left[X\right]_{inside}} \]
For \chemfig{Na^{+}} it is +50 mV and \chemfig{K^{+}} it is -90 mV. Thus,
equilbrium at -70 mV reflects the difference in equilibrium potentials and the
relative number of leak channels for the two ions.

\subsection{Refractory Period}

Thousands of AP can pass per second. After an AP neuron enters refractory and
cannot fire another. There are two phases. During \vocab{absolute refractory
period} a neuron will not fire no matter how strong a membrane depolarization is
induced (the voltage-gated sodium channels are inactivated). During
\vocab{relative refractory period} neuron can fire but harder since it is
hyperpolarized. See \figref{refractory}:

\begin{figure}[H]
  \centering
  \includegraphics[scale=1.7]{refractory.jpg}
  \caption{Neuron refractory periods}
  \label{refractory}
\end{figure}

\section{Synaptic Transmission}

Two types of synapses:
\begin{enumerate}
  \item \vocab{Electrical synapse}\\
    When cytoplasm of two cells are joined by gap junctions so AP will spread
    directly from one cell to another. Not common in nervous system but
    important for smooth and cardiac muscle.
  \item \vocab{Chemical synapse}\\
    Neurotransmitters. The following steps are involved (see \figref{synapse} )

    \begin{enumerate}
      \item AP reaches synaptic knob
      \item Depolarization of presynaptic membrane opens voltage-gated calcium
        channels
      \item Calcium enters presynaptic cell causing release of neurotransmitters
        via secretory vessicls.
      \item Neurotransmitter diffuse across cleft
      \item Neurotransmitters bind to postsynaptic membrane (ligand-gated ion
        channels)
      \item Opening of said channels in postsynaptic cell alters membrane
        polarization
      \item If membrane depolarization reaches threshold then AP is initiated
      \item Neurotransmitter in cleft is degraded or removed
    \end{enumerate}
\end{enumerate}


\begin{figure}[h]
  \centering
  \includegraphics[scale=0.15,frame]{Synapse.jpg}
  \caption{Synaptic process}
  \label{synapse}
\end{figure}

Example of chemical synapse is the \vocab{neuromuscular junction} between
neurons and skeletal muscles. Neurotransmitter is \vocab{Acetylcholine (ACh)}.
Other common neurotransmitters are:
\begin{enumerate}
  \item Glutamate (\textbf{Excitatory})
  \item Gamma-aminobutyric acid (GABA) (\textbf{Inhibitory})
  \item serotonin
  \item dopamine
  \item norepinephrine
\end{enumerate} 

If neurotransmitter depolarizes postsynaptic membrane then it is
\vocab{excitatory}. If it hyperpolarizes then it is \vocab{inhibitory}. This all
depends on the action of the receptor and not the neurotransmitter. Thus, a
neurotransmitter may be excitatory in some situations and inhibitory in others.

\subsection{Summation}

Action potential is initiated at the threshold depolarization of -50 mV. A
postsynaptic neuron has many different neurons with synapses leading to it each
releasing neurotransmitters (NT). The postsynaptic neuron sums the total stimuli
(\vocab[]{summation}) to determine if the threshold is crossed and thus
depolarize (this is called \vocab[]{excitatory postsynaptic potentials}, EPSP).
Similarly, there are \vocab[]{inhibitory postsynaptic potentials} (IPSP).

There are two types of summation. \vocab[!summation]{Temporal summation} occurs
when a presynaptic neuron fires APs quickly enough to cause the excitatory
postsynaptic potentials(inhibitory postsynaptic potentials) to pile up and cross
threshold and depolarize(hyperpolarize). \vocab[!summation]{Spatial summation}
occurs when all EPSPs and IPSPs are summed at a given moment in time.

\section{Functional Organization of the Human Nervous System}

\end{document}
