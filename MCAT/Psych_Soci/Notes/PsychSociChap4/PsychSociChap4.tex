\documentclass[../Psych_Soci_review.tex]{subfiles}

\begin{document}
\chapter{Interacting with the Environment}
\section{Attention}
\subsection{Selective Attention}
\vocab[]{Selective attention} is the process by which one input is attended to
and the rest are tuned out. The \vocab[]{cocktail party effect} is when
information of personal importance from unattended channels catches our
attention. Visual attention is explained through the \vocab[]{spotlight model}
where our focus can scan the visual field (note this is separate from our eyes,
i.e. we do not need to be looking exactly at something for our attention to be
on it).

\subsection{Divided Attention}
\vocab[]{Divided attention} concerns when and if we are able to perform multiple
tasks simultaneously. The \vocab[]{resource model of attention} says that we
have a limited pool of resources to work with. Three factors affect performance
of multitasking: task similarity, task difficulty, and task practice.

\section{Cognition}
\subsection{Information-Processing Models}
Contrary to \vocab[]{behaviorism}, which is concerned mostly with the link
between stimulus and response, \vocab[]{information-processing models} focus on
mental processing of information through a series of steps: \vocab[]{attention},
\vocab[]{perception}, and \vocab[]{storage into memory}. \vocab[]{Alan
Baddeley's model} introduces \vocab[]{working memory}. You can only keep a few
things in your working memory at any time. There are four components
\begin{enumerate}
  \item \vocab[]{phonological loop} repeats verbal information
  \item \vocab[]{visuospatial sketchpad} mental images
  \item \vocab[]{episodic buffer} information in working memory can interact
    with info in long-term memory
  \item \vocab[]{central executive} oversees the process and orchestrates the
    process by shifting and dividing attention
\end{enumerate}

\subsection{Cognitive Development}
\subsubsection{Piaget's Stages of Cognitive Development}
\vocab[]{Jean Piaget} says cognitive development of children depends on
\vocab[]{schemas}. There are four stages of development in his theory
\begin{enumerate}
  \item \vocab[]{Sensorimotor Stage} 0-2 years. Experience world through senses
    and movement. Acquire \vocab[]{object permanence} during this time as well.
  \item \vocab[]{Preoperational Stage} 2-7 years. Learn that things can be
    represented through symbols such as words and images. No logical reasoning.
  \item \vocab[]{Concrete Operational Stage} 7-11 years. Learn
    \vocab[]{conservation} and initial logical reasoning.
  \item \vocab[]{Formal Operational Stage} 12-adulthood. Abstract reasoning and
    moral reasoning.
\end{enumerate}

\subsection{Problem Solving and Decision Making}
We may solve problems through three methods: \vocab[]{trial and error},
\vocab[]{algorithms}, and \vocab[]{heuristics}, which are mental shortcuts.
However, sometimes we encounter two barriers to problem solving:
\begin{description}
  \item \vocab[]{Confirmation bias} We search for information that only supports
    what we preconceived to be the answer.
  \item \vocab[]{Fixation} We are stuck only looking at one method to solve the
    problem trying to make it work and do not consider the problem from a fresh
    perspective. \vocab[]{Functional fixedness} refers to trying to solve a
    problem using something that is meant to solve that problem e.g. looking for
    a box opener to cut a package when a key would also work.
\end{description}

\subsubsection{Heuristics and Biases}
Two heuristics:
\begin{description}
  \item \vocab[]{Representativeness heuristic} the tendency to judge the
    likelihoods of events occurring based on our schemas.
  \item \vocab[]{Availability heuristic} making judgements based on how readily
    available information is in our memories, e.g. see something often you
    believe it happens often.
\end{description}

\section{Consciousness}
\subsection{Sleep}
\subsubsection{Stages of Sleep}
Four stages of sleep and three sleep waves (Note that $ \beta $-waves occur
while you are awake).
\begin{enumerate}
  \item You have when \vocab[]{alpha-waves} you are awake and about to fall
    asleep  
  \item Stage 1: You have \vocab[]{theta waves}
  \item Stage 2: \vocab{K-complex} and \vocab[]{sleep spindles}
  \item Stage 3: You have \vocab[]{delta waves}, heart rate and digestion slow
    and secrete growth hormones
  \item Stage 4: \vocab[]{REM sleep} and you have increased eye movement and
    skeletal muscle movement.
\end{enumerate}
During one sleep cycle you will go 1,2,3,4,4,3,2,1, REM. 

\subsubsection{Sleep and Circadian Rhythms}
Light induces signals to the \vocab[]{pineal gland} which secretes
\vocab[]{melatonin}, a hormone that induces sleep.

\subsection{Hypnosis and Meditation}
Two theories to explain hypnotism:
\begin{enumerate}
  \item \vocab[]{Dissociation theory} suggests hypnotism is an extreme form of
    divided consciousness. Many actions occur on autopilot and in hypnotism the
    hypnotist just takes over the autopilot
  \item \vocab[]{Social Influence theory} suggests that people do and report
    what is expected of them. So you do what the hypnotist tells you to do
    because you are being told to do that
\end{enumerate}

\subsection{Conciousness Altering Drugs}
There are three main types of psychoactive drugs:
\begin{enumerate}
  \item depressants
  \item stimulants
  \item hallucinogenics
\end{enumerate}
\vocab[]{Depressants} include alcohol, barbiturates, and opiates.
\vocab[]{Barbiturates} depress the sympathetic nervous system (``fight or
flight''). \vocab[]{Opiates} depress neural functioning and are pain relievers
by mimicking endorphins.

\section{Emotion}
There are three components of emotion:
\begin{enumerate}
  \item physiological 
  \item behavioral 
  \item cognitive
\end{enumerate}
There are three theories to explain the relation of the components:
\begin{enumerate}
  \item \vocab[]{James-Lange Theory} physiological and behavioral response lead
    to the cognitive aspect of emotion i.e. you label your emotion based on your
    physiological and behavioral response to the situation
  \item \vocab[]{Cannon-Bard Theory} physiological and cognitive occur
    simultaneously and independently then they lead to a behavioral reaction.
    Shortcoming is that it can't explain behavioral reactions leading to
    physical and cognitive aspects of emotion.
  \item \vocab[]{Schachter-Singer Theory} physiological then cognitive then
    behavioral. Suffers from same shortcomings as Cannon-Bard.
\end{enumerate}
They are summarized in Figure \ref{fig:emo}.
\begin{figure}[H]
  \centering
  \includegraphics[scale=0.25,frame]{EmoTheory.jpg}
  \caption{Schematic Comparison of the Theories of Emotion}
  \label{fig:emo}
\end{figure}

\subsection{The Role of Biological Processes in Perceiving Emotion}
The limbic system is a collection of brain structures that lies on both sides of
the thalamus and is the primarily responsible for emotional experiences. The
main structure in the limbic system for emotion processing is the
\vocab[]{amygdala} which can communicate with the \vocab[]{hypothalamus} which
controls physiological aspects of emotion. It also communicates with the
\vocab[]{prefrontal cortex} which controls the behavioral aspects of
emotion.\par

The limbic system also includes the \vocab[]{hippocampus} which stores emotional
memories. The \vocab[]{amygdala} also plays a role for \vocab[]{executive
functions}--higher order thinking processes.

\subsection{The Branches of the Nervous System}
The \vocab[]{autonomic nervous system} is responsible for controlling the
activities of most of the organs and glands, and controls arousal. The
\vocab[]{sympathetic nervous system} is the ``fight or flight'' responses
(blood pressure, heart rate, etc). The \vocab[]{parasympathetic nervous system}
provides signals to the internal organs during a calm resting state when no
crisis is present.

\section{Stress}
\vocab[]{Appraisal} is how stress is interpreted by the individual. Stress
induces release of \vocab[]{cortisol} which shifts the body from using sugar as
an energy source towards using fat as an energy source.

\section{Language}
Three divisions of thought in psychology are exposed in language study:
\begin{enumerate}
  \item \vocab[]{Empiricist} focus on directly observable environmental factors
    as opposed to abstract mental states. These are \vocab{behaviorists}.
  \item \vocab[]{Materialist} everything happens in the grey matter of the brain
    and everything is just a metaphor for that. 
  \item \vocab[]{Rationalist} certain ideas and capabilities cannot come from
    experience and so must be innate. These are \vocab{nativists}.
\end{enumerate}

\vocab[]{Language acquisition} is the process by which infants learn to
understand and speak their own native language. B.F. Skinner's
\vocab{behaviorist} model of language acquisition \imp infants trained in
language acquisition by operant conditioning e.g. infant makes the sound of the
word and parents reward them by being happy and showing affection.\par

\vocab[]{Language acquisition device} aka \vocab[]{universal grammar} is
\vocab[]{Noam Chomsky}'s theory and it is \textit{an innate feature unique to
the human mind that allows people to gain mastery of language from limited
exposure during the sensitive developmental years in early childhood} e.g. at an
early age humans are capable of learning a language rapidly.\par

The \vocab[]{Broca's area} in the left hemisphere of the frontal lobe does
\textit{speech production}. The \vocab[]{Wernicke's area}  in the posterior of
the temporal lobe does \textit{comprehension of speech and language}.

\end{document}
