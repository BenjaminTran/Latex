\documentclass[../Psych_Soci_review.tex]{subfiles}

\begin{document}

\chapter{Learning and Memory\supdag}
Up to the page break below is from Kaplan. The rest is from Princeton Review and may
contain duplicate information.
\section{Learning\supdag}

\vocab{Learning} refers specifically to the way in which we
acquire new behaviors. A \vocab{stimulus} is defined as anything
to which an organism can respond. Responses to stimuli change
over time. Repeated exposure to the same stimulus may lead to a
decrease in response called \vocab{habituation}.
\vocab{Dishabituation} is the recovery of a response to a
stimulus after habituation has occurred. It is often noted when,
late in the habituation of a stimulus, a second stimulus is
presented. Dishabituation is temporary and always refers to
changes in response to the original stimulus,m not the new one.

\subsection{Associative Learning\supdag}

\vocab{Associative learning} is the creation of a pairing, or association,
either between two stimuli or between a behavior and a response.

\subsection{Classical Conditioning\supdag}

\vocab{Classical conditioning} is a type of associative learning where an
association between two unrelated stimuli is created. There are four words to be
concerned about: 
\begin{description}
  \item [\vocab{Unconditioned stimulus}] - stimulus that elicits an
    unconditioned response i.e. a loud bang
  \item [\vocab{Unconditioned response}] - response to the unconditioned
    stimulus i.e. recoiling from the loud bang
  \item [\vocab{Neutral stimulus}] - stimulus that doesn't produce a reflexive
    response i.e. a breeze
  \item [\vocab{Conditioned stimulus}] - a normally neutral stimulus that,
    through association, now causes a conditioned response
  \item [\vocab{Conditioned response}] - the associated response to the
    conditioned stimulus
\end{description}
Thus, classical conditioning is the process of taking advantage of a reflexive,
unconditioned stimulus to turn a neutral stimulus into a conditioned stimulus.
Also called \vocab{acquisition}. If conditioned stimulus is presented enough
without the unconditional stimulus then habituation occurs and leads to
\vocab{extinction}. However, a conditioned response may reappear post
extinction, albeit weakly, in a phenomenon called \vocab{spontaneous recovery}.

\vocab{Generalization} is the broadening effect where similar stimuli to the
conditioned stimuli also elicit the conditioned response. \vocab{Discrimination}
is the opposite process wherein similar stimuli are able to be distinguished.

\subsection{Operant Conditioning\supdag}

\vocab{Operant conditioning} links voluntary behaviors with consequences in an
effort to alter the frequency of those behaviors. It is associated with
\vocab{behaviorism}, the theory that all behaviors are conditioned. A summary of
the four possible relationships between stimulus and behavior are summarized in
\figref{Operant} below.
\begin{figure}[h]
  \centering
  \caption{Operant Conditioning summary}
  \includegraphics[scale=0.1]{Operant.jpg}
  \label{Operant}
\end{figure}

\subsubsection{Reinforcement\supdag}

\vocab{Reinforcement} is the process of increasing the likelihood that an
individual will perform a behavior. There are two categories of learning:
\begin{description}
  \item[\vocab{Positive reinforcers}] increase a behavior by adding a positive
    consequence or incentive following the desired behavior
  \item[\vocab{Negative reinforcers}] increase the frequency of a behavior by
    removing something unpleasant (relieving a headache with an aspirin will
    negatively reinforce you to take it again the next time)
\end{description}
\emph{Any reinforcement increases the likelihood that a behavior will be
performed.} Negative reinforcers are subdivided into two categories:
\begin{description}
  \item[\vocab{Escape learning}] behavior that reduces unpleasantness (taking
    aspirin to escape from pain of headache)
  \item[\vocab{Avoidance learning}] behavior that is meant to prevent the
    unpleasantness of something
\end{description}
\vocab{Primary reinforcers} are biological and are something that one naturally
responds to. A \vocab{secondary reinforcer (conditioned reinforcer)} is a
stimulus which normally would not be a reinforcer on its own.

\subsubsection{Punishment\supdag}

\vocab[]{Punishment} uses conditioning to reduce the occurrence of a behavior.
Two types:
\begin{description}
  \item[\vocab{Positive punishment}] \emph{adds} an unpleasant consequence to
    reduce a behavior
  \item[\vocab{Negative punishment}] \emph{removes} a stimulus to reduce a
    behavior
\end{description}
Positive and negative refer to the addition or removal of something don't get it
confused.

\subsubsection{Reinforcement Schedules}
Two factors: fixed or variable and ratio or interval.
\begin{description}
  \item[\vocab]{Fixed-ratio} schedules] reinforce a behavior after a specific
    number of performances of that behavior e.g. reward every third time
    performing a task.  \vocab{Continuous reinforcement} is when the reward
    occurs every time.
  \item[\vocab{Variable-ratio} schedules] reinforce a behavior after a varying
    number of performances of the behavior, but such that the average number of
    performances to receive a reward is relatively constant.
  \item[\vocab{Fixed-interval} schedules] reinforce the first instance of a
    behavior after a specified time period has elapsed. 
  \item[\vocab{Variable-interval} schedules] reinforce a behavior after a
    specified time period has elapsed. 
\end{description}
Variable ratio schedules teach the behavior the fastest followed by fixed ratio,
variable interval, then fixed interval. So variable versions are better for a
given medium.

\vocab[]{Shaping} is the process of rewarding increasingly specific behaviors.
So you can train complicated behaviors by building on a sequence of simple
behaviors i.e. make a full turn one can reward for 90 degree turns, then 180
degree turns, then 360 degree turns.

\subsection{Cognitive and Biological Factors in Associative Learning}

\vocab[]{Latent learning} is learning that occurs without a reward but is
demonstrated once a reward is introduced.  Like you learn something but don't
display that you learned it until you have an incentive to display it.
\vocab[]{Problem-solving} is another method of learning that steps outside the
standard behaviorist approach.  Not everything may be taught through operant
conditioning.  Animals have a predisposition, called \vocab[]{preparedness} to
learn certain behaviors due to natural abilities and instincts e.g. a bird
responds better to teaching them a pecking-based behavior as opposed to something
that is not immediately natural for a bird. Difficulty in overcoming instinctual
behaviors is called \vocab[]{instinctive drift}. 
\newpage

\setcounter{section}{0}

\section{Types of Learning}
\subsection{Nonassociative Learning}
\vocab[]{Nonassociative learning} occurs when an organism is repeatedly exposed
to one type of stimulus. Two types of learning:
\begin{enumerate}
  \item \vocab[]{Habituation} is the process of getting used to a stimulus e.g.
    after awhile you ignore the sound of tinnitus. \vocab[]{Dishabituation}
    occurs when the stimulus is removed and you return back to normal state of
    mind. If you are exposed to the stimulus again then you will react to it
    again.
  \item \vocab[]{Sensitization} is almost the opposite of habituation. Repeated
    stimulus causes an increased response.
\end{enumerate}

\subsection{Associative Learning}
\vocab[]{Associative learning} describes a process of learning in which one
event, object, or action is directly connected with another: classical
conditioning and operant conditioning.

\subsection{Classical Conditioning}
\vocab[]{Classical conditioning} is a process in which two stimuli are paired in
such a way that the response to one of the stimuli changes e.g. Pavlov's dogs.
Four parts to this that is well explained in the Kaplan portion above but will
be re-explained here in relation to Pavlov.
\begin{enumerate}
  \item \vocab[]{Neutral stimulus} (ringing of the bell) does not elicit any
    intrinsic response.
  \item \vocab[]{Unconditioned stimulus} (presentation of food) does elicit an
    \vocab{unconditioned response} (salivation).
  \item \vocab[]{Conditioned stimulus} (bell) an originally neutral stimulus
    that is paired with an unconditioned stimulus until it can produce the
    conditioned response without the unconditioned stimulus.
  \item \vocab[]{Conditioned response} (salivation) learned response to the
    conditioned stimulus. So bell causes salivation now.
\end{enumerate}

Five terms that pertain to development and maintenance of the conditioned
response:
\begin{enumerate}
  \item \vocab[]{Acquisition} the process of learning the conditioned response.
  \item \vocab[]{Extinction} when conditioned (bell) and unconditioned stimuli
    (bell) are no longer paired.
  \item \vocab[]{Spontaneous recovery} when the pair spontaneously occur again
    some time \textit{after} extinction has already occurred.
  \item \vocab[]{Generalization} other stimuli other than the conditioned
    stimuli elicit the response (e.g. doorbell instead of a bell)
  \item \vocab[]{Discrimination} opposite of generalization.
\end{enumerate}

\end{document}
