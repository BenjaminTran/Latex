\documentclass[../OChemReview.tex]{subfiles}

\begin{document}
\chapter{Structure and Stability}
\section{General Tools}
\subsection{Degree of Unsaturation}
Gives total number of $ \pi $ bonds and/or rings.  \[ U = \dfrac{2C + 2 + N - X
- H}{2} \]

\subsection{Hybridization\supddag}

Rule: Count number of bonds and lone pairs (each count as one) then fill s,p,d,f
according to
\[ sp^{3}d^{5}f^{7} \]

Note:
\begin{enumerate}
  \item \emph{carbocations are $ sp^{2} $ hybridized}
  \item double and triple bonds still count as one.
\end{enumerate}

Organic intermediates may stabilize by:\\ \textbf{inductive effects} - stabilize
charge through $ \sigma $ bonds and \\
\textbf{resonance effects} - stabilize by delocalization through $ \pi $ bonds.

The percentage of s and p character in an \chemfig{sp^{x}} bond is determined by
treating s and p equally so that \chemfig{sp^{2}} has 33\% s character and 67\%
p character. Remember that the more s character there is the stronger the bond.

\subsection{Inductive Effects}

\vocab{Electron-withdrawing groups} - more electronegative than C and stabilize
carbanions\\
\vocab{Electron-donating groups} - less electronegative than C and stabilize
carbocations\\
E.g. Trichloroacetic acid more stable than acetic acid due to EWG chlorines.

\begin{figure}[h]
  \centering
  \setatomsep{2em}
  \definesubmol\nobond{-[3,0.4,,,draw=none]}
  \schemestart
  \chemname{\chemfig{(-[5]HO)(=[2]O)-[7](-[1]Cl)(-[5]Cl)(-[7]Cl)}}{Trichloroacetic
  acid}\arrow(.mid east--.mid
  west){->}\chemfig{(-[5]^{-}O)(=[2]O)-[7](-[5]Cl)(-[1]Cl)(-[7]Cl)}
  \schemestop
\end{figure}

\subsection{Resonance Stabilization}

\textbf{Conjugated systems} contain three or more atoms that each bear a p
orbital aligned so they are parallel, creating possibility of delocalized
electrons. C.f. \figref{Allyl}. 

\begin{figure}[h]
  \centering
  \definesubmol\nobond{-[3,0.2,,,draw=none]}
  \chemfig{(!\nobond\oplus)-[1]=[7]}
  \caption{allyl cation}
  \label{Allyl}
\end{figure}
\newpage
Can push arrows to form resonance structures. Not all resonance structures
contribute equally. Thiophene in \figref{AnilineThiophene} also has resonance
structures but also the sulfur has a tricky hybridization. One of the lone pairs
actually is in an unhybridized orbital. Thus, the accurate hybridization is $
sp^{2} $ as opposed to $ sp^{3} $.

Aniline in \figref{AnilineThiophene} has unshared electron pair on the nitrogen.
Though it has resonance forms, they break the aromaticity of the ring. Since
aromaticity greatly reduces energy, these resonances are not favorable though
they still occur.
\begin{figure}[h]
  \centering
  \setatomsep{2em}
  \chemname{\chemfig{*6(-=-=(-\lewis{2:,N}H_{2})-=)}}{Aniline}
  \hspace{5em}
  \chemname{\chemfig{*5(-S-=-=)}}{Thiophene}
  \caption{}
  \label{AnilineThiophene}
\end{figure}

\noindent\fbox{
  \parbox{\textwidth}{Note: to determine whether or not to count the lone pair,
    ask if it can be used in resonance form.\\
    Yes $ \implies $ not in hybridized orbital so \underline{don't count}\\
    No $ \implies $ in hybridized orbital so \underline{count}.\\
    Idealized angles:
    \begin{enumerate}
      \item  $ sp $ 180
      \item $ sp^{2} $ 120 
      \item $ sp^{3} $ 109
    \end{enumerate}}
  }\\

  THE MORE STABLE A MOLECULE THE LESS REACTIVE. Three basic principles are:
  \begin{enumerate}
    \item No resonance if adjacent atom is $ sp^{3} $
    \item Resonance usually occurs when lone electrons are adjacent to a $ \pi $
      bond or unhybridized p orbital.
    \item Resonance structures of lowest energy are most important. Evaluation
      of resonance stability involves these criteria:
      \begin{enumerate}
        \item Octet rule is satisfied for all atoms (most important)
        \item minimize separation of charge(formal charge)
        \item With formal charges, resonance that have negatives on more
          electronegative atoms and positive on less electronegative atoms.
      \end{enumerate}
  \end{enumerate}

  \subsection{Acidity}

  \textbf{Bronsted-Lowry acid} can donate a proton froming conjugate base.
  Strength of the acid is determined by how stabilized the negative charge is on
  the conjugate base.

  \subsection{Electronegativity Effects }

  Compare propanol to propane. You have alkoxide ion vs. a primary carbanion
  which is highly unstable. Thus propanol is much more acidic than propane.

  \subsection{Resonance Effects}

  If the resulting conjugate base has more resonance structures than another
  conjugate base then it is more stable and therefore is the stronger acid.
  \begin{figure}[H]
    \centering
    \includegraphics[scale=0.2,frame]{acidSum.jpg}
    \caption{General Ranking or Relative acidities of Functional Groups on MCAT}
  \end{figure}
  

  \subsection{Inductive Effects (acidity)}

  EWG substituents can help stabilize the anion of the conjugate base. This
  effect depends on the distance from the anion however and its ability to
  withdraw (electronegativity) so consider different halides as well F > Cl >
  Br. C.f. Figure \ref{fig:order} for an example ordering of acidity.
  \begin{figure}[H]
    \centering
    \includegraphics[scale=0.2,frame]{Order.jpg}
    \caption{Order of Acidity}
    \label{fig:order}
  \end{figure}
  

  \subsection{General Rule for Organic Compound Acidity}

  \fbox{\parbox{\textwidth}{Strong acids $ > $ Sulfonic acids $ > $ Carboxylic
    acids $ > $ Phenols $ > $ Alcohols and waters $ > $ aldehydes and ketones\\
    ($ \alpha $ hydrogens) $ > $ sp hybridized C-H bonds $ > $ $ sp^{2} $
    hybridized C-H bonds $ > $ $ sp^{3} $ hybridized C-H bonds}}

  \subsection{Effects of Substituents on Acidity}

  EWG on phenols \underline{increase acidity}. Consider
  \emph{para}-nitro-phenol. It allows for additional resonance by pulling
  electrons onto the oxygen of the nitro group. This also works for the ortho
  position. Note: this does not occur for meta position. C.f. figure 4 for
  resonance structures.\\
  EDG on phenols have the opposite effect and \underline{decrease acidity} in
  the para and ortho positions since they can donate electrons to the ring
  forcing a carbanion adjacent to the negative charge.	

  \subsection{Nucleophiles and Electrophiles}

  \textbf{Nucleophiles} are species that have unshared pairs of electrons or $
  \pi $ bonds and, frequently, a negative (or partial negative) charge.
  \textbf{They are electron pair donors and thus lewis bases.} Some common
  examples are given on pg. 65. General trends for nucleophilicity:

  \begin{enumerate}
    \item increases as negative charge increases. E.g. $ NH_{2}^{-} > NH_{3}$ 
    \item increases going down a group. E.g. $ F^{-} < Cl^{-} < Br^{-} $
    \item increases going left in period.
  \end{enumerate}

  (2) is due to increasing polarizability of atoms down a group.\\
  (3) is due to electronegativity of nucleophilic. Less electronegative, more
  likely to donate the electrons.\\
  Note: only use these for atoms within the same period or group.\\

  \textbf{Electrophiles} are electron-deficient species. 
  \begin{enumerate}
    \item full or partial positive charge
    \item frequently incomplete octet
    \item electron acceptors and thus \textbf{Lewis acids}
  \end{enumerate}

  In all organic reactions nucleophiles react with electrophiles such that

  \[ E^{+} + :Nu^{-} \longrightarrow E\text{--}Nu \]

  \subsection{Leaving Groups}

  Leaving groups are more likely to dissociate from substrate if they are more
  stable in solution, thus resonance stabilized are the best leaving groups
  (tosylate, mesylate, and acetate) C.f. chapter 6 for further discussion of
  these groups. Weak bases also good (halogen ions); strong bases bad. \emph{In
  general, basicity decreases down a group thus leaving group ability
increases.} However, can make bad leaving group good by protonating a lone pair.
E.g. $ -OH^{-} $ to $ -OH_{2}^{+} $ becomes a good leaving group. This is why
many organic reactions are acid catalyzed.

  \subsection{Ring Strain}

  \textbf{Ring strain} occurs when bonds deviate from idealized angles.
  Cyclopropane and cyclobutane are highly strained and thus more reactive than
  other alkanes. E.g. may undergo hydrogenation reaction whereas
  cyclopentane(hexane) will not.
  \[ \text{cyclopropane(butane)} \xrightarrow[Ni]{H_{2}} \text{propane(butane)}
\]

  \section{Isomerism}

  \subsection{Constitutional Isomerism}

  Same molecular formula, different connections. E.g. n-pentane, isopentane,
  neopentane.

  \subsection{Conformational Isomerism}

  Same molecular formula and connectivity but different rotations about their $
  \sigma $ bonds. Can be \textbf{staggered(eclipsed)} and the $ \sigma $ bond on
  a carbon may bisect(align) with the sigma bond on adjacent carbon. Staggered
  vs. eclipsed conformation:
  \begin{enumerate}
    \item Less electronic repulsion
    \item more stable
    \item Less steric hinderance
  \end{enumerate}

  Recall size of substituents matter too. Put the largest ones apart from each
  other (\textbf{anti conformation}). Largest substituents aligned (\textbf{syn
  conformation}). In between (\textbf{gauche conformation}). \textbf{Note: anti
  conformation is not necessarily always the lowest energy.} Must account for
  possibility of forming intramolecular bonds such as hydrogen bonds i.e.
  1,2-ethandiol where gauche is better.\\
  Cyclopentane exists in puckered \textbf{envelope} form. If planar then all
  carbons in eclipsed conformation. \\
  Cyclohexane exists in \textbf{chair conformation}. There are two such
  conformations that can interconvert at 11 Kcal/mol. Intermediate states
  include half-chair (highest energy), twist boat (local minima), and boat.
  Hydrogens either in plane of carbons (\textbf{equatorial hydrogens}) or above
  or below plane of carbons (\textbf{axial hydrogens}). These hydrogens
  interchange after chair flip. Substituents break degeneracy of conformations.
  \underline{In general large substituents better in equatorial position.} In
  axial position they have steric clashes (\textbf{1,3-diaxial interactions}) 

  \subsection{Stereoisomerism}
  \textbf{Stereoisomer} is an umbrella term encompassing enantiomers and
  diastereomers.\\
  Same molecular formula and connectivity; differ in spatial arrangement of
  atoms. Any rotation cannot bring them to overlap and therefore each molecule
  expresses chirality.

  \subsection{Chirality}

  \textbf{Chiral} refers to any molecule that cannot be superimposed on its
  mirror image. An \textbf{achiral} molecule can be superimposed on its mirror
  image and contains a plane of symmetry. Carbon \textbf{chiral centers
  (stereocenter)} have four different groups attached to it and thus $ sp^{3} $
  hybridized. 

  \subsection{Absolute Configuration}

  Chiral centers may be assigned an \textbf{absolute configuration}. The rules
  are as follows:
  \begin{enumerate}
    \item Priority assigned to atoms directly attached to chiral center by
      increasing atomic number.
    \item If isotopes are present (deuterium) priority is assigned to heavier
      isotope.
    \item In the event of identical atoms, examine next connected atoms instead
      and determine based on the above rules
    \item Multiple bond is counted as two single bonds for both atoms involved
    \item After assignment, rotate molecule so lowest priority points directly
      away from viewer. Trace path from highest priority to second lowest
      priority. \underline{Clockwise - R | CCW - S}\\

  \end{enumerate}
  \textbf{Fischer Projection} - vertical lines assumed to go into the page and
  horizontal lines come out of the page.

  \subsection{Enantiomers\supddag}

  \textbf{Enantiomers} are non-superimposable mirror images, will always have
  opposite absolute configurations (R/S), and any molecule that contains chiral
  carbons and no internal planes of symmetry has an enantiomer. Usually
  enantiomeric pairs have the same physical properties except optical activity. 

  \subsection{Optical Activity}

  A molecule is \textbf{optically active} if it rotates the plane of polarized
  light. \textbf{Dextrorotatory}(+) rotates light clockwise.
  \textbf{Levorotatory} (-) rotates CCW. \textbf{Specific rotation} refers to
  magnitutde of rotation of light and is dependent upon
  \begin{enumerate}
    \item structure
    \item concentration
    \item path length of the light
  \end{enumerate}

  \emph{Enantiomeric pairs will rotate light with equal magnitude but in
  opposite directions.} A \textbf{racemic mixture} contains equal parts of an
  enantiomeric pair and thus does not rotate light at all.

  \subsection{Diastereomers}

  The possible number of stereoisomers is $ 2^{n} $ where n refers to the number
  of chiral centers. Consider the possible isomers 3-bromobutan-2-ol:

  \begin{figure}[h]
    \centering
    \includegraphics[scale=0.1]{4molecules.jpeg}
  \end{figure}

  (I,II) and (III,IV) are enantiomers and the remaining pairs are
  diastereomers. For two chiral centers, \emph{inverting ALL (in this case
  both) chiral centers results in enantiomers, inverting only one chiral results
in diastereomers}.

  \begin{figure}[h]
    \centering
    \includegraphics[scale=0.1]{diastereomersDiag.jpeg}
  \end{figure}

  Diastereomers may have dramatically different physical and chemical properties
  including:
  \begin{enumerate}
    \item melting points
    \item boiling points
    \item solubilities
    \item \emph{specific rotations}
  \end{enumerate}

  There is no relationship between specific rotations of diastereomers.

  \subsection{Resolution of Enantiomers}

  \textbf{Resolution} is the separation of a racemic mixture. Usu. done with
  enantiomerically pure chiral probes that associate with the enantiomeric pair
  producing two diasteromeric salts with different chemical and physical
  properties. Subsequent work up separates the enantiomeric pair. C.f. pg. 95
  for more details.

  \subsection{Epimers\supddag}

  \textbf{Epimers} are a subclass of diastereomers that differ in their absolute
  configuration at a single chiral center (only one stereocenter is inverted).
  Analogous to R and S configurations are the D and L configurations for
  epimers. When the hydroxyl group is on the right(left) of the epimeric carbon
  in the Fischer projection, the molecule is a D(L) sugar. Note: all epimers are
  diastereomers but not all diastereomers are epimers. An example is the
  following C-2 epimer in \figref{epimers} (D-ribose and D-arabinose):

  \begin{figure}[h]
    \centering
    \caption{C-2 Epimers}
    \chemname{\chemfig{(-[4]H)(-OH)(-[2](-[,1.13]OH)(-[2]CHO)(-[4]H))-[6](-[4]H)(-\textcolor{red}{OH})-[6]CH_{2}OH}}{D-ribose}
    \hspace{5em}
    \chemname{\chemfig{(-[4]H)(-OH)(-[2](-H)(-[2]CHO)(-[4]HO))-[6](-[4]H)(-\textcolor{red}{OH})-[6]CH_{2}OH}}{D-arabinose}
    \label{epimers}
  \end{figure}
  \newpage
  \subsection{Anomers}

  \textbf{Anomers} are epimers that form from a ring closure. MCAT only focuses
  on sugar anomers. Consider D-glucose. Open-chain glucose cyclizes when C-5
  hydroxyl group attacks the C=O carbon of C-1. C-1 goes from three different
  substituents to four and thus becomes a chiral center, \textbf{anomeric
  center} in this case. It now may either have the hydroxyl group down(up) $
  \alpha(\beta) $ Note: Leah4Sci video on going from Fischer to Haworth to
  Chair. Rules
  \begin{enumerate}
    \item Right side of Fischer goes down
    \item Left side of Fischer goes up
    \item C-6 goes opposite of the -OH on C-5
  \end{enumerate}

  \subsection{Meso Compounds}

  A \textbf{meso compound} is a pair of molecules that have chiral centers and
  directly superimposable mirror images. \emph{They have an internal plane of
  symmetry.} They do not contribute to the number of stereoisomers a compound 
  has.

  \subsection{Geometric Isomers}

  \textbf{Geometric isomers} are diastereomers that differ in orientation of
  substituents around a ring or double bond. Priority is assigned in a similar
  manner. Label rules for cyclic are:
  \begin{enumerate}
    \item highest priority same side - \emph{cis}
    \item on opposite side - \emph{trans}
  \end{enumerate}

  For double bonds we follow same priority assignment rules and have that:
  \begin{enumerate}
    \item same-side Z (cis)
    \item opposite-side E (trans)
  \end{enumerate}
  \newpage
  \subsection{Summary of Isomers}

  \begin{figure}[h]
    \centering
    \includegraphics[scale=0.2]{isoSum.jpeg}
  \end{figure}	
\end{document}
