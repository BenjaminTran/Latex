% !TeX root = GChemChap7.tex
\documentclass[../GChemReview.tex]{subfiles}

\begin{document}
	\chapter{Phases}
	
	\section{Physical Changes}
	
	A physical change affects only the intermolecular forces between molecules i.e. no forming or breaking of bonds.
	
	\subsection{Phase Transitions}
	
	Temperature is a measure of the internal kinetic energy of the molecules. It is also related to entropy. Some key words:
	\begin{enumerate}
		\item \textbf{Fusion} - melting
		\item Vaporization - boiling
		\item Crystallization - freezing
		\item Condensation - gas to liquid
		\item \textbf{Sublimation} - solid to gas
		\item \textbf{Deposition} - gas to solid
	\end{enumerate}
	
	\section{Heats of Phase Changes}
	
	Energy required to complete a transition is called the heat of transition. The \textbf{heat of fusion} is the heat necessary to melt a solid. The \textbf{heat of vaporization} is self explanatory. The amount of heat is dependent on the substance and the amount. The required heat is given by:
	\begin{equation}
		q = n\times\Delta H_{\text{phase change}}
	\end{equation}
	A \textbf{calorie} is defined as the amount of heat required to raise the temperature of 1 gram of water by $ 1^{\circ} $C. Conversion to joules is $ 1 \text{ cal } \approx 4.2 \text{ J} $.
	
	\section{Calorimetry}
	
	\emph{When a substance absorbs or releases heat: either its temperature changes \emph{or} ir will undergo a phase change \emph{but not both at the same time}}. The amount of heat absorbed or released by a sample is given by:
	\begin{equation}
		q = mc\Delta T
	\end{equation}
	where:
	\begin{itemize}
		\item q = heat added or released
		\item m = mass
		\item c = specific heat of the substance
		\item $ \Delta $T = temperature change
		\item C = mc is the \textbf{heat capacity}
	\end{itemize}
	Thus, the temperature change of an object due to the addition of some amount of heat, q, is inversely proportional to it's specific heat.\\
	Note that:
	\begin{enumerate}
		\item c is dependent upon the phase
		\item Doesn't matter if change in temperature is C or K since 1K = 1C 
	\end{enumerate}
	
	\section{Phase Transition Diagram}
	
	A plot of the temperature of the sample versus the amount of heat absorbed. C.f. below:
	
	\begin{figure}[h]
		\centering
		\includegraphics[scale=0.1]{PhaseTranDiag.jpg}
	\end{figure}
	
	The greater the value of the heat of transition the longer the flat lines during phase transitions. N.b. heat of vaporiation always greater than heat of fusion. Slopes of the lines are given by $ C^{-1} $
	
	\subsection{Phase Diagrams}
	
	The phase a substance is at is also dependent upon the pressure. This is illustrated in a phase diagram c.f. below.
	
	\begin{figure}[h]
		\centering
		\includegraphics[scale=0.1]{PhaseDiag.jpg}
	\end{figure}
	
	The boundary lines represent points at which the two phases are in equilibrium. The \textbf{triple point} is the temperature and pressure at which all three phases exist simultaneously in equilibrium. The \textbf{critical point} is the end of the liquid-gas boundary line and after this point a substance is a supercritical fluid and will display properties of liquids and gases (high density and low viscosity).
	
	\subsection{The Phase Diagram for Water}
	
	Water is special. Since the liquid form is denser than solid, the slope of the solid-liquid boundary is negative as opposed to most substances which have a positive slope (c.f. above). Thus, increasing pressure at a fixed temperature on water will certainly force it into a liquid state but never a solid state.
	
	\begin{figure}[h]
		\centering
		\includegraphics[scale=0.2]{WaterPhase.jpg}
	\end{figure}
	
	\end{document}