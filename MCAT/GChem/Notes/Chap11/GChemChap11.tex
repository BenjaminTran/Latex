\documentclass[../GChemReview.tex]{subfiles}

\begin{document}
\chapter{Acids and Bases}

\section{Definitions}
Three definitions of acids and bases

\subsection{Arrhenius Acids and Bases}

Oldest and least commonly used definition:

\centering
\hfil \\
\begin{varwidth}{\textwidth}
  \begin{enumerate}
    \item[] \emph{Acids ionize in water to produce hydrogen ions}
    \item[] \emph{Bases ionize in water to produce hydroxide ions}
  \end{enumerate}
\end{varwidth}
\\
\flushleft
Remember that \chemfig{H^{+}} does not exist by itself but rather denotes
\chemfig{H_{3}O^{+}}.

\subsection{Bronsted-Lowry Acids and Bases}
The definitions are:

\centering
\hfil \\
\begin{varwidth}{\textwidth}
  \begin{enumerate}
    \item[] \emph{Acids are proton donors}
    \item[] \emph{Bases are proton acceptors}
  \end{enumerate}
\end{varwidth}
\\
\flushleft
Then in the following reaction

\begin{center}
  \schemestart
  \chemfig{H_{2}CO_{3} \+ H_{2}O}\arrow{<=>}\chemfig{H_{3}O^{+} \+ HCO_{3}^{-}}
  \schemestop
\end{center}
The acids are \chem{H_{2}CO_{3}} and \chem{H_{3}O^{+}} and the bases are the rest.

\subsection{Lewis Acids and Bases}

The definitions are:

\centering
\hfil \\
\begin{varwidth}{\textwidth}
  \begin{enumerate}
    \item[] \emph{Lewis acids are electron pair acceptors}
    \item[] \emph{Lewis bases are electron pair donors}
  \end{enumerate}
\end{varwidth}
\\
\flushleft

Then in the following reaction:

\begin{center}
  \schemestart
  \chemfig{AlCl_{3} \+ H_{2}O}\arrow{<=>}\chem{(AlCl_{3}OH)^{-}}\chemfig{\+ H^{+}}
  \schemestop
\end{center}
Thus, \chem{AlCl_{3} \text{and} H^{+}} are acids since they are electron acceptors.

\section{Conjugate Acids and Bases}

When a Bronsted-Lowry acid donates an \chem{H^{+}} the remaining is called a
\vocab{conjugate base}. In the reverse situation it is called a \vocab{conjugate
acid}.

\section{The Strengths of Acids and Bases}

From the general chemical equation:
\begin{center}
  \schemestart
  \chemfig{HA \+ H_{2}O}\arrow{<=>}\chemfig{H_{3}O^{+} \+ + A^{-}}
  \schemestop
\end{center}
We can quantify the strength of an acid or base with the \vocab{acid-dissociation constant}:
\begin{equation}
  \boxed{K_{a} = \dfrac{[\chem{H_{3}O^{+}}][\chem{A^{-}}]}{[\chem{HA}]}}
\end{equation}
If $ K_{a} > 1 $ then we say that it is a \vocab{strong acid} and if $ K_{a} < 1
$ it is a \vocab{weak acid}. When comparing acids \textcolor{red}{The larger the
  $ K_{a} $ value the stronger the acid; the smaller it is the weaker the acid}.
  The following table gives the strong acids:

\begin{center}
  \begin{tabularx}{0.4\textwidth}{Xc}
    \toprule
    Hydroiodic acid & HI \\ 
    Hydrobromic acid & HBr \\ 
    Hydrochloric acid & HCl \\ 
    Perchloric acid & HClO$_{4}$ \\ 
    Sulfuric acid & \chem{H_{2}SO_{4}} \\ 
    Nitric acid & \chem{HNO_{3}} \\ 
    \bottomrule
  \end{tabularx}
\end{center}

If an acid is not on this list then it may be treated as weak on the MCAT. Note
that HF is not a strong acid. This is due to the fact that \chem{F^{-}} has a
very small radius and is therefore not considered stable as a conjugate base.

Equivalently, for bases we have the \vocab{base-dissociation constant}:
\begin{equation}
  \boxed{K_{b} = \dfrac{[\chem{HB^{+}}][\chem{OH^{-}}]}{[\chem{B}]}}
\end{equation}
\textcolor{Blue1}{The larger the $ K_{b} $ value, the stronger the base, the
smaller it is the weaker the base.} On the MCAT the common strong bases are
listed below: 
\begin{table}[H]
  \centering
  \setlength{\aboverulesep}{0pt}
  \setlength{\belowrulesep}{0pt}
  \begin{tabularx}{0.8\textwidth}{Xp{15em}}
    \toprule
    \rowcolor{fu-blue!20} \textbf{Common Strong Bases} & \textbf{Examples}\\
    \midrule
    Group 1 hydroxides & \chem{NaOH} \\
    Group 1 oxides & \chem{Li_{2}O} \\
    Some group 2 hydroxides & \chem{Ba(OH)_{2}}, \chem{Sr(OH)_{2}},
    \chem{Ca(OH)_{2}} \\ 
    Metal amides & \chem{NaNH_{2}} \\
    \bottomrule
  \end{tabularx}
\end{table}

\subsection{The Relative Strengths of Conjugate Acid-Base Pairs}
The strength of a conjugate species is inversely proportional to the strength of
its parent e.g. the conjugate acid of a strong base is a weak acid and the
conjugate base of a strong acid is a weak base. For a weak acid or base the
conjugate species is also, in general, weak but keep in mind the inverse
proportionality.

\subsection{Amphoteric Substances}
When a substance can act as either an acid or base, it is called
\vocab[]{amphoteric}. The conjugate base of a weak polyprotic acid is always
amphoteric. Polyprotic acids become weaker the more you deprotonate.

\section{The Ion-Product Constant of Water} 



\end{document}
