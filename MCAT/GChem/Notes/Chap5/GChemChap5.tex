% !TeX root = GChemChap5.tex
\documentclass[../GChemReview.tex]{subfiles}

\begin{document}

  \chapter{Bonding and Intermolecular Forces}

  \section{Lewis Dot Structures}

  Model valence electrons. Electrons in \emph{d} subshells are not considered
  valence electrons since valence electrons are in the highest \emph{n} level.

  \subsection{Formal Charge}

  To determine correct Lewis structure in degenerate cases use \vocab{formal
  charge}. Formal charge for a given atom is 
  \begin{equation}
    F_{c} = V - \dfrac{1}{2}B - L
  \end{equation}

  Where
  \begin{enumerate}
    \item V valence electrons
    \item B bonding electrons
    \item L Lone paired electrons (count individual electrons)
  \end{enumerate}

  Best lewis structures have octets and zero formal charge of all the atoms. If
  both are not possible then the one that minimizes the magnitude of the formal
  charge is the best.

  \subsection{Resonance}

  If a molecule has non-equivalent resonance structures then the resonance
  hybrid is a weighted average of them as shown with formaldehyde:

        %	\begin{figure}[h]
        %		\centering
        %		\schemestart
        %		\chemname{\chemfig{C(-[5]H)(-[7]H)=[2]\lewis{4:0:,O}}}{major-all atoms\\ have octets and no\\ formal charge} 
        %		\arrow{<->}
        %		
        %		\chemname{\chemfig{\chembelow{C}{+}(-[5]H)(-[7]H)-[2]\lewis{2:4:0:,O}}}{a}
        %		\schemestop
        %	\end{figure}

  \section{Bond Length and Bond Dissociation Energy}

  \vocab{Bond dissociation energy} (BDE) is the energy required to break a bond
  \emph{homolytically} (when each fragment receives a single electron whereas,
  we are normally concerned with \emph{heterolytic} bond cleavage). We will only
  consider homolytic vond cleavage. For similar bonds (as in compare C-C bonds
  only to other C-C bonds), \emph{the higher the bond order, the shorter and
  stronger the bond.}

  \section{Types of Bonds}

  \subsection{Covalent Bonds}

  \vocab{Covalent bond} is formed between atoms when each contributes one or
  more of its unpaired valence electrons. 

  \subsection{Polarity of Covalent Bonds}

  Two atoms bonded together with different electronegativities will have a polar
  bond and thus a dipole moment. Consider HF (polar bond) and FF (nonpolar
  bond).

  \subsection{Coordinate Covalent Bonds}

  When one atom donates \emph{both} of the shared electrons in a bond. E.g.
  \chemfig{BF_{3}} and \chemfig{NH_{3}}. Note: the atom that donates the
  electron pair is by definition a Lewis base. When the coordinate bond breaks
  the electrons return to its original owner.

  \subsection{Ionic Bonds}

  Consider NaCl where sodium \emph{gives} an electron to Cl and thus the
  molecule is held together by the electrostatic attraction of the ions. Since
  the bond strength is dependent on the charge difference, we can compare bond
  strengths based on this fact.

  \section{VSEPR Theory}

  One rule: \emph{Since electrons repel one another, electron pairs, whether
  bonding or nonbonding, attempt to move as far apart as possible.} The total
  number of electron groups on the central atom of a molecule determines its
  bond angles and \emph{orbital geometry}. Note: double and triple bonds count
  only as one electron group.\\ The \vocab{molecular geometry} refers to the
  shape the atoms make (excluding the lone pairs).

  \section{Hybridization}

  The percentage of s and p character in an \chemfig{sp^{x}} bond is determined
  by treating s and p equally so that \chemfig{sp^{2}} has 33\% s character and
  67\% p character.

  \subsection{Sigma Bonds}

  A $ \sigma $ bond consists of two electrons that are localized between two
  nuclei. \emph{Remember that an \chemfig{sp^{3}} carbon atom has 4
  \chemfig{sp^{3}} hybrid orbitals.}

  \subsection{Pi Bonds}

  A $ \pi $ bond is composed of two electrons that are localized to the region
  that lies on opposite sides of the nodal plane formed by the two bonded nuclei
  and immediately adjacent atoms, c.f. below:

  \begin{figure}[h]
    \centering
    \includegraphics[scale=0.2]{PiBond.jpg}
  \end{figure}

  The bond is formed by the proper, parallel, side-to-side alignment of two
  unhybridized p orbitals.

  \section{Molecular Polarity}

  A molecule is polar if it contains an uneven distribution of polar bonds (usu.
  due to electron lone-pairs). It may be nonpolar if it has a symmetric
  distribution of polar bonds.

  \section{Intermolecular Forces}

  Liquids and solids held together by intermolecular forces such as
  dipole-dipole forces and London dispersion forces. A hydrogen bond is the
  strongest dipole-dipole force. \vocab{London dispersion forces} occur when
  an instantaneous dipole of a nonpolar molecule induces a dipole in another
  nonpolar molecule. They are very weak and transient. Molecules larger $
  \implies $ more electrons $ \implies $ more polarizability $ \implies
  $stronger dispersion forces.\\
  Recall this leads to impact on physical properties such as:
  \begin{enumerate}
    \item greater melting points
    \item greater boiling points
    \item greater viscosities
    \item lower vapor pressures
  \end{enumerate}	
  than similar molecules with weaker intermolecular forces. Consider gas state
  of \chemfig{Cl_{2}} and liquid state of \chemfig{Br_{2}}.\\ Note: \vocab{Van
  der Waals forces} encompasses dipole forces, hydrogen bonding, and London
  forces.

  \subsection{Hydrogen Bonding}

  \fbox{\parbox{\textwidth}{Two criteria to fulfill:

    \begin{enumerate}
      \item a molecule must have a covalent bond between H and N, O, or F
      \item another molecule must have a lone pair of electrons on an N, O, or F
        atom.  
    \end{enumerate}

  }}

  can impact physical properties as mentioned above.

  \subsection{Vapor Pressure}

  \vocab{Vapor pressure} is the pressure exerted by the gaseuos phase of a
  liquid that evaporated from the exposed surface of the liquid. Also
  proportionately dependent on temperature of the substance. Thus, vapor
  pressure is indirectly related to the boiling point of a substance.

  \section{Types of Solids}

  \subsection{Ionic Solids}

  A solid held together by the electrostatic attraction between cations and
  anions in a lattice structure. These bonds are strong. However, the solid is
  brittle since a shift in the lattice can place like charges near each other.
  Almost all are solid at room temperature.

  \subsection{Network Solids}

  A solid held together by covalent bonds in a lattice. These are especially
  strong solids e.g. diamond.

  \subsection{Metallic Solids}

  Covalently bound lattice of metal nuclei and inner shell electrons
  surrounded by a sea of electrons. At least one valence electron is free
  of any one atom and is able to move throughout the lattice. These are
  called \vocab{conduction electrons}.

  \subsection{Molecular Solids}
  Lattice in which every lattice point is a molecule. Held together by either
  hydrogen bonds, dipole-dipole forces, or London dispersion forces. Since these
  are weak, typically are liquid or gases at room temperature.  
\end{document}
