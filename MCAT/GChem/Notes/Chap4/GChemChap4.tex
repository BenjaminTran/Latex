\documentclass[../GChemReview.tex]{subfiles}

\begin{document}
\chapter{Chemistry Fundamentals}
\section{Percentage Composition by Mass}

Used to determine empirical formula of compound. Procedure:
\begin{enumerate}
  \item Assume 100g
  \item Determine moles of each element from percentage, gives
  (mole number) \item Divide each mole number by smallest mole
    number
  \item To obtain molecular formula divide molecular weight by
    empirical weight
\end{enumerate}

\begin{problem*}
  Consider 36.5\% Na, 25.4\% S, and 38.1\% O.
\end{problem*}
\begin{solution}
  \begin{align*}
    Na &= \dfrac{36.5}{23} = 1.6 \\
    S &= \dfrac{25.4}{32} = 0.79 \\
    O &= \dfrac{38.1}{26} = 2.4 
  \end{align*}
  Divide by 0.79 gives empirical formula:
  \[ Na_{2}SO_{3} \]
\end{solution}

\section{Equivalent Weight\supdag}

The equivalent weight of a compound is the mass that provides one mole of the
particle of interest. E.g. want one mole of protons, \chemfig{H_{2}SO_{4}} can
yield two moles of protons thus the gram equivalent weight is half its molecular
weight. The equation is simply:

\begin{equation}
  \text{Gram equivalent weight} = \dfrac{\text{Molar mass}}{n}
\end{equation}
where n refers to the number of moles of interest produced or consumed
per molecule of the compound in the reaction: termed an \vocab{equivalent}.
Thus, for \chemfig{H_{2}SO_{4}} n = 2.

\section{Concentration}

\vocab{Normality} is defined by:
\begin{equation}
  \text{Molarity} = \dfrac{\text{Normality}}{n}
\end{equation}

where n is defined in the previous section. It is most commonly used in
reference to acids.
\[ \boxed{\text{molarity} \; (M) = \dfrac{\text{moles solute}}{\text{liters
solution}}}   \]

Mole fraction, $ X_{s} $, is the fraction of moles of a given substance, S,
relative to the total moles in solution:
\[\boxed{X_{s} = \dfrac{\text{moles of substance S}}{\text{total moles in
solution}}}\]

\section{Types of Chemical Reactions\supdag}

\subsubsection{Combination Reactions\supdag}

Combination reaction has two or more reactants forming one product:

\begin{equation}
  \chemfig{A + B + \cdots}\rightarrow\chemfig{C}
\end{equation}

\subsubsection{Decomposition Reaction\supdag}

Decomposition reaction has a single reactant break down into two or more
products.

\begin{equation}
  \chemfig{C}\rightarrow\chemfig{A + B + \cdots}
\end{equation}

\subsubsection{Combustion Reaction\supdag}

A combustion reaction is a special type of reaction that involves a fuel usually
a hydrocarbon and an oxidant (normally oxygen). E.g.

\[ \chemfig{CH_{4} + 2O_{2}}\rightarrow\chemfig{CO_{2} + 2H_{2}0} \]

\subsubsection{Single-Displacement Reaction\supdag}

A single-displacement reaction occurs when an atom or ion in a compound is
replaced by an atom or ion of another element. E.g.

\[ 	\chemfig{Cu + AgNO_{3}}\rightarrow\chemfig{Ag + CuNO_{3}} \]

These reactions are also classified as oxidation-reduction reactions. In the
example above Ag has an initial oxidation state of +1 but afterwards it gains
one electron and copper loses an electron.

\subsubsection{Double-Displacement Reaction\supdag}

Double-displacement reactions are also called \vocab{metathesis reactions} where
elements from two different compounds swap places with each other to form two
new compounds. E.g.

\[ \chemfig{CaCl_{2} + 2AgNO_{3}}\rightarrow\chemfig{Ca\del{NO_{3}}_{2} + 2AgCl}
\]

This usually occurs with a change of phase in one of the products.

\subsubsection{Neutralization Reaction\supdag}

Neutralization reactions are a specific type of double-displacement reaction in
which an acid reacts with a base to produce a salt (and usually water). E.g.

\[ \chemfig{HCl + NaOH}\rightarrow\chemfig{NaCl + H_{2}O} \]


\section{Chemical Equations and Stoichiometric Coefficients}

Number of atoms on RHS must equal number on LHS. When balancing equations, touch
molecules containing single elements last.

\section{Stoichiometric Relationships in Balanced Reactions}

Stoichiometric coefficients give the relative amounts of reactants that combine
and relative amounts of products formed. One can just use dimensional analysis
to figure out the problems.

\section{Limiting Reagent}

To solve these problems compare the stoichiometric ratio to the experimental
ratio. If the experimental ratio is smaller than the stoichiometric ratio then
the numerator is the limiting reagent. If the experimental ratio is larger than
the stoichiometric ratio then the denominator is the limiting reagent.\\
\begin{problem*}
  Consider the following reaction:
  \[ 2 ZnS + 3O_{2} \rightarrow 2 ZnO + 2 SO_{2} \]
  If 97.5 grams of ZnS undergoes this reaction with 32 grams of $ O_{2} $ what
  will be the limiting reagent?
\end{problem*}

\begin{solution}
  Stoichiometric ratio, S:
  \[ S = \dfrac{2 \text{ mol ZnS}}{3 \text{ mol }  O_{2}} = \dfrac{2}{3}\]
  Experimental ratio, E:
  \[ E = \dfrac{1 \text{ mol ZnS}}{1 \text{ mol } O_{2}} = 1\]
  Thus, $ E > S $ then $ O_{2} $ is the limiting reagent.
\end{solution}

\section{Yield\supdag}

The \vocab{theoretical yield} is the maximum amount of product that can be
generated. The \vocab{actual yield} is the amount of product actually obtained.
The ratio is the percent yield defined as:

\begin{equation}
  \text{Percent yield} = \dfrac{\text{Actual yield}}{\text{Theoretical
  yield}}\times 100\%
\end{equation}

\section{Ions\supdag}

\subsection{Nomenclature of Cations and Anions\supdag}

\centering
\begin{enumerate}
  \item For elements that can form more than one positive ion, the charge is
    indicated by roman numerals E.g.  
    \[ \chemfig{Fe^{2+}} \qquad Iron(II) \]
  \item Older, less commonly used is \textbf{-ous} or \textbf{-ic} for lesser
    and greater charge e.g. 
    \[ \chemfig{Fe^{2+}} \qquad \text{Ferrous} \]
    \[ \chemfig{Fe^{3+}} \qquad \text{Ferric} \]
  \item Monatomic anions had \textbf{-ide} added at the end e.g.
    \[ \chemfig{H^{-}} \qquad \text{Hydride} \]
  \item Polyatomic anions containing oxygen are called \vocab{oxyanions}. When
    an element forms two oxyanions, less oxygen ends in \textbf{-ite}, more
    oxygen ends in \textbf{-ate} e.g.
    \[ \chemfig{NO^{-}_{2}} \qquad \text{Nitrite} \]
    \[ \chemfig{NO^{-}_{3}} \qquad \text{Nitrate} \]
  \item For more than two possible species then \textbf{hypo-} and \textbf{per-}
    prefixes are added
    \begin{align*}
      \chemfig{ClO^{-}} \qquad &\text{Hypochlorite} \\
      \chemfig{ClO^{-}_{2}} \qquad &\text{Chlorite} \\
      \chemfig{ClO^{-}_{3}} \qquad &\text{Chlorate} \\
      \chemfig{ClO_{4}^{-}} \qquad &\text{Perchlorate}
    \end{align*}
  \item Polyatomic anions often gain one or more \chemfig{H^{+}} ions. Thus, add
    \textbf{hydrogen} or \textbf{dihydrogen} to the front of the name. Also, can
    use \textbf{bi-} prefix for indicating the addition of a single hydrogen
    ion. E.g. bicarbonate aka Hydrogen carbonate 
  \item Other common polyatomic ions
    \begin{figure}[h]
      \centering
      \includegraphics[scale=0.2]{polyatomic.jpg}
    \end{figure}
\end{enumerate}

\flushleft
\section{Oxidation States}

Oxidation state is meant to indicate how the atom's "ownership" of its
valence electrons changes when it forms a compound. \emph{\textbf{Giving
up} ownership of an electron results in a more \vocab{positive
oxidation state}; \vocab{accepting} ownership results in a more \vocab{negative oxidation state}.}\\

\noindent\fbox{
  \parbox{\textwidth}{\textbf{Rules for assigning oxidation states: (Rules higher in the list take precedence over those lower in the list.)
    \begin{enumerate}
      \item The oxidation state of any element in its standard state is 0.
      \item The sum of the oxidation states of the atoms in a neutral molecule
        must always be 0, and the sum of the oxidation states of the atoms in an
        ion must always equal the ion's charge.  
      \item Group 1 metals have a +1
        oxidation state, and Group 2 metals have a +2 oxidation state.
      \item Flourine has a -1 oxidation state.
      \item Hydrogen has a +1 oxidation state when bonded to something more
        electronegative than carbon, a -1 oxidation state when bonded to an atom
        less electronegative than carbon, and a 0 oxidation state when bonded to
        carbon.
      \item Oxygen has a -2 oxidation state
      \item The rest of the halogens have a -1 oxidation state, and the atoms of
        the oxygen family have a -2 oxidation state.
    \end{enumerate}
    Note: exception to rule 6 in peroxides (such as $ H_{2}O_{2} $ or $
    Na_{2}O_{2} $), oxygen is in a -1 oxidation state.} For more detailed rules
    see
    \url{http://chemed.chem.purdue.edu/genchem/topicreview/bp/ch2/oxnumb.html}}
  }\\

  \hl{The order of electronegativities of some elements from most to least can
    be remembered by \textbf{FONClBrISCH.}} Hence if H bonds to anything before
    C then its oxidation state is +1 and bonds to anything else not in this list
    gives a -1 oxidation state.\\ \hl{Metals will never assume a negative
    oxidation state.} E.g. iron has an oxidation number of +2 in $ FeCl_{2} $
    but an oxidation number of +3 in $ FeCl_{3} $. Oxidation number of a
    transition metal is given by the roman numeral in the name of the compound.
    Therefore, $ FeCl_{2} $ is iron(II) chloride.

  \begin{problem*}
    Find the oxidation number of manganese in $KMnO_{4}$.
  \end{problem*}

  \begin{solution}
    By Rule 3 K is +1 and by Rule 6 O is -2 thus to satisfy Rule 2 Mn must have
    an oxidation of +7.
  \end{solution}

  \chapter{Atomic Structure and Periodic Trends}

  \section{Atoms}

  Atomic number, Z, is the number of protons in the nucleus and may appear as
  the subscript before the element. Mass number, A, is the mass of the nucleus
  and appears as a superscript, i.e. Beryllium is $ _{4}^{9} $Be.

  \section{Isotopes}

  \vocab{Isotopes} differ in their number of neutrons. Isotopes always have the
  same atomic number but different mass numbers.

  \subsection{Atomic Weight}

  The atomic weight of an element is a weighted average of the masses of its
  naturally occurring isotopes. E.g. Boron-10, 20\%, mass is 10.013 amu and
  boron-11, 80\%, mass is 11.009 amu. Therefore, atomic weight is 
  \[ 0.20*10.013 + 0.80*11.009  = 10.810 amu \] 

  \section{Ions}

  \vocab{Anions} are negatively charged ions. \vocab{Cations} are positively
  charged ions. The charge is a superscript following the element symbol.

  \section{Nuclear Stability and Radioactivity}

  Nucleons are held together by the strong force. Unstable nuclei are called
  \vocab{radioactive} and undergo radioactive decay to become more stable. Three
  types of interest
  \begin{enumerate}
    \item $ \alpha $
    \item $ \beta $
    \item $ \gamma $
  \end{enumerate}

  Nucleus that undergoes decay is called the \vocab{parent} and the result
  is the \vocab{daughter}.

  \subsection{Alpha Decay}

  \vocab{Alpha particle} emission denoted by $ ^{4}_{2} \alpha$ consists of 2
  protons and 2 neutrons. The energy is quickly lost and thus does not travel or
  penetrate far (skin can stop it).

  \subsection{Beta Decay}

  Three types of beta decay, $ \beta^{-}, \beta^{+}, $ and electron capture.
  This is the conversion of a neutron to a proton (and some other particles) via
  the weak nuclear force. It is more penetrative than an alpha decay.

  \subsubsection{$ \beta^{-} $ Decay}

  When unstable nucleus contains \hl{too many neutrons}, converts neutron to
  proton and $ \beta^{-} $ (aka. electron) which is ejected. Daughter has 1
  greater atomic number but same mass number as parent. E.g.

  \[ ^{14}_{6}C \rightarrow \; ^{14}_{7}N + \; ^{0}_{-1}\beta \]

  \subsubsection{$ \beta^{+} $ Decay (Positron emission)}

  When unstable nucleus contains \hl{too few neutrons}, converts proton into
  neutron and positron which is ejected. Daughter nucleus 1 less atomic number
  but same mass number. E.g.

  \[ ^{18}_{9}F \rightarrow \; ^{18}_{8}O + \; ^{0}_{1}\beta \]

  \subsubsection{Electron Capture}

  Allows nucleus to \hl{increase number of neutrons} by capturing electron from
  n=1 shell to combine with proton to produce a neutron. Same effect as positron
  emission on atomic and mass numbers. E.g.

  \[ ^{51}_{24}Cr + \; ^{0}_{-1}e^{-} \rightarrow \; ^{51}_{23}V\]

  \subsection{Gamma Decay}

  Nucleus in an excited energy state (usu. after alpha or beta decay) returns to
  ground state by photon emission. Called gamma photon, these are high energy
  and high frequency and can penetrate matter most effectively. There is no
  change to the mass or atomic numbers. E.g.

  $$ ^{31}_{14}Si \xrightarrow{\beta^{-} decay} \; ^{31}_{15}P^{*}
  \xrightarrow{\gamma \; decay} \; ^{31}_{15}P + \; ^{0}_{0}\gamma$$


  \subsection{Half Life}

  \vocab{Half-life} is the time it takes for one-half of some substance to
  decay. The equation is officially 
  \[ N = N_{0}e^{-kt} \quad \text{where } k = \tfrac{ln(2)}{t_{1/2}}\]
  but may be rewritten as
  \[ \boxed{N = N_{0}(\tfrac{1}{2})^{t/t_{1/2}}} \]

  \section{Atomic Structure}

  \subsection{Emission Spectra}

  The energy of emitted photons is given by
  \[ E_{photon} = hf = h\dfrac{c}{\lambda} \]

  \subsection{Bohr Model}

  Bohr orbits are quantized. Higher n means higher energy electrons. Electrons
  can drop(jump)to a lower(higher) level by emitting(absorbing) a photon with
  energy equal to the energy difference between levels. The energy of a given
  level, n, is given by 
  \[ E_{n} = \dfrac{-2.178 \times 10^{-18 }J}{n^{2}} = \dfrac{-13.6 \;
  eV}{n^{2}}\]

  \subsection{Quantum Model of the Atom}

  Quantization of the electron is described by four quantum numbers:
  \begin{enumerate}
    \item shell
    \item subshell
    \item orbital
    \item spin
  \end{enumerate}

  \subsubsection{Energy Shell}

  Denoted by n, is analogous to the circular orbits of the Bohr model of the
  atom. Higher shell = higher energy.

  \subsubsection{Energy Subshell}

  \vocab{Orbital} is the three-dimensional probability distribution of the
  electron around the nucleus. \vocab{Subshells} are comprised of one or more
  orbitals denoted by (s, p, d, or f) that describe the shape and energy of the
  orbital. \hl{Each energy shell has one or more subshells, and each higher
  energy shell contains one additional subshell.} 

  \subsubsection{Orbital Oreintation}

  Each subshell contains one or more orbitals of the same energy (degeneracy)
  and have different 3-D orientations in space. Number of orientations increases
  by 2 in each subsequent subshell.

  \subsubsection{Electron Spin}

  I spent two semesters on this.

  \section{Electron Configuration}

  There are three rules:

  \begin{enumerate}
    \item \vocab{Aufbau Principle} - Electrons occupy the lowest energy orbitals
      available.  
    \item \vocab{Hund's Rule} - Electrons in the same subshell occupy available
      orbitals singly, before pairing up.  
    \item \vocab{Pauli Exclusion Principle} - There can be no more than two
      electrons in any given orbital.
  \end{enumerate}

  \subsection{Diamagnetic and Paramagnetic Atoms}

  \vocab{Diamagnetic} is an atom that has all of its electrons paired and are
  repelled by an external magnetic field. Otherwise the atom is
  \vocab{paramagnetic} in which case the atom is attracted to an external
  magnetic field.

  \subsection{Blocks in the Periodic Table}

  \vocab{Period} are the rows in the periodic table while \vocab{group} are the
  columns. Also know how to write electron configurations.

  \subsection{Anomalous Electron Configurations}

  Some atoms can achieve a lower energy state by having a filled or half-filled
  d subshell. E.g. chromium has configuration [Ar]$ 4s^{1}3d^{5} $. Also copper,
  Cu has configuration [Ar]$ 4s^{1}3d^{10} $. Others are Molybdenum (Mo) and
  silver and gold (Ag and Au).

  \subsection{Electron Configurations of Ions}

  \vocab{Isoelectronic} refers to when an ion has the same electron
  configuration as some atom e.g. $ F^{-} $ and Ne. Ionized atoms lose electrons
  from the highest energy orbital within the highest shell(n). It is trivial
  except for transition metals. Since these have 4s electrons they lose these
  before they lose the 3d electrons. 

  \section{Groups and their Characteristics}

  The valence shell determines the chemical reactivity of each group. Octets are
  highly stable and thus most chemical reactions are the quest for atoms to
  achieve this configuration. Thus halogens tend to form diatomic molecules
  sharing one electron covalently. Therefore, halogens behave as powerful
  oxidizing agents (strip an electron from another atom).

  \section{Periodic Trends}

  \subsection{Shielding}

  The valence electrons experience an effective nuclear charge due to the
  presence of the electrons in the shells between them and the nucleus.

  \subsection{Atomic and Ionic Radius}

  \begin{enumerate}
    \item Left to right period radius decreases, ionization energy increases,
      electronegativity increases, acidity increases, electron affinity more
      negative.  
    \item Down group radius increases, acidity increases,
      electronegativity decreases.
    \item Generally up group electron affinity more negative (except noble gases)
    \item For radius of ions $ X^{+} < X < X^{-} $
  \end{enumerate}

  \subsection{Electron Affinity}

  \vocab{Electron affinity} is the energy associated with the addition of an
  electron to an isolated atom. If energy is released(required) then electron
  affinity is negative(positive). Halogens have large negative electron
  affinity. Elements with filled shells or subshells tend to have positive
  electron affinities since adding them fills a new sub/shell leading to
  unstability e.g. noble gases and alkaline earth metals. 

  \subsection{Electronegativity}

  \vocab{Electronegativity} is a measure of an atom's ability to pull electrons
  to itself when it forms a covalent bond.

  \subsection{Acidity}

  \vocab{Binary acids} have the structure HX and dissociate in water as $ H^{+}
  + X^{-} $. The stronger the acid the more stable the dissociation. The more
  electronegative $ X^{-} $ is the more stable the dissociation and thus the
  more acidic the species. The larger the anion $ X^{-} $ is the more stable the
  dissociation, thus down the group acidity increases.


\end{document}
