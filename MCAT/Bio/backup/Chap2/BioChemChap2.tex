
\documentclass[../Bio_chemistryReview.tex]{subfiles}

\begin{document}
	
\chapter{The Cell\supdag}
	
\section{Cell Theory\supdag}
\begin{infobox}
  There are four tenets of cell theory:
  \begin{enumerate}
    \item All living things are composed of cells
    \item The cell is the basic functional unit of life
    \item Cells arise only from preexisting cells
    \item Cells carry genetic info in the form of DNA and is passed from parent
      to daughter cell 
  \end{enumerate}
\end{infobox}
Note that viruses are not considered living organisms.

\section{Eukaryotic Cells\supdag}

\vocab{Eukaryotic} cells \emph{do have} a true nucleus enclosed in a membrane,
whereas \vocab{prokaryotic} cells \emph{do not have} a nucleus. A cartoon of a
cell is given below:

\begin{center}
  \includegraphics[scale=0.1,frame]{Eukaryotic.jpg}
\end{center}

\subsection{Membrane-Bound Organelles\supdag}
		
Membranes consist of phospholipid bilayer. The \vocab[]{cytosol} allows for the
diffusion of molecules throughout the cell. DNA is organized into
\vocab[]{chromosomes} in the nucleus. Eukaryotic cells reproduce by
\vocab[]{mitosis} to produce identical daughter cells.

\subsubsection{The Nucleus\supdag}

This is the most tested organelle on the MCAT. It is encased in a double
membrane with \vocab[]{nuclear pores} to allow for selective two-way exchange of
material between cytoplasm and nucleus. Linear DNA is wound around organizing
proteins called \vocab[]{histones} and then further into \vocab[]{chromosomes}
(discussed further in chapter 6). The \vocab[]{nucleolus} is where
\vocab[]{ribosomal RNA (rRNA)} is synthesized.

\subsubsection{Mitochondria\supdag}

The mitochondria contains two layers. The \vocab[]{outer membrane} is a barrier
between the cytosol and inner environment of the mitochondria. The
\vocab[]{inner membrane} is impermeable and densely folded into
\vocab[]{cristae}. It contains the molecules and enzymes necessary for the
electron transport chain. The folding increases the surface area available for
electron transport chain. The \vocab[]{matrix} is the interior of the inner
membrane. Mitochondria replicate independently of the cell by \vocab{binary
fission}. Mitochondria are capable of kick-starting \vocab[]{apoptosis}. A
diagram of the mitochondria is given in \figref{fig:Mitochondrial Structure}.
\begin{figure}[H] 
  \centering
  \includegraphics[scale=0.1,frame]{Mitochondrial.jpg} 
  \caption{Mitochondrial Structure}
  \label{fig:Mitochondrial Structure}
\end{figure}

\subsubsection{Lysosomes\supdag}

\vocab{Lysosomes} contain hydrolytic enzymes which when released in a process
known as \vocab[]{autolysis} leads to apoptosis.

\subsubsection{Endoplasmic Reticulum\supdag}

The \vocab[]{endoplasmic reticulum} is actually contiguous with the nuclear
envelope. There are two varieties of ER: rough and smooth. The
\vocab[!endoplasmic reticulum]{rough ER} is studded with ribosomes allowing
translation of proteins destined for secretion directly into its lumen. The
\vocab[]{smooth ER} is utilized primarily for lipid synthesis (e.g.
phospholipids) and detoxification of certain drugs and poisons, and transport of
proteins from RER to the Golgi apparatus.

\subsubsection{Golgi Apparatus\supdag}

The \vocab[]{golgi apparatus} modifies cellular products allowing them to reach
the correct cellular location. Contents are released via exocytosis.

\subsubsection{Peroxisomes\supdag}

Contain hydrogen peroxide used for the breakdown of very long chain fatty acids
via $ \beta $-oxidation. They also participate in synthesis of phospholipids.

\subsection{The Cytoskeleton\supdag}

The \vocab{cytoskeleton} provides structure and shape to the cell and provides a
conduit for transport of materials around the cell. There are three components
of cytoskeleton: \emph{microfilaments, microtubules, and intermediate
filaments}.

\subsubsection{Microfilaments\supdag}

\vocab{Microfilaments} are made of \vocab{actin}. Actin filaments can also use
ATP for movement by interacting with \vocab{myosin}, such as in muscle
contraction. Play a role in \vocab{cytokinesis} (division of materials between
daughter cells) by forming the \vocab{cleavage furrow} leading to the pinching
off of the daughter cells.

\subsubsection{Microtubules\supdag}

\vocab{Microtubules} are hollow polymers of \vocab{tubulin} proteins. They
provide the pathways along which motor proteins like \vocab{kinesin} and
\vocab{dynein} carry vesicles. \vocab{Cilia} (movement of materials along
surface of cell) and \vocab{flagella} (movement of cell itself) are made of
microtubules. They have a \textbf{9 + 2 structure} meaning nine pairs of
microtubules forming an outer ring with two microtubules in the center.
\vocab{Centrioles} (triplet structure) are found in the \vocab{centrosome} and
are the organizing centers for microtubules in the cell. During mitosis become
antipodal, microtubules attach to chromosomes at \vocab{kinetochores} and pull
them apart into sister chromatids.

\subsubsection{Intermediate Filaments\supdag}

\vocab{Intermediate filaments} include keratin, desmin, vimentin, and lamins.
Involved in cell-cell adhesion, maintenance of overall integrity of the
cytoskeleton (making it more rigid), and anchor other organelles including the
nucleus.

\subsection{Tissue Formation\supdag}

There are four tissue types: epithelial, connective, muscle, and nervous. Only
epithelial and connective tissue are considered in detail here, others later on.

\subsubsection{Epithelial Tissue\supdag}

\vocab{Epithelial tissue} cover the body and line its cavities protecting
against pathogen invasion and dessication. Involved in absorption, secretion,
and sensation. Such cells are tightly joined to each other and underling layer
of connective tissue called the \vocab{basement membrane}. In most organs they
constitute the \vocab{parenchyma} (the functional parts of the organ) e.g.
nephrons in kidney or hepatocytes in the liver. Often are polarized, meaning
that one side faces the lumen and the other faces the "outside". 

\begin{infobox}

  \textsc{Classifications of epithelial cells:}\\ \textsc{By Layers}
\begin{enumerate} \item \vocab[!Epithelial tissue]{Simple epithelia} - one layer
    of cells \item \vocab[!Epithelial tissue]{Stratified epithelia} - multiple
    layers \item \vocab[!Epithelial tissue]{Pseudostratified epithelia} - have
      multiple layers due to differences in cell height, but are only one layer
\end{enumerate} \textsc{By Shape} \begin{enumerate} \item \vocab[!Epithelial
    tissue]{Cuboidal} - cube-shaped \item \vocab[!Epithelial tissue]{Columnar} -
      long and thin \item \vocab[!Epithelial tissue]{Squamous} - flat and
        scalelike \end{enumerate} \end{infobox}

\subsubsection{Connective Tissue\supdag}

\vocab{Connective tissue} supports the body and provide framework for epithelial
cells to carry out their function. Contribute to the \vocab{stroma} or support
structure. Produce and secrete materials such as collagen and elastin to form
the \vocab{extracellular matrix}. Include bone, cartilage, tendons, and others.

\section{Classification and Structure of Prokaryotic Cells\supdag}

\vocab{Prokaryotes} include all \vocab{bacteria} and do not contain any
membrane-bound organelles.

\subsection{Prokaryotic Domains\supdag}

There are two domains that fall under prokaryotes: Archaea and Bacteria.

\subsubsection{Archaea\supdag}

\vocab{Archaea} are single-celled and are genetically more similar to eukaryotes
than bacteria. They can live in many places and include \emph{extremophiles}.
Some are photosynthetic and many others are chemosynthetic.

\subsubsection{Bacteria\supdag}

All bacteria have flagella or fimbriae (similar to cilia). Bacteria often share
analog structures with eukaryotes. Some bacteria exist in mutualistic symbiosis
with humans while others are pathogenic. These pathogenic bacteria may live
intracellularly or extracellularly.

\subsection{Classification of Bacteria by Shape\supdag}

Three types of bacteria: \begin{enumerate} \item \vocab[!bacteria]{Cocci} -
    Spherical \item \vocab[!bacteria]{Bacilli} - Rod-shaped \item
      \vocab[!bacteria]{Spirilli} - Spiral-shaped \end{enumerate}

\subsection{Aerobes and Anaerobes\supdag}

Bacteria that require oxygen for metabolism are termed \vocab{obligate aerobes}.
And oppositely, \vocab{anaerobes}. Three different types of anaerobes:
\vocab[anaerobes]{obligate anaerobes} cannot survive in an oxygen-containing
environment, \vocab[anaerobes]{facultative anaerobes} can survive in either
condition, and \vocab[anaerobes]{aerotolerant anaerobes} cannot use oxygen but
are not harmed by it either.

\subsection{Prokaryotic Cell Structure\supdag}

\subsubsection{Cell Wall\supdag}

All prokaryotes have a \vocab{cell wall}(gram positive/negative) followed by the
cell membrane together forming the \vocab{envelope}. Gram positive cell walls
consist of a layer of \vocab{peptidoglycan}. Gram negative bacteria have less
peptidoglycan and also an \emph{outer membrane} outside of the cell wall. They
also have \vocab{lipopolysaccharides}.

\subsubsection{Flagella\supdag}

Composed of three parts: \vocab[!flagella]{filament}, \vocab[!flagella]{basal
body}, and \vocab[!flagella]{hook}. The filament is composed of
\emph{flagellin}. The basal body anchors the flagella to the cytoplasmic
membrane. The hook connects the filament and the basal body.

\subsubsection{Other Organelles\supdag}

\vocab{Plasmids} are small loops of DNA that are not necessary for the survival
of the prokaryote but may confer an advantage such as antibiotic resistance.
Since there are no mitochondria the cell membrane is used for ETC. They do have
a primitive cytoskeleton and ribosomes that have a different structure than
those of eukaryotes.

\section{Genetics and Growth of Prokaryotes\supdag}

Prokaryotes are capable of acquiring genetic material from outside the cell.

\subsection{Binary Fission\supdag}

Binary fission is a simple and quick process. The process is pictured in
\figref{binfis}

\begin{figure}[H] \centering \includegraphics[scale=0.5,frame]{binary.jpg}
  \caption{Binary Fission} \label{binfis} \end{figure}

\subsection{Genetic Recombination\supdag}

Many prokaryotes also contain extrachromosomal material known as
\vocab{plasmids}. These may contain \vocab{virulence factors} that increase its
pathogenicity. \vocab{Episomes} $ \in $ plasmids are capable of integrating into
genome. Three processes in total: \vocab[!plasmids]{transformation},
\vocab[!plasmids]{conjugation}, and \vocab[!plasmids]{transduction}.

\subsubsection{Transformation\supdag}

This is the integration of foreign genetic material into the host genome. Usu.
found from dead bacteria that have lysed in the environment.

\subsubsection{Conjugation\supdag}

This is the bacterial form of mating. It involves a \vocab{conjugation bridge}
made from \vocab[!conjugation]{sex pili}. Transfer unidirectionally from the
donor male(+) to the recipient female(-). To form the pilus need plasmid
\textbf{sex factors} e.g. \textbf{F (fertility) factor} in \emph{E. coli.} Male
can give sex factors to female that doesn't have it and turn it into one that
does. With it the entire genome replicates during conjugation and bacteria
attempt to transfer but usu. bridge breaks before complete transfer. If they are
successful then referred to as \textbf{Hfr (high frequency of recombination)}.

\subsubsection{Transduction\supdag}

Only form that requires a \vocab[transduction]{vector}|a virus that carries
genetic material from one bacterium to another e.g. \vocab{bacteriophage}.

\subsubsection{Transposons\supdag}

\vocab{Transposons} are genetic elements capable of inserting and removing
themselves from the genome. Not limited to just prokaryotes.

\subsection{Growth\supdag}

There are four phases in the growth of a bacterial colony. Since they reproduce
by binary fission they are all identical disregarding mutation.

\begin{infobox} \begin{enumerate} \item \vocab{Lag phase} - initial growth and
        adaptation to new environment \item \vocab{Exponential phase} aka.
        \textbf{log phase} - large growth and consumption of resources leading
      to (3) \item \vocab{Stationary phase} - reproduction reaches equilibrium
        with resources \item \vocab{Death phase} - resources depleted leading to
          death \end{enumerate} \end{infobox}

\begin{figure}[h] \centering \includegraphics[scale=0.5]{GrowthCurve.jpg}
  \caption{Bacterial Growth Curve} \end{figure}

\section{Viruses and Subviral Particles\supdag}

\subsection{Viral Structure\supdag}

Viruses are composed of genetic material(circular or linear, single- or
double-stranded DNA or RNA), a protein coat (\vocab{capsid}), and sometimes an
envelope containing lipids (if so bacteria is more sensitive to heat,
detergents, and dessication). The viral progeny are called \vocab{virions}.\par
Bacteriophages inject their genetic information into host through the
\vocab[!bacteriophage]{tail sheath}. The \vocab[!bacteriophage]{tail fibers}
help the bacteriophage recognize and connect to the correct host cell.

\subsection{Viral Genomes\supdag}

Single-stranded RNA viruses may be either \vocab{positive sense} or
\vocab{negative sense}. Positive sense viruses have RNA that may be directly
translated to functional proteins by the ribosomes of the host cell. Negative
sense viruses require their RNA have a \emph{complementary strand} synthesized
($ \implies $ virion carries \vocab{RNA replicase}) and then that can be used as
a template for protein synthesis.\par \vocab{Retroviruses} are enveloped,
single-stranded RNA viruses usu. containing two identical RNA molecules. they
carry \vocab{reverse transcriptase} to synthesize DNA from the single-stranded
RNA. This integrates into the host genome and remains indefinitely. HIV is an
example.

\subsection{Viral Life Cycle\supdag}

\subsubsection{Infection\supdag}

Bacteriophages inject as mentioned before and enveloped viruses fuse with the
plasma membrane.

\subsubsection{Translation and Progeny Assembly\supdag}

Depending on the type of genetic material carried, translocation to correct
locations in the cell must occur.  \begin{itemize} \item DNA goes to the nucleus
    to be transcribed into mRNA \item positive-sense RNA stays in the cytoplasm
    \item negative-sense RNA make complementary RNA first then has ribosomes
      translate \item DNA formed through reverse transcription in retroviruses
        also travel to the nucleus \end{itemize}

\subsubsection{Progeny Release\supdag}

Three methods of release:

\begin{enumerate} \item Trigger cell death \item Lysis \item \vocab[]{Extrusion}
      - virion fuses with plasma membrane and exits \end{enumerate}

\subsubsection{Lytic and Lysogenic Cycles\supdag}

Specifically for bacteriophages. During the \vocab[]{lytic cycle}, the
bacteriophage produces virions until the cell lyses. Bacteria in this phase are
termed \vocab[]{virulent}.\par During the \vocab[]{lysogenic cycle} it
integrates into the host genome as a \vocab[]{provirus} or \textbf{prophage}. It
remains dormant and reproduces as the host reproduces eventually re-entering the
lytic cycle.

\subsection{Prions and Viroids\supdag}

\vocab[]{Prions} are infectious proteins that are misfolded and trigger
misfolding of other proteins, usu. converting from an $ \alpha $-helical
structure to a $ \beta $-pleated sheet thereby drastically reducing the
solubility of the protein.\par \vocab[]{Viroids} are very short circular
single-stranded RNA that infects plants. They silence genes in the genome. There
are some that infect humans such as the hepatitis D virus.

\end{document}
