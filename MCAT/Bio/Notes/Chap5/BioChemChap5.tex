\documentclass[../Bio_chemistryReview.tex]{subfiles}

\begin{document}
\chapter{Nonenzymatic Protein Function and Protein Analysis\supdag}

\section{Cellular Function\supdag}

\subsection{Structural Protein\supdag}

The primary structural proteins in the body are 
\begin{description}
  \item \vocab{Collagen}  Makes up most of the extracellular matrix of
    connective tissue. Provides strength and flexibility.  
  \item \vocab{Elastin} (extracellular matrix) Primarily stretch and recoil
    like a spring to restore original shape of tissues 
  \item \vocab{Keratin} Intermediate filament proteins found in epithelial
    cells. Hair and nails 
  \item \vocab{Actin} makes up microfilaments and thin filaments in
    myofibrils. Has both positive and negative charges $ \implies $ motor
    proteins travel unidirectionally.  
  \item \vocab{Tubulin} Makes up microtubules whcih provide structure,
    chromosome separation in mitosis and meiosis, and intracellular transport.
    Has polarity: negative end adjacent to nucleus, positive end in periphery
\end{description}

Structural proteins usu. have repetitive organization of secondary structural
elements called a \vocab{motif}.

\subsection{Motor Proteins\supdag}

\emph{Display enzymatic activity}, acting as \vocab{ATPases} that power the
conformational change necessary for motor function. Transient interactions with
either actin or microtubules. \vocab{Myosin} - primary motor protein that
interacts with actin. It is the thick filament in a myofibril and is involved in
cellular transport. Has one head and one neck. \vocab{Kinesins} and
\vocab{dyneins} are the motor proteins associated with microtubules. Has two
heads, one remains attached to tubulin at all times. Both important for vesicle
transport: kinesins bring vesicles toward positive end of microtubule and
dyneins negative end. 

\subsection{Binding Proteins\supdag} 

Include hemoglobin, calcium-binding proteins, DNA-binding proteins, etc. Usually
transport or sequester molecules. Have different dissociation curves based on
role.

\subsection{Cell Adhesion Molecules\supdag}

\vocab{CAMs} integral membrane proteins found on surface and allow cells to bind to the extracellular matrix of other cells. Three major families:
\begin{description}
  \item \vocab{Cadherins} mediate calcium-dependent cell adhesion, usu. hold
    similar cell types together, each cell type has different type.
  \item \vocab{Integrins} Have two membrane spanning chains $ \alpha $ and $
    \beta $ important for binding and communicating with extracellular matrices
    (cell signaling).
  \item \vocab{Selectins} Unique in binding(weakest) to carbohydrates that
    project from other cell surfaces.  
\end{description}

\begin{figure}[h]
  \centering
  \includegraphics[scale=0.1]{CAMS.jpg}
\end{figure}

\subsection{Immunoglobulins\supdag}

\vocab{Antibodies} are produced by B-cells. Y-shaped with heavy and light chains
held together by disulfide linkages. Three possible outcomes from binding of
antigen:
\begin{enumerate}
  \item Neutralize the antigen
  \item \vocab{Opsonization} - Mark for destruction by white blood cells
  \item \vocab{Agglutinating} - Clumping into large antigen-antibody complexes
    for phagocytosis by macrophages.  
\end{enumerate}

\begin{figure}[h]
  \centering
  \includegraphics[scale=0.2]{Antibody.jpg}
\end{figure}

\section{Biosignaling\supdag}

Proteins participate in biosignaling by acting as extracellular ligands,
transporters for facilitated diffusion, receptor proteins, and second
messengers. Can have functions in substrate binding or enzymatic activity.

\subsection{Ion Channels\supdag}

\vocab{Ion channels} are proteins that create specific pathways for charged
molecules. \vocab{Facilitated diffusion}, a type of passive transport, the
diffusion of molecules (large, polar, or charged) down a concentration gradient
through a pore created by a transmembrane protein. Three types of ion channels.

\begin{description}
  \item \vocab{Ungated Channel} No gate and thus no regulation.
  \item \vocab{Voltage-Gated Channels} 	Regulated by the membrane potential
    change near the channel.  
  \item \vocab{Ligand-Gated Channels} Regulated by binding of specific substance
    or ligand.
\end{description}

Note that enzyme kinetics can be applied to these ion channels and in general
all transporters.

\subsection{Enzyme-linked Receptors\supdag}

\vocab{Enzyme-linked receptors} are membrane receptors with three primary
protein domains: membrane-spanning domain, ligand-binding domain, and catalytic
domain. Often results in \vocab{second messenger cascade}. \emph{Receptor
tyrosine kinases} are a classic example.

\subsection{G Protein-Coupled Receptors\supdag}

\vocab{G protein-coupled receptors} - a large family of integral membrane
proteins involved in signal transduction. 7 membrane-spanning $ \alpha
$-helices. GPCRs utilize a \vocab{heterotrimeric G protein}. Three main types
of G proteins:
\begin{description}
  \item  \vocab{G\textsubscript{s}}  stimulates adenylate cyclase $ \implies $
    increase in cAMP 
  \item \vocab{G\textsubscript{i}} inhibits adenylate cyclase
  \item \vocab{G\textsubscript{q}} activates phospholipase C, which cleaves a
    phospholipid from the membrane to form \chemfig{PIP_{2}} which is cleaved
    into DAG and \chemfig{IP_{3}} which can open calcium channels in the
    endoplasmic reticulum, increasing calcium in the cell.  
\end{description}

Inactive form, $ \alpha $-subunit bound to GDP, activates when $ \alpha
$-subunit GDP phosphorylated to GTP, subunit dissociates, affects adenylate
cyclase based on s or i version, loses \chemfig{P_{i}} and associates with G
protein complex to become inactive again.

\section{Protein Isolation\supdag}

\subsection{Electrophoresis\supdag}

Uses electric field to separate charged molecules based on net charge, z, and
size. \vocab{Migrational velocity}, v, is given by: 
\begin{equation}
  v = \dfrac{Ez}{f}
\end{equation}

Polyacrylamide gel is the standard medium for protein electrophoresis. It is
slightly porous thus allowing smaller molecules to travel faster through the gel
than larger ones.

\subsubsection{Native PAGE\supdag}

\vocab{Polyacrylamide gel electrophoresis (PAGE)} is a method for analyzing
proteins in their native states. Limited by the varying mass-to-charge and
mass-to-size ratios of cellular proteins. Most useful to compare molecular size
or charge of proteins known to be similar in size.

\subsubsection{SDS-PAGE\supdag}

\vocab{Sodium dodecyl sulfate (SDS)} separates on mass alone. Like PAGE but adds
SDS detergent to disrupt all noncovalent interactions and make the samples all
negatively charged to a magnitude proportional to their mass.

\subsubsection{Isoelectric Focusing\supdag}	

\vocab{Isoelectric focusing} separates on the basis of a protein's pI. Using a
gel with a pH gradient, the electric field moves the charged proteins until they
reach the pH at which they become neutral (pI). The anode must be on the acidic
end, remember by (\chemfig{H^{+}}) is positive, and the electrode bust be on the
basic end.

\subsection{Chromatography (see organic chemistry)\supdag}

See Organic Chemistry Review.

\section{Protein Analysis\supdag}

\subsection{Amino Acid Composition\supdag}

Amino acid composition determined by protein hydrolysis and subsequent
chromatographic analysis, not for sequencing. For sequence of small proteins (50
to 70) use \vocab{Edman degradation}. Sequentially removes N-terminal amino
acids. For large proteins use chymotrypsin, trypsin, and cyanogen bromide.

\subsection{Activity Analysis\supdag} 

Activity generally determined by monitoring a known reaction with given
concentration of substrate and comparing to a standard. Reactions with a color
change allow for rapid identification of samples.

\subsection{Concentration Determination\supdag}

Concentration determined almost exclusively through spectroscopy. Proteins usu.
have many aromatic side chains thus analyzed with \vocab{UV spectroscopy}.
Colorimetric change reactions include \vocab{bicinchoninic acid assay} (BCA),
\vocab{Lowry reagent assay}, \vocab{Bradford protein assay}.

\subsubsection{Bradford Protein Assay\supdag}

Uses Coomassie Brilliant Blue dye which originally is brown-green. Loses protons
to ionizable groups in the protein and turns blue in the process. Absorbance is
measured to create a \emph{standard curve}. Then unknown sample is exposed to
same conditions and concentration is determined based on the standard curve.
Accurate only when one type of protein is present in solution. Limited by
presence of detergent or excessive buffer.  \end{document}
