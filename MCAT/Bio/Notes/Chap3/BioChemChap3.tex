% !TeX root = BioChemChap3.tex
\documentclass[../Bio_chemistryReview.tex]{subfiles}

\begin{document}

\chapter{Biologically Important Molecules}
	
\section{Amino Acid Structure and Nomenclature}

Generic formula of amino acid consists of four groups attached to tetrahedral carbon:
\begin{center}
  \setatomsep{2.3em}
  \setcrambond{4pt}{}{}
  \chemfig{
    {\color{magenta}H_{2}\lewis{6:,N}}-[1](<[:60]{{\color{blue}R}})(<:[:120]H)-[7](=[6]{{\color{red}O}})(-[1]{{\color{red}OH}})
  }
\end{center}
All chiral amino acids in eukaryotes are L-amino acids (S configuration), so
amino group is drawn on the left in a Fischer projection, (except cysteine which
is still L but has R configuration since the -\chemfig{CH_{2}SH} group has
priority over the carboxyl group).\\ 
\vocab{Sidechain} distinguishes each amino
acid. Important properties of side chains include:
\begin{enumerate}
  \item Shape
  \item Charge
  \item Ability to H-bond
  \item Ability to act as acids or bases
\end{enumerate}

Four categories of amino acids with structures given on the next page: \\
\hfill \\
\noindent\parbox[t]{3in}{\raggedright%
  \textbf{\underline{Nonpolar,Hydrophobic}}
  \begin{enumerate} [topsep=2pt,itemsep=-2pt,leftmargin=13pt]
    \item \textcolor{red}{(Gly, G)} glycine \textcolor{blue}{(Hydrogen)}
    \item \textcolor{red}{(Ala, A)} alanine \textcolor{blue}{(alkyl)}
    \item \textcolor{red}{(Val, V)} valine \textcolor{blue}{(alkyl)}
    \item \textcolor{red}{(Leu, L)} leucine \textcolor{blue}{(alkyl)}
    \item \textcolor{red}{(Ile, I)} isoleucine \textcolor{blue}{(alkyl)}
    \item \textcolor{red}{(Phe, F)} phenylalanine \textcolor{blue}{(aromatic)}
    \item \textcolor{red}{(Trp, W)} tryptophan \textcolor{blue}{(aromatic)}
    \item \textcolor{red}{(Met, M)} methionine \textcolor{blue}{(sulfur)}
    \item \textcolor{red}{(Pro, P)} proline \textcolor{blue}{(cyclic)}
  \end{enumerate}
}
\hspace{5em}
\noindent\parbox[t]{3in}{\raggedright%
  \textbf{\underline{Polar, Neutral}}
  \begin{enumerate} [topsep=2pt,itemsep=-2pt,leftmargin=13pt]
    \item \textcolor{red}{(Ser, S)} serine \textcolor{blue}{-OH}
    \item \textcolor{red}{(Thr, T)} threonine \textcolor{blue}{-OH}
    \item \textcolor{red}{(Asn, N)} asparagine \textcolor{blue}{(amide)}
    \item \textcolor{red}{(Gln, Q)} glutamine \textcolor{blue}{(amide)}
    \item \textcolor{red}{(Cys, C)} cysteine \textcolor{blue}{(thiol)}
    \item \textcolor{red}{(Tyr, Y)} tyrosine \textcolor{blue}{(phenylalanine
      with an -OH)} \end{enumerate}
}
\vspace{10pt}
\newline
\noindent\parbox[t]{3in}{\raggedright%
  \textbf{\underline{Polar, Acidic}}
  \begin{enumerate} [topsep=2pt,itemsep=-2pt,leftmargin=13pt]
    \item \textcolor{red}{(Asp, D)} aspartic acid \textcolor{blue}{(carboxylic)}
    \item \textcolor{red}{(Glu, E)} glutamic acid \textcolor{blue}{(carboxylic)}
  \end{enumerate}
}
\hspace{5em}
\noindent\parbox[t]{3in}{\raggedright%
  \textbf{\underline{Polar, Basic}}
  \begin{enumerate} [topsep=2pt,itemsep=-2pt,leftmargin=13pt]
    \item \textcolor{red}{(His, H)} histidine \textcolor{blue}{(imidazole)}
    \item \textcolor{red}{(Arg, R)} arginine \textcolor{blue}{(3 N)}
    \item \textcolor{red}{(Lys, K)} lysine \textcolor{blue}{(terminal amine)}
  \end{enumerate}
}
\newpage
\subsubsection{Hydrophobic [Nonpolar] Amino Acids}
\begin{enumerate} [topsep=2pt,itemsep=-2pt,leftmargin=13pt]
  \item[-] Have aliphatic or aromatic side chains
  \item[-] Interior of folded globular proteins
\end{enumerate}

\subsubsection{Polar [Neutral] A.A}
\begin{enumerate} [topsep=2pt,itemsep=-2pt,leftmargin=13pt]
  \item[-] R group polar enough to form H-bonds w/ water but not polar enough to
    act as acid/base 
  \item[-] Serine, Threonine, Tyrosine often modified (attachment of phosphate by
    kinase) $\implies$ change in structure
    $\implies$ used in regulating protein activity
  \item[-] Amides of asparagine and glutamine do not gain or lose electrons.
    Remember this category is NEUTRAL
  \item[-] Thiol of cysteine makes it prone
    to oxidation
\end{enumerate}

\subsubsection{Polar,acidic}
\begin{enumerate} [topsep=2pt,itemsep=-2pt,leftmargin=13pt]
  \item[-] Have carboxylic groups instead of amides in reference to asparagine
    and glutamine 
  \item [-] Have negative charges on their side chains at
    physiological pH
\end{enumerate}

\subsubsection{Basic Amino Acids}
\begin{enumerate} [topsep=2pt,itemsep=-2pt,leftmargin=13pt]
  \item Lysine - 10
  \item Arginine - 12
  \item \textcolor{red}{Histidine} - 6.5 \hfill \\ At physiological pH (7.4)
    proton donor/acceptor $\implies$ prevalent at protein active sites
\end{enumerate}

\subsubsection{Sulfur-containing}
\begin{enumerate} [topsep=2pt,itemsep=-2pt,leftmargin=13pt]
  \item Cysteine - thiol group $\implies$ polar
  \item Methionine - thioether $\implies$ non-polar
\end{enumerate}

\subsubsection{Essential Amino Acids}
\begin{inparaenum}[1)]
\item Valine
\item Leucine
\item Isoleucine
\item Phenylalanine
\item Tryptophan
\item Methionine
\item Threonine
\item Lysine
\end{inparaenum}

\section{Acid-Base Chemistry of Amino Acids\supdag}

AA are \vocab{amphoteric} such that they can act as an acid or base depending on the pH of the environment. Remember that:

\begin{itemize}
  \item Ionizable groups tend to gain protons under acidic conditions and lose
    them under basic conditions.
  \item The p$ K_{a} $ is the pH at which half of the molecules of that species
    are deprotonated.
    \[ 
      \left\{ \begin{array}{ll}
        \text{pH} < \text{p}K_{a}, &\text{majority protonated} \\
        \text{pH} > \text{p}K_{a}, &\text{majority deprontonated}
      \end{array}\right\}
    \]
    This is understood through Henderson-Hasselbalch
    \begin{align}
      \boxed{\text{pH} = \text{p}K_{a} +
      \log\cbr[3]{\dfrac{[\chem{A^{-}}]}{[\chem{HA}]}}} \\
      \eqname{\textbf{Henderson-Hasselbalch}}
    \end{align}
\end{itemize}

\subsection{Protonation and Deprotonation\supdag}

All aa have at least two p$ K_{a} $ values; p$ K_{a1} $ carboxyl group (2) and
p$ K_{a2} $ amino group (9-10). For ionizable side chains will have a third p$
K_{a} $.

\subsubsection{Positively Charged Under Acidic Conditions\supdag}

At acidic pH below the p$ K_{a} $'s of the groups in the aa we expect them to be
protonated and thus positively charged.

\subsubsection{Zwitterions at Intermediate pH\supdag}

At physiological pH the carboxyl group is deprotonated but the amino group is
protonated for an overall neutral charge, a state that is called a
\vocab{zwitterion}.

\subsubsection{Negatively Charged Under Basic Conditions\supdag}

At higher pH like 10.5, both carboxyl and amino groups will be deprotonated and
thus the aa will carry a negative charge.

\subsection{Titration of Amino Acids\supdag}

The titration curve of 1 M glycine is shown below:

\begin{center}
  \includegraphics[scale=0.1]{Gly-Tit.jpg}
\end{center}

Note that at 0.5 equivalents, we have deprotonated exactly half of the carboxyl
groups and thus the pH = p$ K_{a} $ of the carboxyl group i.e. p$ K_{a}  =
2.34$. At 1 equivalent we reach the \vocab{isoelectric point} which is the pH
where all aa become zwitterions. For neutral amino acids it can be calculated by
averaging the two p$ K_{a} $ values for the amino and carboxyl group. 

\subsubsection{Amino Acids with Charged Side Chains\supdag}

For the acidic side chain aa (glutamic acid and aspartic acid), the
deprotonation occurs in the sequence, main carboxyl, side chain carboxy, main
amino. The isoelectric point is then: \[ \text{pI}_{\text{acidic amino acid}} =
\dfrac{\text{p}K_{a,\text{R group}} + \text{p}K_{a,\text{COOH group}}}{2} \]

For the basic aa such as lysine the deprotonation sequence is main carboxyl,
main amino, side chain amino. the isoelectric point is then: \[
\text{pI}_{\text{basic amino acid}} = \dfrac{\text{p}K_{a,\text{R group}} +
\text{p}K_{a,\text{amino group}}}{2} \]

\emph{aa with acidic side chains have relatively low pI while basic aa have
relatively high pI.}

\section{Protein Structure}

Two types \textcolor{red}{Peptide bonds} and \textcolor{red}{Disulfide bridges}

\subsection{Peptide bonds}
\begin{description}
  \item[-] Bond between carboxyl and $\alpha$-amino group (loss of water)
  \item[-] Individual A.A. in chain called \vocab{Residue}
  \item[-] Thermodynamically: chain is less favorable than individual residues
  \item[-] In cells: chain maintained b/c activation energy of hydrolysis is too
    high
  \item[-] Hydrolysis of protein by another protein -
    \vocab{proteolysis/proteolytic cleavage} cutting protein -
    \vocab{proteolytic enzyme/protease}
  \item[-] Many enzymes only cleave peptide bond adjacent to specific amino
    acid. \emph{Chymotrypsin} cleaves at the carboxyl end of the aromatic aa and
    \emph{trypsin} cleaves at the carboxyl end of arginine and lysine.
\end{description}

\subsection{Disulfide Bond}
\begin{description}
  \item[-] Sulfur of cysteine bonds to another. Residue is then called
    \textit{cystine} 
  \item[-] Important for stabilization of tertiary protein
    structure (\textbf{not neccessary} for correct folding) 
  \item[-] \vocab{Oxidation} - loss of electrons \vocab{Reduction} - gain of
    electrons	
  \item[-] Sulfur in cystine is more oxidized than cysteine.
  \item[-] Inside cells = reducing environment $\implies$ disulfide bridges
    most likely in extracellular proteins
\end{description}

\section{Protein Structure in Three Dimensions}

\begin{description}
  \item[-] \vocab{Denaturation} - disruption of a protein's shape without breaking peptide bonds.
  \item[-] Each level dependent on a type of bond
  \item[-] Proteins denatured by
    \begin{enumerate}
      \item urea (which disrupts hydrogen bonding)
      \item extremes of pH
      \item extremes of temperature
      \item changes in salt-concentration (tonicity) 
    \end{enumerate}
\end{description}

\subsection{Primary Structure: The Amino Acid Sequence}

\begin{description}
  \item[-] Sequence of A.A. dependent on \vocab{peptide bond}
\end{description}

\subsection{Secondary Structure: Hydrogen Bonds Between Backbone Groups}

\begin{description}
  \item[-] Initial folding of polypeptide chain stabilized by \textcolor{blue}{hydrogen bonds} between backbone NH and CO.
  \item[-] Certain motifs: \vocab{$\alpha$-helix} and \vocab{$\beta$-pleated sheet}
  \item[-] Properties of $\alpha$-helices (refer to pg. 47 Figure 5) \\
    \begin{inparaenum}[1)]
      \item 5 angstrom width
      \item 1.5 angstroms rise per A.A.
      \item 3.6 A.A residues per turn
      \item $\alpha$-carboxyl oxygen H-bonded to $\alpha$-amino proton three
        residues away 
    \end{inparaenum}
  \item[-] proline forces kink in chain $\implies$ proline residue never in
    $\alpha$-helix.  
  \item[-] $\alpha$-helix often found in transmembrane regions of cell membrane
    \begin{enumerate}
      \item all polar NH and CO are H-bonded so don't interact w/ hydrophobic
        membrane interior 
      \item have hydrophobic R groups radiate out from helix,
        interact w/ hydrophobic interior of membrane
    \end{enumerate}

  \item[-] $\beta$-pleated sheets stabilized by H-bond CO and NH
  \item[-] Two types of $\beta$-sheets:
    \begin{enumerate}
      \item \textcolor{red}{\textbf{parallel}} - adjacent polypeptide strands in
        \textit{same} direction 
      \item \textcolor{red}{\textbf{antiparallel}} -
        \textit{opposite} direction
    \end{enumerate}	
\end{description}

\subsection{Tertiary Structure: Hydrophobic/Hydrophilic Interactions}

\begin{description}
  \item[-] Folding via \textcolor{blue}{interactions between A.A. residues}
    located distantly from each other in chain 
  \item[-] Hydrophobic R
    $\rightarrow$ interior of protein \hfill \\ Hydrophilic R $\rightarrow$
    exterior of protein 
  \item[-] includes disulfide bonds within a single chain
\end{description}

\subsection{Quaternary Structure: Various Bonds Between Separate Chains}

\begin{description}
  \item[-] \textcolor{blue}{Interactions between polypeptide subunits} important
    for protein function 
  \item[-] \vocab{Subunit} - single polypeptide chain that
    is part of a large complex containing many subunits
  \item[-] Arrangement of subunits in multisubunit complex 
  \item[-] Includes
    disulfide bonds between different chains
\end{description}

\section{Carbohydrates}

Carbohydrates $\rightarrow \; CO_{2}$ by \vocab{oxidation} 

\subsection{Monosaccharides and Disaccharides\supddag}

\vocab[]{Monosaccharides} have general formula \chem{C_{n}(H_{2}O)_{n}} for $
3<n<\text{6 or 7} $ while for complex sugars the general formula is
\chem{C_{n}(H_{2}O)_{m}} where m<n since the joining of a sugar is a dehydration
reaction. For the joining of multiple monosaccharides we have the following
terminology: 
\begin{description}
  \item[\vocab{disaccharide}] 2 monos.
  \item[\vocab{oligosaccharide}] several
  \item[\vocab{polysaccharide}] many
\end{description}
A \vocab{glycosidic linkage} is a bond between sugars. Two types of linkages
depending on if anomeric carbon is $ \alpha \text{ or } \beta $:
\begin{enumerate}
  \item $\alpha$ - OH is (axial or equatorial) down
  \item $\beta$ - OH is (axial or equatorial) up
  \item discussion at
    \url{http://www.chem.ucla.edu/harding/ec_tutorials/tutorial08.pdf}
\end{enumerate}
Common disaccharides on MCAT
\begin{enumerate}
  \item sucrose - glucose + fructose
  \item maltose - glucose + glucose
  \item cellobiose - two $ \beta $-glucose molecules linked by a $ \beta $-1,4
    linkage 
  \item lactose - glucose + galactose 
\end{enumerate}
If the carbohydrate contains an aldehyde group then it is classified as a
\vocab[]{aldose} and those with a ketone group is classified as a
\vocab[]{ketose}. Thus, a six-carbon sugar with an aldehyde group is an
\emph{aldohexose}. The simplest aldose is
\emph{glyceraldehyde}\index{glyceraldehyde!aldose}: 
\begin{center}
  \definesubmol\nobond{-[4,2,,,draw=none]}
  \setatomsep{2.5em}
  \chemname{\chemfig{C(!\nobond\textcolor{red}{2})(-OH)(-[4]H)(-[6]CH_{2}OH(!\nobond\textcolor{red}{3}))-[2]C(!\nobond\textcolor{red}{1})(-[1]H)=[3]O}}{\textcolor{red}{Glyceraldehyde}}
\end{center}
The lowest number carbon in a sugar is always going to be the carbonyl since it
is the most oxidized. Glycosidic linkages occur at the (C-1) carbon.  
The simplest ketose is \emph{dihydroxyacetone}\index{dihydroxyacetone!ketose}:
\begin{center}
  \centering
  \setatomsep{2em}
  \chemname{\chemfig{(-[5]-[3]OH)(-[7]-[1]OH)=[2]O}}{\textcolor{red}{Dihydroxyacetone}}
\end{center}
Note that the carbonyl cannot be made carbon one so now we number as to make it
the lowest number possible. Glycosidic linkages occur at the (C-2) carbon here.
Note that \emph{for every monosaccharide, every carbon other than the carbonyl
has a hydroxyl group.} Below are four sugars that you should be familiar with
and their common names: 
\begin{figure}[h]
  \centering
  \includegraphics[scale=0.2]{CommonSugars.jpg}
\end{figure}
The \vocab{Fisher Projection} shown in \figref{Fisher} above has the following
stereochemical interpretation with horizontal bonds as wedges and the vertical
as dash and wedge:
                %	\begin{figure}[h]
                %		\centering
                %		\includegraphics[scale=0.1]{FisherProj.jpg}
                %	\end{figure}
\begin{figure}[h]
  \centering
  \caption{\textbf{Fisher Projection}}
  \vspace{1em}
  \chemfig{(<:[2]Cl)(<:[6]CH_{3})(<Br)<[4]H} \qquad $ \equiv $ \qquad
  \chemfig{(-[2]Cl)(-[6]Cl)(-[4]H)-Br}\\ [0.5cm]
  \label{Fisher}
\end{figure}
Now for sugars in the Fisher Projection: \emph{D-sugars} have the OH of their
highest chiral carbon (the carbon above the \chem{CH_{2}OH}) on the right and
\emph{L-sugars} have theirs on the left. Since D and L sugars are enantiomers
then every chiral center has opposite configuration. For Fisher this means that
the OH's will be on the opposite sides. The following four molecules have
multiple interesting features:

\begin{figure}[h]
  \centering
  \includegraphics[scale=0.2]{StereoExam.jpg}
\end{figure}

The D and L forms are enantiomers of each other. D-erthyrose and D-threose are
epimers and thus diastereomers (see OChemReview 2.2.10). Remember that a
compound can only have one enantiomer (makes sense since the enantiomer switches
all R's to S's and vice versa).

\subsection{Cyclic Sugar Molecules\supdag}

The hydroxyl group and the carbonyl group of a monosaccharide can form cyclic
\vocab{hemiacetals} and \vocab{hemiketals}. Due to ring strain, only
six-membered \vocab{pyranose} rings and five-membered \vocab{furanose} rings are
allowed. The C-1 carbon becomes an anomeric carbon.

\subsubsection{Hexose Conformations\supdag}

For cyclic structures, aside from the Fisher projection there is the
\vocab{Haworth projection}. Anything on the right side of the Fisher projection
goes on the bottom face of the Haworth projection.

\subsubsection{Mutarotation\supdag}

Exposing hemiacetal rings to water causes spontaneous cycling between open and
closed forms. Since the C-1 can rotate freely in open form, when the ring forms
it may switch its anomeric form. This process is called \vocab{mutarotation} and
is catalyzed with acid or base.

\section{Chemistry of Monosaccharides\supdag}

Since monosaccharides contain alcohols and aldehydes and ketones, they can
undergo oxidation and reduction, esterification, and nucleophilic attack.

\subsection{Oxidation and Reduction\supdag}



\subsection{Hydrolysis of Glycosidic Linkages}

\begin{itemize}
  \item Hydrolysis of polysaccharides thermodynamically favored
  \item Enzymes catalyze hydrolysis of sugars (e.g. maltase for maltose)
  \item Mammals can't break $\beta$ linkages expect lactose w/ lactase.
\end{itemize}

\section{Lipids}

\begin{itemize}
  \item Three physiological roles
    \begin{description}	
      \item[adipose cells] triglycerides store energy
      \item [cellular membranes] made of phospholipids
      \item[Cholesterol] building block for hydrophobic steroid hormones
    \end{description}
\end{itemize}

\subsection{Fatty Acid Structure}

\begin{itemize}
  \item Fatty acids - long unsubstituted alkanes (14-18 C's), end in carboxylic
    acid. 
  \item Two types
    \begin{description}
      \item[Saturated] no double bonds (saturated with H)
      \item[Unsaturated] double bonds (almost always \emph{cis})
    \end{description}
  \item Forms micelles
\end{itemize}

\subsection{Triacylglycerols [TG]}

\begin{itemize}
  \item Formed by three fatty acids esterifed to a glycerol molecule (condensation rxn).
  \item \textbf{Lipases} enzymes that hydrolyze fats.
  \item TG better for energy storage than carbs b/c
    \begin{description}
      \item[Packing]: hydrophobicity $\implies$ tighter packing. Also, higher
        carbon density than carbohydrates which have lots of water molecules.
      \item[Energy Content]: more energy than carbs $\because$ much more reduced
      and energy metabolism requires oxidation to release energy
    \end{description}
\end{itemize}

\subsection{Lipid Bilayer Membranes}

\begin{itemize}
  \item Membrane lipids are phospholipids 
  \item Phospholipids are
    \vocab{detergents} - substances that efficiently solubilize oils while
    remaining highly water-soluble
  \item Membrane fluidity is affected by
    \begin{enumerate}
      \item Unsaturation decreases packing (higher fluidity)
      \item Shorter fatty acid tail (higher fluidity)
      \item Cholesterol optimizes fluidity
    \end{enumerate}
\end{itemize}

\subsection{Terpenes}

Terpenes built from isoprene units below

\begin{center}
  \setatomsep{2.3em}
  \setcrambond{4pt}{}{}
  \chemfig{
                                        %	-[:30](-[:0]=[:-30])-[:30]
    -[1](-[2])=[7]-[1]
  }
\end{center}
\begin{itemize}
  \item Terpenes may be cyclic or linear 
\end{itemize}

\subsection{Steroids}

\begin{itemize}
  \item All have tetracyclic ring system based on structure of cholesterol
  \item Cholesterol carried in blood via \vocab{lipoproteins}
  \item Steroid hormones highly hydrophobic so can dissolve through membrane
    $\implies$ no receptors on membrane all are inside. \emph{Peptide} hormones
    rely on receptors on membrane.  
\end{itemize}

\section{Phosphorus-Containing Compounds}

\begin{itemize}
  \item Phosphoric acid largely anionic form at physiological pH. $K_{a,2}=7.2$
  \item 2 phosphates bond via \vocab{anhydride linkate} to make
    \vocab{pyrophosphate} (contains lot of energy) 
  \item Reasons for high energy storage:
    \begin{enumerate}
      \item negative charges of linked phosphates repel
      \item phosphate more resonance forms than linked phosphates $\implies$
        lower free energy 
      \item phosphate more favorable interaction with biological solvent than
        linked phosphates
    \end{enumerate}
\end{itemize}

\subsection{Nucleotides}

\begin{itemize}
  \item Nucleotide composed of:
    \begin{enumerate}
      \item ribose sugar
      \item purine or pyrimidine base on C1 of ribose ring
      \item 1-3 phosphate units on C5 of ribose ring 
    \end{enumerate}
\end{itemize}
\end{document}
