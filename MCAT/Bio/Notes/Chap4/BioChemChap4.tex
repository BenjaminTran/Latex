\documentclass[../Bio_chemistryReview.tex]{subfiles}

\begin{document}
\chapter{Biochemistry}

\section{Thermodynamics}

Two forms of energy in chem: heat and potential (bond) energy. Laws of
thermodynamics:\\ 
\vocab{First law}: conservation of energy\\
\vocab{Second law}: \vocab{Entropy}, S, always increases\\
Gibbs free energy:
\begin{equation}
  \boxed{\Delta G = \Delta H - T\Delta S}
\end{equation}
Where H is enthalpy defined by:
\begin{equation}
  \boxed{\Delta H = \Delta E - P\Delta V}
\end{equation}
$ \Delta H <(>) 0 $ is called \vocab{exothermic(endothermic)}. Former releases
energy, latter requires input of energy.\\
$ \Delta G <(>) 0 $ is called \vocab{exergonic(endergonic)}. Former energy exits
system. In physiology, reactions driven by coupling to exergonic reactions.\\
Note: Signs are assigned from point of view of system.  From (4.1) and the
second law if $ \Delta G <(>) 0$ a reaction is spontaneous(nonspontaneous).\\
$ \Delta G $ depends on concentrations of reactants and products. To compare
reactions a standard free energy at pH 7 is determined with relation to the
equilibrium constant of the reaction: 
\begin{equation}
  \boxed{\Delta G^{o\prime} = -RT\ln K_{eq}^{\prime}}
\end{equation}
Where $ K_{eq}^{\prime} $ is given by:
\begin{equation}
  K_{eq}^{\prime} = \dfrac{[C]_{eq}[D]_{eq}}{[A]_{eq}[B]_{eq}}
\end{equation}
Note: when $ K_{eq}^{\prime} = 1, \Delta G^{o'} = 0$\\
For nonstandard conditions\\
\begin{equation}
  \boxed{\Delta G = \Delta G^{o\prime} + RT \ln Q}
\end{equation}

Where Q has same form as $ K_{eq}^{'} $ but at arbitrary concentration. Can also
be rewritten as 
\begin{equation}
  \boxed{\Delta G = \Delta G^{o\prime} + RT\ln \dfrac{Q}{K_{eq}^{\prime}}}
\end{equation}

Recall \vocab{Le Chatlier's Principle} can be used to shift reactions in
different directions.\\ Thermodynamics says nothing about reaction rates.

\section{Kinetics and Activation Energy [$ E_{A} $]}

All reactions proceed through an intermediate that is unstable and requires
energy to produce. The amount of energy is called the \vocab{activation energy}
$ E_{a} $. A \vocab{catalyst} lowers $ E_{a} $ without changing $ \Delta G $
(\emph{thus it cannot make a nonspontaneous reaction spontaneous}) by
stabilizing the transition state. Enzymes are catalysts: some common ones are
given below:

\begin{tabular}{|c|p{15cm}|}\hline
  Enzyme Class & Reaction \\ \hline 
  Hydrolase & Hydrolyzes chemical bonds usu. named only for their substrate (includes ATPases, proteases, and others) \\ \hline
  Isomerase & rearranges bonds within a molecule to form an isomer. Catalyzes reactions between stereoisomers and constitutional isomers. Some can also be classified as oxidoreducatases, transferases, or lyases. \\ \hline
  Ligase & forms a chemical bond (addition or synthesis) usu. between large similar molecules and often requires ATP (e.g., DNA ligases) \\ \hline
  Lyase & Cleaves one molecule into two by means other than oxidation or hydrolysis (e.g. pyruvate decarboxylase) Reverse direction builds up and is then called a \emph{synthase.} \\ \hline
  Kinase & transfers a phosphate group to a molecule from a high energy carrier, such as ATP \\ & (e.g., phosphofructokinase PFK) \\ \hline
  Oxidoreductase & runs redox reactions, transfer of electrons between biological molecules (includes oxidases, reductases, \emph{dehydrogenases}, and others) \\ \hline
  Polymerase & polymerization (e.g. addition of nucleotides to the leading strand of DNA by DNA polmerase III) \\ \hline
  Phosphatase & remoces a phosphate group from a molecule \\ \hline
  Phosphorylase & transfers a phosphate group to a molecule from inorganic phosphate (e.g. glycogen phosphorylase) \\ \hline
  Protease & hydrolyzes peptide bonds (e.g. trypsin, chymotrypsin, pepsin, etc.)\\ \hline
  Transferase\supdag & Catalyze movement of a functional group from one molecule to another. E.g. \emph{aminotransferase} \\ \hline
\end{tabular} 

\subsection{ATP as an Energy Source: Reaction Coupling}

ATP hydrolysis is the usual reaction coupled to nonspontaneous reactions in the
body with $ \Delta G $ = -12 kcal/mol in cellular conditions. Several ways ATP
hydrolysis may drive unfavorable reactions is by \emph{conformational changes in
protein} or \emph{phosphorylation of a substrate}. Recall that free energy
changes are additive thus coupling multiple negative $ \Delta G $ reactions with
positive ones can make it overall negative. 

\section{Enzyme Structure and Function}

Protein folding is important for enzymes for proper formation of the
\vocab{active site} implying most enzymes fold into a globular shape as opposed
to fibrous. \vocab{Substrates} are the reactants. Two models exist and are
accepted but ultimately, the purpose is to stabilize the transition state
intermediate.  Active sites are stereospecific (\emph{note: in animals the more
common configurations are L aa and D sugars.})\\
\vocab{Proteases} have an active site with a serine residue whose OH group acts
as a nucleophile and attacks the carbonyl carbon of an aa residue in a
polypeptide. Will also have a \vocab{recognition pocket} to attract certain
residues on substrate.\\
Enzymes are sensitive to temperature and pH. Extremes can lead to
\vocab{denaturing} which renders it useless.\\
Enzyme function may also require or depend on \vocab{cofactors} - small metal
ions or molecules. If organic it is called a \vocab{coenzyme}.

\section{[K2.2] Enzyme-Substrate Binding\supdag}

Of the two theories, \emph{Lock and Key Theory} and \emph{Induced Fit}, induced
fit is the more acceptable and MCAT tested theory.

\subsection{Lock and Key Theory\supdag}

This theory states that the enzyme has an active site with the perfect shape to
fit the substrate.

\subsection{Induced Fit Model\supdag}

The enzyme and substrate do not necessarily look like they fit together
initially. Once the substrate and enzyme begin to bind they undergo
conformational changes to fit each other.

\subsection{Cofactors and Coenzymes\supdag}

Cofactors and coenzymes tend to be small in size so that they can bind to the
active site and participate in the catalysis of the reaction. Enzymes without
their cofactors are called \vocab{apoenzymes} and with their enzymes are called
\vocab{holoenzymes}. Recall that vast majority of enzymes are vitamins. The
\emph{water-soluble} vitamins are B complex vitamins and absorbic acid (vitamin
C). Must be regularly replenished since easily excreted. The \emph{fat-soluble}
vitamins are A, D, E, and K. Some B vitamins are listed below and DO NOT NEED TO
BE MEMORIZED, but are mentioned in passages:

\begin{enumerate}
  \item \chemfig{B_{1}} : thiamine
  \item \chemfig{B_{2}} : riboflavin
  \item \chemfig{B_{3}} : niacin
  \item \chemfig{B_{5}} : pantothenic acid
  \item \chemfig{B_{6}} : pyridoxal phosphate
  \item \chemfig{B_{7}} : biotin 
  \item \chemfig{B_{9}} : folic acid
  \item \chemfig{B_{12}} : cyanocobalamin
\end{enumerate}

\section{Regulation of Enzyme Activity}

Activity of enzymes in metabolic pathways is usually regulated in the following
ways.

\subsection{Covalent Modification\supddag}

Different groups attached to proteins can \emph{regulate activity, lifespan,
and/or location.} E.g. phosphorylation of the hydroxyl of serine, threonine, or
tyrosine residues via kinases can regulate activity. \vocab{Phosphorylases} do
same thing but with $ P_{i} $ not from ATP. \vocab{Phosphatases}
unphosphorylate. \vocab{Glycosylation} is the addition of sugar moities to
molecules and can be used to tag an enzyme for transport.

\subsection{Proteolytic Cleavage\supddag}

Enzymes synthesized in inactive forms, called \emph{zymogens}, are activated by
cleavage. Zymogens have catalytic (active) domain and regulatory domain which is
the part that must be cleaved. 

\subsection{Association Polypeptides}

When catalytic activity is regulated by association with a separate subunit.
E.g. protein may exhibit continuous rapid catalysis if regulatory subunit is
removed. This protein thus exhibits \vocab{constitutive activity}.

\subsection{Allosteric Regulation\supddag}

Modification of active site through interactions of molecules with other
specific sites on the enzyme not including active site. Needed to turn on or off
an enzyme and can alter the activity for better or worse. Generally non-covalent
and reversible. Leads to sigmoidal shape on Michaelis-Menten plots.

\subsection{Feedback Inhibition\supddag}

\vocab{Negative feedback (feedback inhibition)} when a product shuts off an
enzyme used to produce it. \vocab{Feedforward stimulation} is when an enzyme is
stimulated by its substrate or by a molecule produced earlier in the synthesis;
this may include regulation as well. 

\section{Basic Enzyme Kinetics\supddag}

\vocab{Enzyme Kinetics} study of the rate of formation of products from
substrates in the presence of an enzyme. \vocab{Reaction rate} is the amount of
product formed per unit time, in moles per second. Enzymes have limited active
sites thus they can be saturated by high concentrations of substrate. Rxn rate
at this point is $ V_{max} $. The \vocab{Michaelis constant} denoted $ K_{m} $
is the substrate concentration at which enzyme is $ \tfrac{V_{max}}{2} $.
\underline{Substrate affinity inversely proportional to $ K_{m} $.} These ideas
are demonstrated in the picture below and in the following equation:

\begin{align}
  \boxed{v = \dfrac{v_{max}[S]}{K_{m} + [S]}} \label{Michaelis-Menten} \\
  \eqname{Michaelis-Menten Equation} 
\end{align}

\begin{center}
  \includegraphics[scale=0.1]{Mic-men.jpg}
\end{center}

\subsection{Lineweaver-Burk Plots\supdag}

Since Eqn. \eqref{Michaelis-Menten} is hyperbolic it is easier to rewrite the
equation as: 
\[ \dfrac{1}{v} = \dfrac{K_{m}}{V_{max}}\dfrac{1}{[S]} +
\dfrac{1}{v_{max}} \]
This is linear in the variables 1/v  and  1/[S] and allows for easy
determination of $ v_{max} \text{ and } K_{m} $. A plot is given below.

\begin{figure}[h]
  \centering
  \includegraphics[scale=0.1]{Line.jpg}
\end{figure}

\subsection{Cooperativity\supddag}

When binding of a substrate increases enzyme affinity for more substrate. Enzyme
needs to have more than one active site, and generally are composed of more than
one protein chain (multisubunit complexes). \emph{Leads to sigmoidal curve.} Can
also apply to non-enzymes as well such as hemoglobin.\par May exist in a
low-affinity tense state (T) or a high-affinity relaxed state (R).

\subsection{Effects of Local Conditions on Enzyme Activity\supdag} 

Note that the terms \vocab{enzyme activity, enzyme velocity,} and \vocab{enzyme
rate}.

\subsubsection{Temperature\supdag}

As a rule of thumb enzyme velocity tends to double for every increase in 10$
^{\circ} $C increase until optimal activity is reached. After optimal
temperatures, activity drops quickly.

\subsubsection{pH\supdag}

pH can also denature enzymes. Optimal for humans is 7.4 (except the digestive
tract) and below 7.35 is deemed acidemia. 

\subsubsection{Salinity\supdag}

Large concentrations of salts can affect hydrogen bonds and ionic bonds.

\subsection{Inhibition of Enzyme Activity}

Four types: \vocab{Competitive inhibition},  \vocab{noncompetitive
inhibition},  \vocab{uncompetitive inhibitor}, and \vocab{mixed-type
inhibition}. In this study, determine the effects of each on the Lineweaver-Burk
Plot.

\subsubsection{Competitive Inhibition\supddag}

Molecules that compete with substrate for binding at active site. Thus, have
similar structure to substrate; best if similar structure to the transition
state which normally stabilizes active site. Can overcome competitors by
increasing substrate. Thus $ V_{max} $ is not affected but requires higher
concentration. However, $ K_{m} $ is increased.

\begin{center}
  \includegraphics[scale=0.1,frame]{CompInhib.jpg}
\end{center}

\subsubsection{Noncompetitive Inhibition}

Bind at an allosteric site but not active site, thus cannot overcome by adding
more substrate. Thus $ V_{max} \text{ and } \tfrac{V_{max}}{2}$ decreases,
however, $ K_{m} $ is not affected because the substrate may still bind to the
enzyme but no activity will occur.

\begin{figure}[h]
  \centering
  \includegraphics[scale=0.1,frame]{NoncompInhib.jpg}
\end{figure}

\subsubsection{Uncompetitive Inhibition}

Can only bind to the \emph{enzyme-substrate} complex. Decreases $ V_{max} $ and
increases apparent affinity (decrease $ K_{m} $) because the complex cannot
dissociate. Must bind to an allosteric site which is formed when enzyme and
substrate bind. Appear as parallel lines on Lineweaver-Burk Plot. 

\subsubsection{Mixed-type Inhibition\supddag}

Can bind to both enzyme and enzyme-substrate complexes but have different
affinity for each. Bind to allosteric sites. Three cases: 
\begin{infobox}
  \underline{Case I}: \emph{enzyme greater affinity} for inhibitor in free
  form\\
  Lower affinity for substrate $ \boxed{ \sim  competitive}$ inhibition
  ($ K_{m} $) increases.\\
  \underline{Case II}: \emph{enzyme-substrate greater affinity} for inhibitor\\
  Greater affinity for substrate  $\boxed{\sim  uncompetitive}$ inhibition ($
  K_{m} $) decreases\\
  \underline{Case III}: (rare) \emph{equal affinity} for both\\
  $ \boxed{Noncompetitive} $ inhibitor.
\end{infobox}

\subsection{Summary of Inhibition}

\begin{center}
  \includegraphics[scale=0.15,frame]{InhibitionSum.jpg}
\end{center}

\section{Cellular Respiration}

\subsection{Energy Metabolism and the Definitions of Oxidation and Reduction}

Plants and animals store chemical energy in carbohydrates and fats that are
oxidized to produce \chem{CO_{2}} and ATP. All reactions boil down to
\vocab{oxidation} (uptake of oxygen) and \vocab{reduction} (removal of
oxygen).\\
Three meanings of oxidize:
\begin{enumerate}
  \item attach oxygen (or increase the number of bonds to oxygen)
  \item remove hydrogen
  \item remove electrons
\end{enumerate}
To reduce is to do the opposite of the above.\\

Note: E.g.  \chem{O_{2} \rightarrow H_{2}O}  is a reduction because bonds to oxygen
are decreased.\\
When something is oxidized something is also reduced hence \vocab{redox pair}.\\
\vocab{Catabolism(anabolism)} is the process of breaking down(building)
molecules. So energy is obtained from glucose by \vocab{oxidative catabolism} in
four steps.  
\begin{enumerate}
  \item glycolysis
  \item pyruvate dehydrogenase complex (PDC)
  \item Krebs cycle
  \item electron transport/oxidative phosphorylation
\end{enumerate} 

\emph{We make the unfavorable synthesis of ATP by coupling it to the favorable
oxidation of glucose.}

\subsection{Introduction to Cellular Respiration}

Oxidation of glucose does not generate a lot of ATP. Reduction of high-energy
electron carriers, (nicotinamide adenine dinucleotide) $ \boldsymbol{NAD^{+} }$
and (flavin adenine dinucleotide) \textbf{FAD} is the trick. Reduced to
\textbf{NADH} and $\boldsymbol{FADH_{2}}$ and later oxidized when delivering $
e^{-} $ to $ e^{-} $ transport chain generating proton gradient to generate ATP.
These two also have other roles as cofactors e.g. FAD can associate with a
protein to become a \vocab{flavoprotein}.\\ Some highlights of the four-step
process:
\begin{infobox}
  \begin{enumerate}
    \item \vocab{Glycolysis} splits glucose into two three carbon pyruvic acid
      molecules producing small amount of ATP and NADH. Occurs in cytoplasm and
      requires no oxygen.  
    \item \vocab{Pyruvate Dehydrogenase Complex}
      decarboxylates pyruvate to form acetyl group and attached to Coenzyme A
      transferring it to the Krebs cycle. Small NADH is produced.  
    \item \vocab{Krebs(TCA) Cycle} - acetyl group attached to oxaloacetate $
      \rightarrow $ citric acid $ \rightarrow $ decarboxylated and isomerized to
      regenerate oxaloacetate. Large(small) amounts of ATP/NADH($ FADH_{2} $)
      are produced.  
    \item \vocab{Electron transport} - Only stage that requires
      oxygen. $ NADH \text{ and } FADH_{2} $ dump electrons at beginning which
      cascade down to reduce $ O_{2} \text{ to } H_{2}O $. Energy generated
      along the cascade used to pump protons out of matrix $ \rightarrow $
      proton gradient. Protons flow back in via ATP synthase to generate ATP. 
  \end{enumerate}
\end{infobox}

(2) and (3) occur in the matrix of mitochondrium. (4) on the membrane separating
matrix from outer compartment.

\subsection{Glycolysis}	

Summarizing equation:
\[ \text{Glucose } + 2 \text{ADP }+ 2 P_{i} + 2\text{NAD}^{+} \rightarrow 2 \text{ pyruvate} + 2\text{ATP } + 2 \text{NADH} + 2 H_{2}O + 2H^{+} \]

Just memorize what goes in and what comes out. Some highlights:
\begin{enumerate}
  \item \vocab{Hexokinase} catalyzes first step of glycolysis; glucose $
    \rightarrow $ G6P.  
  \item \vocab{Phosphofructokinase (PFK)} catalyzes transfer of phosphate from ATP
    to fructose-6-phosphate to form fructose-1,6-bisphosphate (F1,6bP). $ \Delta G
    << 0$ for this reaction so it is not reversible. This is the step that is
    subject to allosteric regulation by ATP and thus the regulation point of
    glycolysis.
\end{enumerate}

\subsection{Fermentation}

Anaerobic process evolved to regenerate NAD$^{+}$ to continue glycolysis. Note:
without oxygen electron transport does not work. Fermentation uses pyruvate as
the acceptor of high energy electrons from NADH. Two options 
\begin{enumerate}
  \item reduction of pyruvate to ethanol
  \item reduction of pyruvate to lactate
\end{enumerate}
\begin{figure}[!ht]
  \centering
  \schemedebug{false}
  \schemestart
  \chemname{\chemfig{C(-[6]CH_{3})(=O)-[2]C(=[3]O)-[1]O^{-}}}{pyruvate}\arrow(.mid east--.mid west){-U>[\tiny \chemfig{ H^{+} }][\tiny \chemfig{CO_{2}}]}\chemname{\chemfig{C(=O)(-[2]H)-[6]CH_{3}}}{acetylaldehyde}\arrow(.mid east--.mid west){-U>[\tiny \chemfig{NADH + H^{+}}][\tiny NAD][][0.3][100]} \chemname{\chemfig{C(-[2]OH)(-[4]H)(-[6]CH_{3})-H}}{ethanol}
  \schemestop

  \vspace{1cm}

  \schemestart
  \chemname{\chemfig{C(-[6]CH_{3})(=O)-[2]C(=[3]O)-[1]O^{-}}}{pyruvate}\arrow(.mid east--.mid west){-U>[\tiny\chemfig{NADH + H^{+}}][\tiny\chemfig{NAD^{+}}]}\chemname{\chemfig{C(-[4]H)(-[6]CH_{3})(-OH)-[2]C(=[3]O)-[1]O^{-}}}{lactate}
  \schemestop

\end{figure}

\subsection{The Pyruvate Dehydrogenase Complex}

Pyruvate is oxidatively decarboxylated by PDC before entering the Krebs cycle
\emph{releasing \chemfig{CO_{2}} and NADH} and becoming an activated acetyl unit
attached to CoA. AMP is an inhibitor of PDC.

\begin{figure}[h]
  \centering
  \schemestart
  \chemname{\chemfig{C(-[6]CH_{3})(=O)-[2]C(=[3]O)-[1]O^{-}}}{pyruvate} + \chemfig{CoA-SH}\arrow(.mid east--.mid west){-U>[\tiny \chemfig{NAD^{+}}][\tiny \chemfig{NADH + H^{+}+CO_{2}}][][0.5]}\chemname{\chemfig{CoA-S-C(=[2]O)-CH_{3}}}{acetyl-CoA}
  \schemestop	
\end{figure} 

A \vocab{prosthetic group} is a nonprotein molecule (cofactor) tightly
covalently bound to an enzyme as part of the enzyme's active site and is
necessary for proper function. PDC and $ \alpha\text{-ketoglutarate
dehydrogenase complex} $ contains thiamine pyrophosphate (TPP) prosthetic group.

\subsection{Krebs Cycle}

\emph{Produces total of 6 \chemfig{NADH}, 2 \chemfig{FADH_{2}}, and 2 GTP per
glucose (will transfer P to ADP later on).}\\

\subsubsection*{Stage 1}

The 2 carbons in acetate of acetyl-CoA are condensed with the four carbon
compound \vocab{oxaloacetate} (OAA) to make \vocab{citrate}. 

\subsubsection*{Stage 2}

Each oxidative decarboxylation produces NADH and \chemfig{CO_{2}}. The carbons
that entered via acetate are not lost in their first cycle but in later cycles.

\subsubsection*{Stage 3}

OAA is regenerated and produces 1 \chemfig{NADH}, 1 \chemfig{FADH_{2}}, and 1
\chemfig{GTP} in the process.

\begin{figure}[h]
  \centering
  \includegraphics[scale=0.1,frame]{KrebSum.jpg}
  \caption{Cellular Respiration}
\end{figure}

\subsection{Compartmentalization of Glucose Catabolism in Eukaryotes: The
Mitochondria}

Components of mitochondria:
\begin{enumerate}
  \item outer membrane - smooth and contains large pores formed by \vocab{porin}
  \item inner membrane - impermeable and densely folded into \vocab{cristae}.
    (electron transport) 
  \item matrix - innermost space of mitochondrion (Krebs
    cycle)
  \item intermembrane space - continuous with cytoplasm
\end{enumerate}

2 NADH comes from glycolysis in the cell membrane and the rest occur in Krebs in
the matrix. In prokaryotes since no membrane bound organelle, just use cell
membrane.

\subsection{Electron Transport and Oxidative Phosphorylation}

\vocab{Oxidative phosphorylation} is the oxidation of the high-energy electron
carriers NADH and \chemfig{FADH_{2}} coupled to the phosphorylation of ADP.
Electron transport chain is composed of 5 electron carriers three of which are
large complexes classified as \vocab{cytochromes}. In order electrons flow
through: 
\begin{infobox}
  \begin{enumerate}
    \item NADH dehydrogenase (MB)
    \item ubiquinone (Coenzyme Q) - \chemfig{FADH_{2}} enters here
    \item Cytochrome C reductase (MB)
    \item Cytochrome C
    \item Cytochrome C oxidase (MB)
  \end{enumerate}
\end{infobox}

The three (MB) complexes pump protons into intermembrane space generating proton
gradient. Protons fall down gradient through ATP synthase to produce ATP.

\subsection{Energetics of Glucose Catabolism}

10 protons pumped per NADH, in sequence of complexes it is 4,3,3. Since
\chemfig{FADH_{2}} (and NADH from glycolysis) enters at CoQ only 6 protons are
pumped. ATP production requires 4 protons. Refer to the figure below for
counting of protons.

\begin{figure}[h]
  \centering
  \includegraphics[scale=0.1,frame]{ATPtotal.jpg}
\end{figure}

\subsection{Other Metabolic Pathways}

\subsubsection{Glycogenolysis}

Breakdown of glycogen, a polymer of glucose, the main form of carbohydrate
storage in animals. Glycogenolysis occurs in response to hormone glucagon, when
blood sugar levels are low. Similarly, in plants except they use starch.
Glycogen and starch are $\alpha-1,4$ and $ \alpha-1,6 $ glycosidic bonds
respectively.

\subsubsection{Gluconeogenesis}

Production of glucose primarily in the liver. Uses intermediates of Krebs cycle,
pyruvate, lactate, and carbon skeleton of amino acids to essentially run
glycolysis-in-reverse.

\subsubsection{Amino Acid Catabolism}

Manufactured proteins are eventually broken down back into aa. From diet, the
carbon skeleton of aa ($ \alpha-keto $ acid) can be broken into water and
\chemfig{CO_{2}} or converted to glucose or acetyl-CoA.

\subsubsection{Pentose Phosphate Pathway (PPP)}

Diverts glucose-6-phosphate to form ribose-5-phosphate which can be used to
synthesize other nucleotides. Generates NADPH via \vocab{glucose-6-phosphate
dehydrogenase} (G6PDH), commonly involved in anabolic processes and removal of
reactive oxygen species. A deficiency of G6PDH can lead to cell death.

\subsection{Metabolic Summary}

\begin{figure}[h]
  \centering
  \includegraphics[scale=0.1,frame]{MetaSum.jpg}
\end{figure}

\section{Metabolic Regulation}

Important to prevent antagonist processes from occurring at the same time.

\subsection{Regulation of Glycolysis and Gluconeogenesis}

Regulation focused on enzymes catalyzing irreversible reactions. Two
allosterically regulated opposing enzymes are PFK and
\vocab{fructose-1,6-bisphosphatase} (F-1,6-BPase). Recall AMP activates PFK and
therefore deactivates F-1,6-BPase. Another regulatory intermediate is
\vocab{fructose-2,6-bisphosphate} (F-2,6-BP) whose intracellular concentration
is controlled by PFK-2 (synthesizes) and F-2,6-BPase (breaksdown).
\emph{Insulin} and \emph{glucagon}.\\
Blood glucose falls, glucagon released
binds to liver cells, production of cAMP and activation of \vocab{protein kinase
A} leading to deactivation of PFK-2 and activation of F-2,6-BPase (via
phosphorylation). Leads to increased consumption of F-2,6-Bp enhancing activity
of F-1,6-BPase and decreasing PFK. Net result in gluconeogenesis and an
inhibition of glycolysis.\\
Summary in the figure below

\begin{figure}[h]
  \centering
  \includegraphics[scale=0.1,frame]{Gly_Glu_reg.jpg}
\end{figure}

\subsection{Regulation of the Krebs Cycle}

Conversion of pyruvate to acetyl-CoA is highly regulated since acetyl-CoA cannot
be used in gluconeogenesis. Enzymes that produce exergonic steps in Krebs are
also regulated e.g. \vocab{isocitrate dehydrogenase}.

\subsection{Regulation of Glycogen Synthesis and Glycogenolysis}

\vocab{Glycogen synthase} and \vocab{Glycogen phosphorylase} (catabolism of
glycogen) are reciprocally controlled.

\subsection{Overview} Some guidelines: 
\begin{enumerate} 
  \item Enzymes that catalyze irreversible steps are frequently sites of
    regulation 
  \item Increased concentrations of intermediates in a pathway generally serve
    to decrease the activity of that pathway 
  \item Each pathway responds to the
    energy state of the cell 
\end{enumerate}

\begin{figure}[h] \centering \includegraphics[scale=0.1,frame]{MetaReg.jpg}
\end{figure}

\section{Fatty Acid Metabolism}

\subsection{Fatty Acid Oxidation}

Following initial steps of digestion, \vocab{chylomicrons} composed of fat and
lipoprotein go to liver, heart, lungs, and other organs. The triacylglycerol is
then hydrolyzed to liberate free fatty acids which can then undergo several
rounds of $ \boldsymbol{\beta}- $\textbf{oxidation} each round producing an
acetyl-CoA until the last producing four-carbon acyl-CoA to generate two
acetyl-CoA. Process begins at outer mitochondrial membrane, catalyzed by
\vocab{acyl-CoA synthetase}, moves into matrix and cleaved to form acetyl-CoA.
Other products include $ - $2 ATP, +1 \chemfig{FADH_{2}} and +1 NADH. \\
For unsaturated double bond fatty acids an isomerase is neeed to move the double
bond. If more than one double bond then need isomerase and reductase.

\subsection{Ketogenesis}

During times of starvation, body produces \vocab{ketone bodies} including
acetone, acetoacetate, and $ \beta $-hydroxybutyrate which can be converted back
to acetyl-CoA upon arrival at the target organ and enter Krebs cycle. 

\subsection{Fatty Acid Synthesis}

Similar to $ \beta $-oxidation with several exceptions. Takes place in
cytoplasm. Repeated addition of two-carbon units. Acetyl-CoA is first activated
in a carboxylation reaction (initial committed step) and facilitated by
\vocab{acetyl-CoA carboxylase} to generate \vocab{malonyl-CoA}.\\
\vocab{Fatty acid synthase} catalyzes a decarboxylation reaction where
malonyl-CoA provides two carbons to the growing fatty acid. Synthesis requires
NADPH obtained from pentose phosphate pathway (PPP).  

\end{document}
