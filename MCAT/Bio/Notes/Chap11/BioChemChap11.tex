\documentclass[../Bio_chemistryReview.tex]{subfiles}

\begin{document}
\chapter{The Circulatory, Lymphatic, and Immune Systems}
\section{Overview of the Circulatory System}
The circulatory system does the following:
\begin{enumerate}
  \item Distribute nutrients from the digestive tract, liver, and adipose tissue
  \item Transport oxygen from the lungs to the entire body and carbon dioxide
    from the tissues to the lungs
  \item Transport metabolic waste products from tissues to the excretory system
  \item Transport hormones from endocrine glands to targets and provide feedback
  \item Maintain homeostasis of body temperature
  \item \textit{Hemostasis} (blood clotting)
\end{enumerate}
The flow of blood through a tissue is known as \textbf{perfusion}. Inadequate
blood flow is termed \vocab[]{ischemia} and adequate circulation but lack of
oxygen to tissue is called \vocab[]{hypoxia}.

\subsection{Components of the Circulatory System}
\begin{wrapfigure}{r}{0.4\textwidth}
  \centering
  \vspace{-11pt}
  \includegraphics[width=0.4\textwidth,frame]{Pul.jpg}
  \caption{Pulmonary and Systemic Circuits}
  \label{fig:pul}
\end{wrapfigure}
Vessels that carry blood away from the heart are \vocab[]{arteries} and those
that carry blood towards the heart are \vocab[]{veins}. Further away from the
heart, arteries branch into smaller arteries called \vocab[]{arterioles} and
then further into \vocab[]{capillaries} where the walls are thin enough for the
exchange of materials between the blood and tissue. After passing through
capillaries, blood collects in small veins called \vocab[]{venules} and then
back into veins leading back to the heart.\par

The inner lining of all blood vessels is formed by a thin layer of
\vocab[]{endothelial cells}. These cells function in:
\begin{itemize}
  \item Vasodilation and vasoconstriction regulate diameter and thus blood
    pressure.
  \item Inflammation promotes expression of adhesion molecules near sites of
    injury.
  \item Angiogenesis (the formation of new blood vessels).
  \item Thrombosis (blood clotting).
\end{itemize}
The heart is separated into the right and left side. The right side pumps blood
to the lungs through \vocab[]{pulmonary circulation} and the left side pumps
blood to the rest of the body through \vocab[]{systemic circulation}. By having
two circulations, most blood passes through only one set of capillaries before
returning to the heart. \vocab[]{Portal systems} are an exception to this. The
portal systems evolved as direct transport systems, to transport nutrients
directly from the intestine to the liver or hormones form the hypothalamus to
the pituitary, without passing through the whole body. There are three of these
in total (Hepatic, Renal, Hypophyseal).
The pulmonary and systemic circuits are shown in Figure \ref{fig:pul}

\section{The Heart}
There are two kinds of chambers in the heart. The \vocab[]{atria} serve as the
``waiting rooms'' for blood returning to the heart from the veins and the
\vocab[]{ventricles} pump the blood. So blood goes in an AVAV cycle -
\ul{A}rteries, \ul{V}eins, \ul{A}trium, \ul{V}entricle. The right atrium
receives deoxygenated blood from the systemic circulation through the
\vocab[]{inferior vena cava} and the \vocab[]{superior vena cava}. The left
ventricle pumps oxygenated blood out through the \vocab[]{aorta}. The heart gets
its oxygenated blood from the \vocab[]{coronary arteries} which branch from the
aorta and surround the heart. The deoxygenated blood then travels through the
\vocab[]{coronary veins} and collects in the \vocab[]{coronary sinus} which
drains directly into the right atrium.

\subsection{Valves}
Valves ensure one-way flow. In particular, ventricular pressure is very high and
atrial pressure is lower so the \vocab[]{atrioventricular valve} (AV valve)
between each ventricle and its atrium is necessary to prevent backflow. The
valve on the left side of the heart is called the \vocab[]{bicuspid valve} or
\vocab[]{mitral valve}. For the right side it is called the \vocab[]{tricuspid
valve}. The valves between the large arteries and the ventricles is the
pulmonary and aortic semilunar valves.

\subsection{The Cardiac Cycle}
The cycle is separated into two parts:
\begin{enumerate}
  \item \vocab[]{systole} the contraction period
  \item \vocab[]{diastole} the relaxation period
\end{enumerate}
Now think about all of the things mentioned up to this section and rebuild the
circulation of blood through the heart.

\subsection{Heart Sounds, Heart Rate, and Cardiac Output}
One can measure the \vocab[]{cardiac output} from knowing the \vocab[]{stroke
volume} and the heart rate. The stroke volume is the volume that is pumped per
heart beat. Note that the blood output of either ventricle is the same which
must be true from a physical standpoint.

\subsection{The Frank-Starling Mechanism and Venous Return}
The \vocab[]{Frank-Starling mechanism} states that if the heart muscle is
stretched then it will contract more forcefully. Thus, if more blood is pumped
into the heart and the muscle is stretched then it will contract more
forcefully. This can significantly increase the stroke volume.

\subsection{Cardiac Muscle}
All muscle cells share with neurons the ability to propagate an action potential
across their surface. Cardiac muscle is a \vocab[]{functional syncytium} which
is a tissue in which the cytoplasm of different cells can communicate via gap
junctions, which for cardiac muscles are found in the \vocab[]{intercalated
disks}. The atria and ventricle are separate syncytia, and the action potential
for contraction goes from atria to ventricles by the \vocab[]{cardiac conduction
system}. It is delayed as it passes through the A-V node. This is necessary
because we can't have all the heart contract at the same time.\par

\subsection{Rhythmic Excitation of the Heart}
\begin{wrapfigure}{r}{0.5\textwidth}
  \centering
  \vspace{-11pt}
  \includegraphics[width=0.5\textwidth,frame]{SA.jpg}
  \caption{The Pacemaker Potential of the SA Node}
  \label{fig:SA}
\end{wrapfigure}
The \vocab[]{sinoatrial (SA) node} provides the initiation of each action
potential. The action potential is divided into 3 separate phases: Phase 0,
Phase 3, and Phase 4. The SA node has an \textit{unstable resting potential}.
\vocab[]{Phase 4} slowly depolarizes (the potential becomes less negative)
through \vocab[]{sodium leak channels} until the potential for the voltage-gated
calcium channels is reached and then \vocab[]{Phase 0} occurs with an influx of
calcium ions. \textit{Note: that this is opposite of all other action potentials
with respect to calcium and sodium ion usage.} \vocab[]{Phase 3} is
repolarization through closure of Ca\textsuperscript{2+} channels and opening of
K\textsuperscript{+} channels for the outward flux of those ions. Figure
\ref{fig:SA} summarizes the process.\par

%\begin{wrapfigure}{r}{0.5\textwidth}
%  \centering
%  \includegraphics[width=0.5\textwidth,frame]{Phase.jpg}
%  \caption{Phases of the Membrane Potential in a Cardiac Muscle Cell}
%  \label{fig:Phases}
%\end{wrapfigure}
For the cardiac muscle cells they have a resting membrane potential of -90mV and
a longer duration action potentials. They have Phases 0--4 summarized in Figure
\ref{fig:Phases}.
\begin{figure}[H]
  \centering
  \includegraphics[scale=0.3,frame]{Phase.jpg}
  \caption{Phases of the Membrane Potential in a Cardiac Muscle Cell}
  \label{fig:Phases}
\end{figure}
So for each heartbeat,
\begin{enumerate}
  \item AP starts at SA node
  \item spreads through atria and simultaneously down the special conduction
    pathway (aka \vocab[]{internodal tract}) reaching the AV node almost
    instantaneously while the other AP is slowed by the atria muscle cell
    conduction.
  \item AV node delays the AP and then relays to the ventricles through the
    \vocab[]{AV bundle} (\vocab{bundle of HIS}).
  \item AV bundle divides into right and left bundle branches and then into the
    Purkinje fibers to spread evenly (horizontally) over the ventricles towards
    the inferior portion of the ventricles (located at the bottom of the heart).
  \item The inferior portion contracts first and then the superior. (If it
    sounds backwards it is).
\end{enumerate}
\begin{figure}[H]
  \centering
  \includegraphics[scale=0.3,frame]{Card.jpg}
  \caption{The Cardiac Conduction System}
\end{figure}
 
\subsection{Regulation of the Heart by the autonomic Nervous System}
The automaticity of heart contractions is regulated by the parasympathetic
system through the vagus nerve via the release of acetylcholine to bind to
inhibit depolarization. Note that Ach is usually a stimulatory neurotransmitter.
The sympathetic nervous system can stimulate the heart in the ``fight or
flight'' response.

\section{Hemodynamics}
\subsection{Resistance}
The driving force for blood flow is a difference in pressure from arteries to
veins. It can be modeled by a ``Ohm's Law''-like relation
\begin{equation}
  \Delta P = Q \times R
\end{equation}
where $ \Delta P $ is the pressure gradient and $ Q $ is the blood flow in
(L/min) and $ R $ is the \vocab{peripheral resistance} controlled by the
sympathetic nervous system. The principal determinant of resistance is the
constriction of arteriolar smooth muscle, also known as \vocab[]{precapillary
sphincters}. Smaller vessels \imp higher resistance. 

\subsection{Blood Pressure}
The blood pressure that is measured is the \vocab[]{systemic arterial pressure}
which is the pressure on the walls of the arteries. Normal is 120/80 where 120
is the \vocab[]{systolic pressure} and 80 is the \vocab[]{diastolic pressure}.
The difference between these is the \vocab[]{pulse pressure}.\par

\subsection{Local Autoregulation}
Locally, the body cannot regulate blood flow as this would require too much
overhead. Instead, the metabolic waste products can cause arteriolar smooth
muscles to relax and thus dilate the arteries and increase blood flow. This
process is called \vocab[]{local autoregulation}.

\section{Components of Blood}
The blood is composed of cells and a liquid portion called \vocab[]{plasma}. The
plasma contains multiple components. The principal sugar in the blood is
glucose. \vocab[]{Albumin} is essential for maintenance of \vocab[]{oncotic
pressure} (osmotic pressure in the capillaries). The blood also carries
immunoglobulins and \vocab[]{fibrinogen} (blood clotting). \vocab[]{Lipoproteins}
which carry around lipids in the blood stream. The waste products are
\vocab[]{urea} and \vocab[]{bilirubin} (from the breakdown of hemoglobin).
\vocab[]{Hematocrit} is the volume of blood occupied by red blood cells.

\subsection{Blood Typing}
Blood typing depends on two blood groups which combine together to give the
phenotype expressed on blood cells, the ABO blood group and the Rh blood group.
The following table gives the possible combinations:
\begin{table}[H]
  \centering
  \caption{Blood Group Genotypes and Phenotypes}
  \begin{tabularx}{0.8\textwidth}{XXXXX}
    \toprule
    & $ I^{A}I^{A} $ or $ I^{A}i $ & $ I^{B}I^{B} $ or $ I^{B}i $ & $
    I^{A}I^{B} $ & $ ii $\\
    \midrule
    RR or Rr & type A+ & type B+ & type AB+ & type O+\\
    rr & type A- & type B- & type AB- & type O-\\
    \bottomrule
  \end{tabularx}
\end{table}

\section{Transport of Gases}

\subsection{Oxygen \& Carbon Dioxide}
Oxygen uses \vocab[]{hemoglobin} to bind oxygen to transport it. Hemoglobin has
four sites that do not bind oxygen independently. As oxygen binds, the affinity
increases going from the \vocab[hemoglobin!]{tense} state to the
\vocab[hemoglobin!]{relaxed} state. Thus, hemoglobin binds
\vocab[hemoglobin!]{cooperatively}. Factors that stabilize the tense state are
called the \vocab[]{Bohr effect} and include:
\begin{enumerate}
  \item Decreased pH
  \item Increased $ P_{CO_{2}} $
  \item Increased temperature
\end{enumerate}
Carbon dioxide is transported in the blood through bicarbonate, attaching to
certain proteins on hemoglobin (it stabilizes the tense state), and solubility
in the blood.

\subsection{Exchange of Substances Across the Capillary Wall}
Three types of substances must pass the capillary walls: nutrients, wastes, and
white blood cells; through the \vocab[]{intercellular clefts}.\par

Amino acids and glucose go from the digestive tract to the liver through the
\vocab[]{hepatic portal vein}. Fats are packaged into \vocab[]{chylomicrons}
which eventually end up in the liver which converts them into another type of
lipoprotein which carries the fats to the fat cells.\par

Wastes to be excreted from the digestive area is passed into the gut as
\vocab[]{bile}. Macrophages and neutrophils can pass through the clefts via
amoeboid motility.\par

Water leaks out of the capillaries due to the pressure created by the heart and
the osmolarity of the tissues. The osmotic pressure of plasma,
\vocab[]{oncotic pressure} is provided by high concentrations of Albumin.
However, most water does flow back in towards the ends of the capillaries due to
decrease in hydrostatic pressure. Fluids, proteins, and white blood cells are
returned to the bloodstream via the lymphatic system.

\section{The Lymphatic System}
The \vocab[]{lymphatic system} is a one-way flow. It begins with tiny lymphatic
capillaries in the tissues that merge to form larger lymphatic vessels then
ducts then the \vocab[]{thoracic duct} which empties out into a large vein in
the neck. The \vocab{lymph} is filtered through \vocab[]{lymph nodes}.

\section{The Immune System}
This section is omitted because I took the class for it.

\end{document}
