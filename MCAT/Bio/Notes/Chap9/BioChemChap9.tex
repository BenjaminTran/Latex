\documentclass[../Bio_chemistryReview.tex]{subfiles}

\begin{document}
\chapter{Genetics and Evolution}
\section{Introduction to Genetics}
DNA is the genetic information of heritability. The first section of the book
discusses the experiments leading to this discovery. Probably of not much
relevance to the MCAT so it will be omitted here.

\subsection{Genes and Alleles}
\vocab[]{Genes} are the basic units of inheritance. \vocab[]{Alleles} are the
different types of these genes. There are many different types of alleles but
normally an individual only has two of them. The \vocab[]{locus} is the specific
location of a gene on a specific chromosome.\par

The human genome is split into 23 different chromosomes, but each individual has
two of each chromosome. In the pair, one is from the father and the other is
from the mother. The difference is in the DNA sequence which will code for the
different alleles of the gene.

\subsection{Genotype vs. Phenotype}
The \vocab[]{genotype} is the DNA sequence of the alleles a person carries. If
there are two different alleles at a given locus then the person is termed a
\vocab[]{heterozygote}, the definition of \vocab[]{homozygote} follows. The
\vocab[]{phenotype} is the physical expression of the genotype. If the phenotype
expressed is of one allele regardless of the phenotypic expression of the other
allele then the allele is said to be \vocab[Alleles!]{dominant} and the other is
said to be \vocab[Alleles!]{recessive}. Of course this can only apply to diploid
organisms.\par

\section{Meiosis}
\begin{wrapfigure}{r}{0.45\textwidth}
  \centering
  \vspace{-11pt}
  \includegraphics[width=0.45\textwidth,frame]{ProMetI.jpg}
  \caption{Prophase I and Metaphase I}
  \label{fig:pro}
\end{wrapfigure}

Production of haploid gametes requires halving the number of chromosomes.
Mitosis does not do this but \vocab[]{meiosis} does. Specialized cells termed
\vocab[]{spermatogonia} and \vocab[]{oogonia} undergo meiosis. Spermatogenesis
and oogenesis share the same basic features of meiosis but differ in many of the
specific features of gamete production.\par

Mitosis and meiosis are very similar. Both undergo one round of replication but
meiosis has recombination between homologous chromosomes. Since after
replication there are four copies of the genome, meiosis will divide twice
during \vocab[meiosis!]{meiosis I} and \vocab[meiosis!]{meiosis II} to produce
four haploid gametes. 

In meiotic prophase I, the difference from mitosis is that homologous
chromosomes pair with each other precisely in \vocab[]{synapsis} to undergo
recombination also called \vocab[]{crossing over}. Synapsis is mediated by a
protein structure called the \vocab[]{synaptonemal complex} (SC). Its job is to
connect the homologous chromosomes in a precise manner. The paired homologous
chromosomes are called a \vocab[]{bivalent} or \vocab[]{tetrad}. See Figure
\ref{fig:pro} for an artistic representation.\par

In metaphase I the tetrads (opposed to the sister chromatids in mitosis) align
at the center of the cell, the metaphase plate. In anaphase I, homologous
chromosomes separate, and sister chromatids remain together. The cell then
divides in telophase I. Each cell now has a single set of chromosomes and are
thus haploid. But each chromosome is made of two identical sister chromatids so
the second division will split these sister chromatids into individual cells.

\subsection{Nondisjunction}
Sometimes during meiosis I and meiosis II the sister chromatids and chromosomes
fail to separate. This is called \vocab[]{nondisjunction}. This can result in
gametes not having a chromosome and therefore some gametes having multiple
chromosomes. The result can be a zygote with three copies of a chromosome
(\vocab[]{trisomy}) or one copy of a chromosome (\vocab[]{monosomy}).

\section{Mendelian Genetics}
Mendel has two laws of genetics:
\begin{infobox}
\begin{description}
  \item [\vocab{Law of segregation}] states that the two alleles of an
    individual are separated and passed on to the next generation singly.
  \item [\vocab{Law of independent assortment}] states that the alleles of one
    gene will separate into gametes independently of alleles for another gene.
    An exception to this is if two genes of interest are right next to each
    other on the chromosome then a phenomenon called \vocab[]{linkage} occurs
    and independent assortment is violated.
\end{description}
\vspace{5pt}
\end{infobox}
Consider pea plants: A \vocab[]{pure-breeding strain} refers to a line of pea
plants that consistently yields the same phenotypic progeny. Using a
pure-breeding strain, one can determine the genotype of an unknown by
performing a \vocab[]{test-cross} where the plant of interest is crossed with
another individual that is homozygous recessive. The progeny are called the
\vocab[]{F\textsubscript{1} generation}.

\subsection{The Punnett Square}
The punnett square is straightforward. For keeping track of two traits then the
possible gametes are the possible permutations e.g. (see Figure \ref{fig:Pun})
\begin{figure}[H]
  \centering
  \includegraphics[scale=0.2,frame]{Pun.jpg}
  \caption{A Punnett Square Depicting a Cross with Two Traits Involved}
  \label{fig:Pun}
\end{figure}

\section{Extending Mendelian Genetics}
Some different types of inheritance patterns are discussed in this section.

\subsection{Incomplete Dominance}
\vocab[]{Incomplete dominance} is when the phenotype expressed is a blend of
both alleles. Alleles that are incompletely dominant are always given upper-case
letters.

\subsection{Codominance}
\vocab[]{Codominance} is a situation in which two alleles are both expressed but
\textit{ not blended }. The example given is that of the ABO blood group. Here
the genes code for certain antigens I\textsuperscript{A},
I\textsuperscript{B}, and i where i does not code for an antigen. Both
I\textsuperscript{A} and I\textsuperscript{B} are codominant so the ABO blood
group will have type AB blood.\par

Some additional terms in regard to inheritance:
\begin{infobox}
\begin{description}
  \item \vocab[]{Pleiotropism} If the expression of a gene alters many
    different, seemingly unrelated aspects of the organism's total phenotype.
    For example, a mutation in a gene may cause altered development of heart,
    bone, and inner ears.
  \item \vocab[]{Polygenism} Complex traits that are influenced by many different
    genes. These traits tend to display a range of phenotypes in a continuous
    distribution.
  \item \vocab[]{Penetrance} This describes the likelihood that a person with a
    given genotype will express the expected phenotype. Consider the fact that
    many people may carry the mutation for breast cancer but the probability of
    occurrence is not absolute.
  \item \vocab[]{Epistasis} When expression of alleles for one gene is dependent
    on a different gene.
  \item \vocab[]{Recessive Fetal Alleles} These are alleles that are recessive
    and will express lethal traits. 
\end{description}
\end{infobox}

\subsection{The Sex Chromosomes}
Males are XY and females are XX. Thus, clearly it is the male who determines
the sex of the child. Since males only have one X chromosome then any trait
coded for by the X chromosome will be expressed regardless if it is recessive or
not. \vocab[]{Sex-linked traits} will be discussed presently.

\section{Linkage}
\begin{wrapfigure}[12]{r}{0.4\textwidth}
  \centering
  \vspace{-11pt}
  \includegraphics[width=0.4\textwidth,frame]{linkage.jpg}
  \caption{Linkage of Alleles during Meiosis}
  \label{fig:link}
\end{wrapfigure}
If genes are located very close to each other on the same chromosome, then they
will probably not be inherited independently of each other since crossing over
occurs by switching chunks of chromosomes. Thus, Figure \ref{fig:link} shows
what can occur due to linkage. A different kind of Punnett square can be drawn
considering the linkage of the possible alleles. Note that the linkage must be
specified to you in the problem.

\subsection{Linkage and Recombination}
Meiotic recombination provides the condition of linkage (note: the book makes
the typo of saying an exception to linkage). Genes that are close together on
the chromosome may break off with each other during recombination and so these
genes are ``linked''. The frequency of recombination is proportional to the
physical distance between the genes along the linear length of the DNA molecule.
The \vocab[]{frequency of recombination} is defined as
\[ \text{RF} = \frac{\mbox{number of recombinants}}{\mbox{total number of
offspring}} \]
\vocab{ Maximal frequency } of recombination is the frequency in the case that
there is no linkage and the genes are assorted independently.

\section{Inheritance Patterns and Pedigrees}
There are six \vocab{inheritance patterns} to be familiar with:
\vocab[inheritance patterns!]{autosomal recessive}, \vocab[inheritance
patterns!]{autosomal dominant}, \vocab[inheritance patterns!]{mitochondrial},
\vocab[inheritance patterns!]{Y-linked}, \vocab[inheritance patterns!]{X-linked
recessive}, and \vocab[inheritance patterns!]{X-linked dominant}. See Table
\ref{tab:inh}
\begin{figure}[H]
  \centering
  \includegraphics[scale=0.35,frame]{sumInh.jpg}
  \caption{Summary of Inheritance Patterns}
  \label{tab:inh}
\end{figure}
\noindent\vocab[]{Pedigrees} are used to study inheritance through family trees. The
following conventions are used:
\begin{enumerate}
  \item Males are squares and females are circles
  \item A cross is represented by a horizontal line
  \item Offspring are connected to their parents by a vertical line, and to each
    other by a horizontal line with vertical branches for each sibling
  \item Offspring of unknown gender (unborn) are represented by a diamond shape
  \item Afflicted individuals are shaded.
\end{enumerate}
Individuals marrying in are \textit{assumed} to be homozygous normal unless their
phenotype tells you differently. To analyze a pedigree follow these steps:
\begin{description}
  \item[Step 1:] Is the allele that causes the trait dominant or is it
    recessive? Recessive traits commonly skip generations but dominant traits do
    not.
  \item[Step 2:] Is the gene involved sex-linked? If so, there will be an
    unequal distribution of affected males.
  \item[Step 3:] If it is sex-linked is it X or Y? Y-linked will have
    father-to-son transmission, while X-linked will not.
  \item[Step 4:] Check for mitochondrial inheritance. Affected females will have
    all affected children, but affected males cannot pass the trait on.
  \item[Step 5:] Figure out the genotypes and calculate the probabilities of
    inheritance where necessary. When writing genotypes for sex-linked traits,
    make sure to include the chromosomes (e.g. X\textsuperscript{A}Y, or
    X\textsuperscript{A}X\textsuperscript{a}, etc.). When writing genotypes for
    autosomal traits, make sure NOT to include the chromosomes.
  \item[Step 6:] If more than one trait is involved, go through 1--5 for each.
\end{description}

\section{Population Genetics}
\vocab[]{Population genetics} describes the inheritance of traits in populations
over time. The sum total of all genetic information in a population is called
the \vocab[]{gene pool}. So for instance, if there are 5000 hippos in a
population, out of which there are 100 homozygotes of an autosomal allele h and
400 heterozygotes, then the frequecny of the h allele in the population is 
\[ \frac{200 + 400}{2*5000} = 0.06 \]

\subsection{Hardy-Weinberg in Population Genetics}
\begin{infobox}
The \vocab[]{Hardy-Weinberg law} states that the \textit{frequencies of alleles
in the gene pool of a population will not change over time,} provided that a
number of assumptions are true:
\begin{enumerate}
  \item There is no mutation
  \item There is no migration
  \item There is no natural selection
  \item There is random mating
  \item The population is sufficiently large to prevent random drift in allele
    frequencies
\end{enumerate}
\end{infobox}
We can model this with probability. Define $ p $ to be the frequency of a
dominant allele and $ q $ to be the frequency of the recessive allele in a
population, then if there are only two alleles for the gene then the following
must hold
\begin{equation}
  p + q = 1
\end{equation}
and of course squaring both sides implies
\begin{equation} 
  p^{2}  + 2pq + q^{2} = 1
\end{equation}
where
\begin{align*}
  p^{2} &= \mbox{the frequency of the GG genotype} \\
  2pq &= \mbox{the frequency of the Gg genotype} \\
  q^{2} &= \mbox{the frequency of the gg genotype}
\end{align*}
After one generation a population will reach \vocab[]{Hardy-Weinberg
equilibrium}. So in a problem statement the P generation is not at
Hardy-Weinberg equilibrium but the F\textsubscript{1} generation will be. Note
that in reality the conditions of Hardy-Weinberg equilibrium are not naturally
met. Just think about how easy it is to break the assumptions listed.

\section{Evolution by Natural Selection}
The theory of \vocab[]{evolution} by \vocab[]{natural selection} is encapsulated
as follows:
\begin{enumerate}
  \item In a population, there are heritable differences between individuals.
  \item Heritable traits (alleles of genes) produce traits (phenotypes) that
    affect the ability of an organism to survive and have offspring.
  \item Some individuals have phenotypes that are advantageous for survival
    and hence reproduction passing on their alleles. Eventually these alleles
    dominate.
\end{enumerate}
In evolution \vocab[]{fitness} refers to an animals ability to survive long
enough to reproduce.

\subsection{Sources of Genetic Diversity}
Natural selection can only alter allele frequencies of existing alleles. It
cannot cause new alleles to appear in the population. Also, note that mutations
introduce new alleles to the population only if the mutations occur in the
germ line.

\subsection{Modes of Natural Selection}
There are several different manners in which natural selection works:

\begin{enumerate}
  \item \vocab[]{Directional selection} Normally polygenic traits follow a
    bell-shaped curve (of course the peak is the average). If an event kills one
    tail of the curve then the average shifts away from that end.
  \item \vocab[]{Divergent selection} When events kill the members near the
    average of the bell curve then over time the tail ends become their own
    populations. This can possible lead to a new species.
  \item \vocab[]{Stabilizing selection} When both extremes of the curve are
    selected against and everyone is driven towards the average
  \item \vocab[]{Artificial selection} Humans forcing certain animals to mate
    with each other
  \item \vocab[]{Sexual selection} Animals don't mate randomly and so have a
    preference for certain traits
  \item \vocab[]{Kin selection} Animals that live socially often share alleles
    and will sacrifice oneself to protect said alleles in another.
\end{enumerate}

\section{The Species Concept and Speciation}
A \vocab[]{species} is a group of organisms which are capable of reproducing
with each other sexually and have a \textit{fertile} offspring. This does not
mean animals of different species cannot have offspring as a horse and a donkey
may produce a mule which is healthy but will never be fertile. Note that a
population are those members that \textit{do} reproduce with each other and
hence a population is always a subset of the species. \vocab[]{Reproductive
isolation} keeps existing species separate. There are two types:
\begin{description}
  \item \vocab[Reproductive isolation!]{Prezygotic} barriers prevent the
    formation of a hybrid zygote.
  \item \vocab[Reproductive isolation!]{Postzygotic} barriers prevent the
    development, survival, or reproduction of hybrid individuals e.g.\ the
    mule cannot produce offspring.
\end{description}

The creation of new species is known as \vocab[]{speciation}. Different kinds of
speciation occur:
\begin{description}
  \item[Cladogenesis] branching speciation where one species diversifies and
    becomes two or more new species
  \item[Anagenesis] is when one biological species simply becomes another by
    changing so much that an individual would not be able to reproduce with its
    ancestors.
  \item[Allopatric isolation] is a type of cladogenesis initiated by
    geographical isolation that causes reproductive isolation
  \item[Sympatric] is when a species gives rise to a new species in the same
    geographical area
\end{description}
Cladogenesis has led to \vocab[]{homologous structures} which are physical
features shared by two different species as a result of a common ancestor.
\vocab[]{Analogous structures} serve the same function in two different species,
but not due to common ancestry e.g. sperm flagella vs. bacterial flagella.
\vocab[]{Convergent evolution} is when two different species come to possess
many analogous structures due to similar selective pressures. The opposite of
convergent evolution is \vocab[]{divergent evolution} which is the usual
evolution people think about. \vocab[]{Parallel evolution} describes the
situation in which two species go through similar evolutionary changes due to
similar selective pressures. This may be confusing with the convergent evolution
definition, so think about birds and bats for convergent evolution and organisms
evolving in the Ice Age to tolerate cold for parallel evolution.

\section{Taxonomy}
\vocab[]{Taxonomy} is the biological classification of organisms. There are
eight principal categories: Kingdom, Phylum, Class, Order, Family, Genus, and
Species. You should know how humans are classified and the defining
characteristics of each category. Tables summarizing these things are displayed
in the following pages.
\begin{figure}[H]
  \centering
  \includegraphics[scale=0.4,frame]{TaxChar.jpg}
  \caption{Taxonomic Characteristics}
\end{figure}
\begin{figure}[H]
  \centering
  \includegraphics[scale=0.45,frame]{HumTax.jpg}
  \caption{Human Taxonomy}
\end{figure}
\begin{figure}[H]
  \centering
  \includegraphics[scale=0.45,frame]{VertTax.jpg}
  \caption{Characteristics of the Vertebrate Classes}
\end{figure}



\end{document}
