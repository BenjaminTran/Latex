\documentclass[../Bio_chemistryReview.tex]{subfiles}

\begin{document}
\chapter{Microbiology}
\section{Viruses}
Viruses are \vocab[]{obligate intracellular parasites} meaning that they may
only reproduce inside of a host cell. Viruses lack the machinery necessary to
perform any of the chemical reactions necessary for life, such as ATP synthesis.
\textit{Viruses are not considered cells or living organsisms}.

\subsection{Viral Structure and Function}
\begin{wrapfigure}{r}{0.35\textwidth}
    \centering
    \vspace{-11pt} 
    \includegraphics[scale=0.15,frame]{BacteriophageT4.jpg}
    \caption{Bacteriophage T4}
    \label{fig:T4}
\end{wrapfigure}
All viruses possess a nucleic acid genome packaged in a protein shell called the
\vocab[]{capsid}. A viral genome may consist of either DNA or RNA that is either
\textit{single}- or \textit{double}-stranded and is either \textit{linear} or
\textit{circular}.  Viruses seem to utilize almost every conceivable form of
nucleic acids.  Hepatitis B for instance has a circular DNA genome that is part
single stranded and part double-stranded. A mature virus does not contain
nucleic acid other than its genome.\par

All viruses are limited by their size, which is much smaller than the cells that
they may infect. To keep their genome size down, they have few genes and rely on
the host machinery instead. They may also have more than one reading frame
within a piece of DNA so that genes may overlap with each other.\par

The capsid is used to classify viruses. It is made from a repeating patter of
only a few protein building blocks. \textit{Helical} capsids are rod-shaped,
while \textit{polyhedral} capsids are multiple-sided geometric figures with
regular faces. The T4 bacteriophage is a mix of both of these see Figure
\ref{fig:T4}.\par
Note that animal viruses do not inject their nucleic acids like bacteriophages
must because animal cells don't have a cell wall that needs to be punctured.\par

Many animal viruses possess an \vocab[]{envelope} that surrounds the capsid.
This comes from the host cells' membrane and allows the virus to fuse the
envelope with the outer membrane of another cell and infect. A virus without an
envelope such as those for plants and phages, are referred to as \vocab[]{naked
viruses}.

\subsection{Bacteriophage Life Cycles}
The virus will bind to the exterior of a bacterial cell in a process called
\vocab[]{attachment} or \vocab[]{adsorption}. The next step is injection if the
viral genome in a process termed \vocab[]{penetration} or \vocab[]{eclipse}. The
phage may then enter either the \vocab[]{lytic cycle} or the \vocab[]{lysogenic
cycle}.
\\
\subsection{The Lytic Cycle of Phages}
\begin{wrapfigure}[17]{r}{0.6\textwidth}
    \centering
    \vspace{-10pt}
    \includegraphics[scale=0.2,frame]{Lytic.jpg}
    \caption{The Lytic Cycle}
    \label{fig:lytic}
\end{wrapfigure}
This cycle is the immediate production of virus. After infection, one of the
first gene products (\vocab[]{early gene}) is \vocab[]{hydrolase}, a hydrolytic
enzyme that degrades the entire host genome. From the dNTPs of the degraded host
genome, multiple copies of the phage genome and capsid proteins are produced.
The new virus are assembled and \vocab[]{lysozyme} (\vocab[]{late gene}) is
produced to destroy the bacterial cell wall. The process is summarized in Figure
\ref{fig:lytic}.

\subsection{The Lysogenic Cycles of Phages}
\begin{wrapfigure}{r}{0.5\textwidth}
    \centering
    \vspace{-10pt}
    \includegraphics[scale=0.2,frame]{Lysogenic.jpg}
    \caption{The Lysogenic Cycle}
    \label{fig:lysogenic}
\end{wrapfigure}
This cycle is the sit-and-wait version. Upon infection, the phage genome is
incorporated into the bacterial genome and is now referred to as a
\vocab[]{prophage} and the host is now called a \vocab[]{lysogen}. Dormancy is
achieved by a phage-encoded repressor protein that binds to specific DNA
elements in phage promoters (operators). When the prophage becomes active it
removes itself from the host genome in a process called \vocab[]{excision} and
enters the lytic cycle. It is possible that a portion of the host genome is
excised with the viral genome and packaged into the progeny. When this progeny
infects other cells it will confer the traits coded for by that ``stolen''
genome which will be expressed by the infected cell. The process is summarized
graphically in Figure \ref{fig:lysogenic}.

\subsection{Replication of Animal Viruses}
Animal viruses are not referred to as phages. The general outline of the life
cycle is the same. Viruses attach to animal viruses through the receptors found
on the cell surface. Penetration into the cell is done by \vocab[]{endocytosis},
which is not done for phages because bacterial cell walls prevent bacteria from
performing endocytosis. Once inside the virus may enter one of three cycles:
\paragraph{Lytic Cycle} is the same as in phages.
\paragraph{\vocab{Productive Cycle}} is similar to the lytic cycle but does not
destroy the cell because the progeny may bud from the cell since there is no
cell wall to prevent them from doing so.
\paragraph{Lysogenic Cycle} The process is the same as in phages but the dormant
form of the viral genome is called the \vocab[]{provirus}.
\newpage
\subsection{Viral Genomes}
This subsection will present guidelines to determine what proteins a virus must
encode or actually carry in its capsid based on its genome type.

\subsubsection{[+] RNA Viruses}
---must \textit{encode} RNA-dependent RNA pol (and do not have to carry it).\\
\hfil\\
\noindent A (+) RNA virus is one which has a single-stranded RNA that serves as
mRNA. As soon as the virus infects a cell translation begins right away. In
order to replicate its RNA genome, the virus must code for RNA-dependent RNA
polymerase since no machinery of the eukaryotic host cell produces mRNA from
mRNA. Remember that the virus needs a cell membrane to bud from and have an
envelope and to replicate the mRNA, the RNA-dependent RNA pol produces the
complementary strand first, the (-) strand, as an intermediate and then the
(+) strand is produced.

\subsubsection{[-] RNA Viruses}
---must \textit{carry} RNA-dependent RNA pol (and encode it too).\\
\hfil\\
\noindent Of course from above, the RNA-dependent RNA pol is necessary to
produce the (+) strand since this type of virus carries the (-) template strand.
So the only change from the [+] RNA viruses is that the (+) strand must be made
after initial infection and the RNA-dependent RNA pol must be produced and
packaged into the progeny.

\subsubsection{Retroviruses}
---must \textit{encode} reverse transcriptase.\\
\hfil\\
\noindent These viruses undergo lysogeny. In order to integrate into the host
genome their genome must be converted into ds-DNA via \vocab[]{reverse
transcription}. Thus, they have coded an \vocab{RNA-dependent DNA polymerase} in
their genome which is produced after initial infection. Note that retroviruses
are not required to carry these with them but some do such as HIV.

\subsubsection{Double-stranded DNA Viruses}
---often \textit{encode} enzymes required for dNTP synthesis and DNA
replication.\\
\hfil \\
\noindent These viruses often have large genomes that include genes for enzymes
involved in deoxyribonucleotide synthesis (which we do whenever we make DNA) and
DNA replication. These enzymes are found within the host cell but only produce
dNTPs in preparation for replication but the virus carry it so that they can
replicate their DNA whenever they want. RNA viruses do not do this because
transcription is always happening and so NTPs (which is what RNA needs) are
always present. Some DNA viruses induce host cells to enter mitosis and can even
override cellular inhibition of cell division so as to cause cancer.

\section{Subviral Particles}
\subsection{Prions}
Prions do not strictly follow the Central Dogma because they are
self-replicating proteins and cut out the transcription and translation process.
The prion is a misfolded version of a protein that already exists. When normally
folded proteins come into contact with prions they too become misshapen. The
class of diseases caused by prions in mammals is referred to as the
\vocab[]{transmissible spongiform encephalopathies}. Prions can be genetically
linked through mutations in the genes coding for the prion protein but these are
quite rare.

\subsection{Viroids}
Viroids consist of a short piece of circular, single-stranded highly
self-complementary RNA and thus may have regions of double-strandedness.
Generally, they do not code for proteins and lack capsids. Some may produce
siRNAs when replicated and thus silence normal gene expression.\par

Replication occurs via a viroid RNA-dependent RNA polymerase synthesizing a
(-) strand which is circularized by an RNA ligase derived from the host. This is
then used as a template to make more (+) copies that match the original RNA
viroid sequence.

\section{Prokaryotes [Domain Bacteria]}
\subsection{Cell Theory}
The tenets of cell theory are discussed in chapter 2 on the cell (taken from
Kaplan). The main points are listed here again:
\begin{infobox}
  There are four tenets of cell theory:
  \begin{enumerate}
    \item All living things are composed of cells
    \item The cell is the basic functional unit of life
    \item Cells arise only from preexisting cells
    \item Cells carry genetic info in the form of DNA and is passed from parent
      to daughter cell 
  \end{enumerate}
\end{infobox}
Additional tenets have been added but don't seem to be particularly
important.\par

All living organisms can be classified as either \vocab[]{prokaryotes} or
\vocab[]{eukaryotes}. At least some members from both groups can perform
photosynthesis, the Krebs cycle, and oxidative phosphorylation to produce ATP.
The primary feature of prokaryotes is that they do not contain membrane-bound
organelles. The prokaryotes include: bacteria, archea, and blue-green algae.\par

There are three domains and within these domains there are kingdoms. The
breakdown of interest is below:
\begin{enumerate}
    \item Bacteria
    \item Archae
    \item Eukarya
        \begin{itemize}
            \item Animalia
            \item Plantae
            \item Fungi
        \end{itemize}
\end{enumerate}

\subsection{Bacterial Structure and Classification}
\subsubsection{Contents of the Cytoplasm}
The genome is a single double-stranded circular DNA chromosome. Transcription
and translation occur in the same place at the same time with the possibility of
multiple ribosomes translating a single piece of mRNA at the same time forming a
complex called a \vocab[]{polyribosome}. Recall that the bacterial ribosome has
a different structure from those of eukaryotes. This was discussed in Chapter
6.\par

Another genetic element that can be found in prokaryotic cells are
\vocab[]{plasmids}. These are small circular pieces of double-stranded DNA
referred to as \vocab[]{extrachromosomal genetic elements} and can encode
advantageous gene products. Plasmids are transferred during
\vocab[]{conjugation}.

\subsubsection{Bacterial Shape}
Bacteria are often classified by shape. There are three classifications:
\begin{description}
    \item[Cocci] - Round 
    \item[Bacilli] - Rod-shaped 
    \item[Spirochetes or spirilla] - Spiral-shaped 
\end{description}

\subsubsection{The Cell Membrane and the Cell Wall}
The cell wall provides support and prevents lysis due to osmotic pressure. It is
composed of \vocab[]{peptidoglycan} which is only unique to prokaryotes. The
cell wall is the target of many antibiotics and also of \textit{lysozyme}.

\subsubsection{Gram Staining of the Cell Wall}
\vocab[]{Gram staining} is a procedure used to determine the structure of the
cell wall.\\ 
\hfil\\
\vocab[Gram staining!]{Gram-positive} bacteria stain a dark purple
color and have a thick peptidoglycan layer outside of the cell membrane and no
other layer beyond this.\\
\vocab[Gram staining!]{Gram-negative} bacteria stain a
light pink color and have a thinner peptidoglycan layer but have an outer later
of lipopolysaccharides. The intermediate space (which includes the peptidoglycan
cell wall) between cell membrane and the outer layer is called the
\vocab[]{periplasmic space}.

\subsubsection{The Capsule}
The \vocab[]{capsule} or \vocab[capsule!]{glycocalyx} is a sticky layer of
polysaccharide ``goo'' surrounding the bacterial cell and/or colony. Allows for
them to stick to smooth surfaces and evade attack from immune system.

\subsubsection{Flagella}
\begin{wrapfigure}[10]{r}{0.4\textwidth}
    \centering
    \vspace{-12pt}
    \includegraphics[width=0.4\textwidth,frame]{ProFlag.jpg}
    \caption{\color{blue}The Prokaryotic Flagellum}
    \label{fig:proflag}
\end{wrapfigure}
\vocab[]{Flagella} serve as the only means of motility. Bacteria may be:
\begin{itemize}
    \item \textbf{Monotrichous} one flagellum located at the end
    \item \textbf{Amphitrichous} flagellum located both either end
    \item \textbf{Peritrichous} multiple flagella
\end{itemize}
The structure is composed of the \textbf{filament}, the \textbf{hook}, and the
\textbf{basal structure} (see Figure \ref{fig:proflag}). The basal structure
serves to anchor the flagellum and rotate the \textbf{rod} and the rest of the
attached flagellum.  
\textit{The most important thing to remember is that prokaryotic flagellum is
different in structure from eukaryotic ones}.
The bacteria motion is directed towards attractants in a process termed
\vocab[]{chemotaxis}. There are \vocab[]{chemoreceptors} on the cell surface
that bind attractants or repellents and transmit a signal to influence direction
of flagellar rotation. The response is dependent on the gradient of the
concentration with respect to time and not on absolute concentration.

\subsubsection{Pili}
Pili are long projections on the bacterial surface involved in attaching to
different surfaces. The \vocab[]{sex pilus} facilitates the formation of
\vocab[]{conjugation bridges}. \vocab[]{Fimbriae} are smaller structures that
are not involved in locomotion or conjugation but are involved in adhering to
surfaces.

\subsection{Bacterial Growth Requirements and Classification}
\subsubsection{Temperature}
The ability to tolerate environmental variables such as temperature is another
way to classify bacteria. Three types are:
\begin{description}
    \item[Mesophiles] favor mild temperatures such as what humans favor
    \item[Thermophiles] favor very hot temperatures such as boiling springs
    \item[Psychrophiles] favor very cold environments
\end{description}

\subsubsection{Nutrition}
Bacteria can be classified according to their \textit{carbon source} or
\textit{energy source}. There are four base types:
\begin{description}
    \item[Autotrophs] utilize \chem{CO_{2}}
    \item[Heterotrophs]  rely on organic nutrients created by other organisms
    \item[Chemotrophs] rely on chemicals
    \item[Phototrophs] rely on light
\end{description}
Each bacterium is either a chemo- or phototroph and either an auto- or
heterotroph. So there are four kinds of bacteria:
\begin{description}
    \item[Chemoautotrophs] build organic macromolecules from \chem{CO_{2}} using
        energy from chemicals e.g. oxidizing molecules like \chem{H_{2}S}.
    \item[Chemoheterotrophs] require organic molecules such as glucose made by
        other organisms as their carbon source and for energy. (We are this).
    \item[Photoautotrophs] use only \chem{CO_{2}} as a carbon source and obtain
        energy from the sun
    \item[Photoheterotrophs] get their energy from the Sun, but require an
        organic molecule made by another organism as their carbon source
\end{description}

\subsubsection{Oxygen Utilization and Tolerance}
The different types are:
\begin{description}
    \item[Obligate aerobes] require oxygen
    \item[Obligate anaerobes] are poisoned by oxygen
    \item[faculative anaerobes] will use oxygen when it's around, but don't need
        it
    \item[Tolerant anaerobes] can grow with or without oxygen but do not use it.
\end{description}

\subsubsection{Fermentation vs. Respiration}
Covered in Chapter 5 (Chapter 4 of the book). Respiration is glucose catabolism
with use of an inorganic electron acceptor such as oxygen. Fermentation is
glucose catabolism without an electron acceptor. 

\subsubsection{Anaerobic Respiration}
This refers to glucose metabolism with electron transport and oxidative
phosphorylation relying on an external electron acceptor other than
\chem{O_{2}}.

\subsection{Bacterial Life Cycle}
Bacteria reproduce asexually through \vocab[]{binary fission} and thus do not
increase the genetic diversity in so doing. But they can perform conjugation to
exchange genetic material. There are four stages of growth of a colony:
\begin{description}
    \item[\vocab{Lag phase}] when bacteria in ideal media are not growing
        immediately because they need to produce the necessary components for
        cell division first
    \item[\vocab{Log phase}]  when bacteria begin to divide yielding
        exponential growth
    \item[\vocab{Stationary phase}] when resources are used up and bacteria no
        longer divide
    \item[\vocab{Death phase}] when bacteria die off due to lack of resources
\end{description}
The maximum population at the stationary phase is referred to as the
\vocab[]{carrying capacity}.

\subsection{Endospore Formation}
When conditions are unfavorable for growth then some types of Gram-positive
bacteria may form \vocab[]{endospores} capable of withstanding temperatures
above 100C. Metabolic reactivation of an endospore is termed
\vocab[]{germination}.

\subsection{Genetic Exchange Between Bacteria}
Three methods of acquiring new genetic material:
\begin{description}
    \item[\vocab{Transduction}] Cf. Chapter 6. When DNA is introduced during
        the lysogenic cycle.
    \item[\vocab{Transformation}] Bacteria internalizing and incorporating DNA
        found in the environment under certain conditions.
    \item[\vocab{Conjugation}] is the normal bacterial function of obtaining
        more genetic information. It is discussed below.
\end{description}

\subsection{Conjugation}
\begin{wrapfigure}[21]{r}{0.6\textwidth}
    \centering
    \vspace{-11pt}
    \includegraphics[width=0.5\textwidth,frame]{Conjugation.jpg}
    \vspace{-5pt}
    \caption{Conjugation}
    \label{fig:conj}
\end{wrapfigure}
\vocab{Conjugation} (see Figure \ref{fig:conj}) occurs through sex pili forming a
\vocab{conjugation bridge} where a male containing the \vocab[]{F (fertility)
factor} (F\textsuperscript{+}) uses it's sex pili to penetrate a female cell,
which does not contain the F factor and is therefore considered
F\textsuperscript{--}, and exchange genetic information, including the F factor,
unidirectionally to the female. This process can only occur between male
(F\textsuperscript{+}) and female (F\textsuperscript{--}).\par


The F factor can be integrated into the bacterial chromosomes through
recombination. A cell with the F factor integrated into its genome is called an
\vocab[Conjugation!]{Hfr} (high frequency of recombination) cell. When an Hfr
cell undergoes conjugation, the replication of the F factor can also replicate
the DNA around it and that will be transferred to the female bacteria.

\subsection{Domain Archaea}
Archaea bacteria are generally found in extreme environments but they can also
be found in normal settings. Their cell walls lack peptidoglycan and genetically
they share traits with eukaryotes including the presence of introns and the use
of many similar mRNA sequences. They also reproduce through fission.

\subsection{Parasitic Bacteria}
Parasitic bacteria may be \textit{obligate} meaning that they must be inside a
host cell to replicate or they may be \textit{facultative} meaning that they can
live inside or outside of the cell. This is the same framework as in reference
to aerobes and anaerobes.

\subsection{Symbiotic Bacteria}
These bacteria coexist with a host where both derive a benefit. Since these
bacteria have a dependent relation with their host, they have a smaller genome
with a more limited number of cellular products. Thus, they do not survive long
outside of the host cell.

\section{Fungi}
\subsection{Fungal Structure}
Fungi possess all of the features that distinguish eukaryotes from prokaryotes.
All fungi possess a rigid cell wall composed of \vocab[]{chitin}. All fungi are
chemoheterotrophs. Most fungi are obligate aerobes, although yeast are
facultative anaerobes. Digestion of nutrients takes place outside the fungal
cell. This method is termed \vocab[]{absorptive}.\par

After the cell, the next level of structure is the \vocab[]{hypha}, which is a
long filament of cells joined end-to-end. In \vocab[]{septae hyphae}, the cells
are separated by walls called septae. \vocab[]{Aseptae hyphae} are composed of
cells joined together in a long tube i.e. these fungi are multinucleate. Hyphae
specialized to digest and absorb nutrients in a parasitic fashion are called
\vocab[]{haustoria}.\par

A meshwork of hyphae is called a \vocab[]{mycelium}. A large fungal structure
visible to the naked eye is called a \vocab[]{thallus}. The
\vocab[]{vegetative} portion of the thallus is involved in obtaining nutrients,
and the \vocab[]{fruiting body} functions in reproduction.

\subsection{Fungal Life Cycles}
\subsubsection{Asexual Reproduction}
Fungi reproduce both sexually and asexually. Asexual reproduction occurs through
\begin{description}
    \item[Budding] a new smaller hypha grows outward from an existing one
    \item[Fragmentation] mycelium is broken into smaller pieces and each piece
        forms a separate mycelium
    \item[Asexual spore formation] occurs through mitosis to generate many
        spores from one cell
\end{description}

\subsubsection{Fungal Sexual Reproduction}
A cell or species with only one copy of each chromosome is \vocab[]{haploid} and
those with two copies are \vocab[]{diploid}. In humans, \vocab[]{gametes} are
those reproductive cells that are haploid e.g. sperm. Fusion of two gametes form
a diploid \vocab[]{zygote}.\par

Contrary to humans, fungi are haploid adults rather than diploid. Initially,
they form from the fusion of haploid cells derived from haploid adults to make
a diploid zygote, which undergoes meiosis to produce haploid cells again. This
process continues.

\end{document}
