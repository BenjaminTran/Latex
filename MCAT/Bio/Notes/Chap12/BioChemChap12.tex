\documentclass[../Bio_chemistryReview.tex]{subfiles}

\begin{document}
\chapter{The Excretory and Digestive Systems}
\section{The Excretory System Overview}
A quick overview of the main excretory organs:
\paragraph{Liver}
Deals with hydrophobic or large waste products, which cannot be filtered out by
the kidney. The liver breaks down old heme units into \vocab[]{bilirubin} and
tags them with glucuronate for excretion with bile. It also synthesizes and
excretes urea.

\paragraph{Colon}
The \vocab[]{large intestine} reabsorbs water and ions from feces. It may also
excrete excess ions into the feces.

\paragraph{Skin}
Skin produces sweat, which contains water, ions, and urea, but this process is regulated by temperature.

\paragraph{Kidneys}
Excretion of hydrophilic wastes. The kidneys are complex and serve homeostatic
roles as well. The roles of the kidney are
\begin{enumerate}
  \item Excretion of hydrophilic wastes   
  \item Maintenance of constant solute concentration and constant pH
  \item Maintenance of constant fluid volume (important for blood pressure and
    cardiac output)
\end{enumerate}
These goals are accomplished via three processes:
\begin{enumerate}
  \item \vocab[]{filtration} Pressurized blood over a filter causes water and
    waste but also some glucose, to be squeezed into the \vocab[]{renal tubule}
    while proteins and cells remain in the blood. The fluid in the renal tubule
    is called \vocab[]{filtrate} and will eventually be made into urine.
  \item \vocab[]{Selective reabsorption} which is the take back of useful items
    (glucose, water, amino acids), while leaving wastes and some water in the
    tubule.
  \item \vocab[]{Secretion} involves the addition of substances to the filtrate.
\end{enumerate}
The last step is concentration and dilution of the urine by the selective
reabsorption of water.
\newpage
\section{Anatomy and Function of the Urinary System}
\begin{wrapfigure}{r}{0.5\textwidth}
  \centering
  \includegraphics[width=0.5\textwidth,frame]{kidney.jpg}
  \caption{Internal Anatomy of the Kidney}
  \label{fig:kid}
  \vspace{-16pt}
\end{wrapfigure}
Blood enters the kidney through the \vocab[]{renal artery} and reenters the
circulatory system through the \vocab[]{renal vein} which empties into the
inferior vena cava. Figure \ref{fig:kid} shows the anatomy of the kidney. Urine
collects in the \vocab[]{collecting ducts} of the \vocab[]{medullary pyramids}.
Urine empties from the papilla into the \vocab[]{calyx } which converges to form
the \vocab[]{renal pelvis} which empties into the ureter.

\subsection{Simplified Microscopic Anatomy and Function}
The functional unit of the kidney is the \vocab[]{nephron}. It consists of the
\vocab[]{capsule}, a rounded region surrounding the capillaries where filtration
takes place, and the \vocab[]{renal tubule} which empties into a collecting
duct. Many blood vessels surround the nephron to carry blood to them and
filtered blood from them along with reabsorbed substances. A more in-depth look
is warranted. Refer to figure \ref{fig:nep} for the following discussion.
\begin{figure}[H]
  \centering
  \includegraphics[scale=0.3,frame]{nephron.jpg}
  \caption{Simplified View of the Renal Tubule Plus Blood Vessels}
  \label{fig:nep}
\end{figure}


\subsection{Filtration}
Blood from the renal artery (note that artery refers to blood flowing away from
the heart by definition and cannot be used to mean blood flowing away from
anything in general) enters the \vocab[]{afferent arteriole} which branches out
into the \vocab[]{glomerulus}. From where the blood flows into an
\vocab[]{efferent arteriole} where constriction of it increases the pressure and
causes fluids to pass out of the plasma and through the \vocab[]{glomerular
basement membrane} and enters the \vocab[]{Bowman's capsule}.

\subsection{Selective Reabsorption}
Substances are selectively reabsorbed by the \vocab[]{peritubular capillaries}.
Most of the reabsorption occurs at the \vocab[]{proximal convoluted tubule}
(PCT) and is not regulated. It just absorbs as much as it can. In contrast to
the \vocab[]{distal convoluted tubule} (DCT) which is regulated. 

\subsection{Secretion}
This is just the active transport of unnecessary substances and waste into the
filtrate via active transport. Occurs all along the tubule.

\subsection{Concentration and Dilution}
Urine must have appropriate volume and osmolarity. Adjustments to the filtrate
occur at the \vocab[]{distal nephron} which includes the \vocab[]{DCT} and the
\vocab[]{collecting duct}. It is controlled by \vocab[]{ADH} and
\vocab[]{aldosterone}.
\begin{description}
  \item[ADH] from the posterior pituitary gland affects the permeability of the
    distal nephron to water. With it the membrane becomes permeable to water. It
    is released whenever water needs to be retained.
  \item[Aldosterone] from the adrenal cortex is released when blood pressure is
    low. It causes increased reabsorption of \chem{Na^{+}} by the distal
    nephron leading to increased plasma osmolarity and thus increased thirst and
    water retention which raises blood pressure.
\end{description}
So both work in conjunction to increase blood pressure.

\subsection{The Loop of Henle}
Figure \ref{fig:nep} shows a simplified view of the true anatomy. In reality,
after the glomerulus and the PCT (located in the renal cortex) the tubule flows
into the \vocab[]{Loop of Henle}. The \vocab[Loop of Henle!]{descending limb}
is the portion that enters the renal medulla and is thin walled. The
\vocab[Loop of Henle!]{ascending limb} is the portion that reenters the cortex.
It is partially thick walled and thin walled. The Loop then continues to the
DCT.\par

The loop is a countercurrent multiplier which is to say that it expends energy
to create a concentration gradient. This is summarized in the following
statement: \textit{The loop of Henle is a countercurrent multiplier that makes
the medulla very salty, and that this facilitates water reabsorption from the
collecting duct. This is how the kidney is capable of making urine with a much
higher osmolarity than plasma}. 

\subsection{The Vasa Recta are Countercurrent Exchangers}
The ascending portions of the vasa recta are next to the descending portion of
the loop of Henle and so serve to reabsorb the water from the descending loop.
It runs counter to the current of the descending limp and thus it is a
countercurrent exchanger, an efficient process. It also maintains the high
concentration of salt in the medulla.

\section{Renal Regulation of Blood Pressure and PH}
The \vocab[]{glomerular filtration rate} depends directly on pressure so the
kidney can locally regulate it. The \vocab[]{juxtaglomerular apparatus
} (JGA) forms a contact point between the afferent arteriole and the distal
tubule. Cells in the afferent areteriole are called \vocab[]{juxtaglomerular
cells} (JG) and those in the distal tubule are known as \vocab[]{macula densa}. A
decrease in blood pressure causes JG cells to release \vocab[]{renin} which
ultimately yields \vocab[]{angiotensin II} by \vocab[]{angiotensin-converting
enzyme} (ACE). This increases the blood pressure and stimulates the release of
aldosterone. 

\subsection{Renal Regulation of pH}
Kidney is important for maintenance of constant blood pH by excreting
bicarbonate or hydrogen ion. These adjustments are slow and often require
several days to return pH back to normal after a disturbance.

\section{Endocrine Role of the Kidney}
The kidneys play a role in hormones. The relevant hormones are the following:
\begin{description}
    \item[Aldosterone] sodium reabsorption and potassium secretion to increase
        blood pressure
    \item[ADH] Increases water reabsorption and secreted when plasma volume is
        too low, blood pressure is too low, or plasma osmolarity is too high
    \item[Calcitonin] Decreases \chem{Ca^{2+}} levels in serum
    \item[Parathyroid hormone] Opposite function of calcitonin
    \item[EPO] Increase synthesis of red blood cells
\end{description}

\section{The Digestive System--An Overview}
\subsection{GI Epithelium}
\begin{wrapfigure}{r}{0.5\textwidth}
    \centering
    \includegraphics[width=0.5\textwidth,frame]{Epit.jpg}
    \caption{Epithelial Cells}
    \label{fig:epi}
\end{wrapfigure}
The GI lumen has a lining of epithelial cells attached to a \vocab[]{basement
membrane}. The surface facing the lumen is the \vocab[]{apical surface} and the
surface facing the basement is the \vocab[]{basolateral surface}. The cells are
separated by \vocab[]{tight junctions} which run around the sides of epithelial
cells. See Figure \ref{fig:epi}.

\subsection{GI Smooth Muscle}
GI muscle is smooth muscle. \vocab[]{GI motility} refers to the rhythmic
contraction of GI smooth muscle and is affected by hormonal input. Like cardiac
muscle, GI smooth muscle exhibits automaticity and is a \vocab[]{functional
syncytium}. The GI tract also contains its own massive nervous system called the
\vocab[]{enteric nervous system}. The parasympathetic nervous system stimulates
motility and causes sphincters to relax while the sympathetic stimulation does
the opposite. Movement of food down the GI tract is accomplished by an orderly
form of contraction known as \vocab[]{peristalsis}.

\subsection{Enteric Nervous System}
The enteric nervous system is a branch of the autonomic nervous system. It can
operate independently of the other two branches but can also be affected by them
as well. The enteric is made of two networks of neurons:
\begin{enumerate}
    \item \vocab[]{myenteric plexus} regulates gut motility
    \item \vocab[]{submucosal plexus} regulate enzyme secretion, gut blood flow,
        and ion/water balance in the lumen
\end{enumerate}

\subsection{GI Secretions}
Secretions may be either \vocab{endocrine} or \vocab{exocrine}. The difference
is that exocrine secretions use a duct whereas endocrine glands are ductless. 

\section{The Gastrointestinal Tract}
\subsection{Mouth}
Not much to say here. The mouth secretes \vocab[]{salivary amylase} (ptyalin).

\subsection{Pharynx and Esophagus}
The \vocab[]{pharynx} is the throat and opens to the \vocab[]{trachea} and the
\vocab[]{esophagus}.The esophagus goes to the stomach and the trachea to the
lungs.

\subsection{Stomach}
The following is just a list of facts and definitions about the stomach.
\begin{enumerate}
    \item Gastric pH is about 2
    \item \vocab{Pepsin} catalyzes proteolysis and is secreted by \vocab{chief
        cells}
    \item \vocab[]{Chyme} is food mixed with gastric secretions.
    \item The \vocab[]{pyloric sphincter} prevents the passage of food from the
        stomach into the duodenum. The hormone responsible for regulating this
        is the \vocab[]{cholecystokinin}.
\end{enumerate}

\subsection{Small Intestine}
The small intestine is separated into three segments: the \vocab[]{duodenum},
\vocab[]{jejunum}, and \vocab[]{ileum}.

\subsection{Bile and Pancreatic Secretions in the Duodenum}
Two ducts empty into the duodenum: the \vocab[]{common bile duct} and the
\vocab[]{pancreatic duct}. Both ducts go through the \vocab[]{sphincter of
Oddi}.

\subsection{Duodenal Enzymes}
Duodenal \vocab[]{enterokinase} activates the pancreatic zymogen
\vocab[]{trypsinogen} to trypsin.

\subsection{Duodenal Hormones}
There are three main duodenal hormones
\begin{enumerate}
    \item Cholecystokinin (CCK) is secreted in response to fats.
    \item Secretin is released in response to acid in the duodenum
    \item Enterogastrone decreases stomach emptying
\end{enumerate}

\subsection{Jejunum and Ileum}
Substances not absorbed in the duodenum must be absorbed in these lower segments
of the intestine. Vitamin B\textsubscript{12} is absorbed only in the ileum.

\subsection{Colon [Large intestine]}
The purpose is to reabsorb water and minerals. The first portion of the colon is
the \vocab[]{cecum}. It is separated from the ileum by the \vocab[]{ileocecal
valve} which also blocks chyme from entering the colon. Defecation is controlled
by the \vocab[]{anal sphincter} which has an internal smooth muscle portion and
an external voluntary portion. Note that this is the same setup as the urinary
sphincters. \vocab[]{Colonic bacteria} in the colon out compete any possible
deadly bacteria and also produce \vocab[]{vitamin K} which we absorb from them.

\section{The GI Accessory organs}
\subsection{Exocrine Pancreas}
Some of the enzymes produced are
\begin{description}
    \item \vocab[]{Pancreatic amylase} hydrolyzes polysaccharides to
        disaccharides
    \item \vocab[]{Pancreatic lipase} hydrolyzes triglycerides at the surface of
        a micelle
    \item \vocab[]{Nucleases} hydrolyze DNA and RNA
    \item There are several different types of \vocab[]{pancreatic proteases}
\end{description}

\subsection{Endocrine Pancreas}
Consists of small regions known as \vocab[]{islets of Langerhans}. Three types
of cells each secreting a particular hormone
\begin{enumerate}
    \item $ \alpha $ cells secrete \vocab[]{glucagon} in response to low blood
        sugar. Tells liver to hydrolyze more glycogen to release glucose into
        the blood.
    \item $ \beta $ cells secrete \vocab[]{insulin} in response to elevated
        blood sugar. Initiates storage of glucose. 
    \item $ \delta $ cells secrete \vocab[]{somatostatin}. It inhibits many
        digestive processes.
\end{enumerate}

\subsection{Blood Glucose}
To lower blood glucose one releases insulin and to raise it one can release
glucagon, epinephrine, or cortisol.

\subsection{Hormonal Control of Appetite}
When the stomach is empty, gastric cells produce the hormone \vocab[]{ghrelin}
to stimulate appetite. When the colon is full, the jejunum produces
\vocab[]{peptide YY} to reduce appetite. The hormone \vocab[]{leptin} produced
by fat tissue is an appetite suppressant.

\section{A Day in the Life of Food}
There are three main types of dietary nutrients: carbohydrates, proteins, and
fats.

\subsection{Carbohydrates}


\end{document}

