\documentclass[../Bio_chemistryReview.tex]{subfiles}
\definecolor{fu-blue}{RGB}{0,51,102}
\newcommand{\shaderow}{\rowcolor{fu-blue!10}}
\begin{document}
\chapter{Eukaryotic Cells}
\section{The Organelles}
A table of the organelles and their functions is given below in Table
\ref{tab:org}.  
\begin{table}[htp]
  \caption{Animal Cell Organelles}
  \centering
  \rowcolors{1}{fu-blue!10}{white}
  \setlength{\aboverulesep}{0pt}
  \setlength{\belowrulesep}{0pt}
  \begin{tabularx}{\textwidth}{lXc}
    \toprule
    \rowcolor{fu-blue!20}\textcolor{Red3}{\textbf{Organelle}} & \textcolor{Red3}{\textbf{Function}} &
    \textcolor{Red3}{\textbf{\# membranes}}\\
    \midrule
    nucleus  & Contain \& protect DNA, transcription, partial assembly of ribosomes  & 2 \\
    mitochondria  & Produce ATP via the Krebs cycle and oxidative phosphorylation & 2 \\
    ribosomes & Synthesize proteins & 0 \\
    RER & Location of synthesis/modification of secretory, membrane-bound, \& organelle proteins & 1 \\
    SER & Detoxification \& glycogen breakdown in liver; steroid synthesis in gonads & 1 \\
    Golgi apparatus & Modification \& sorting of protein, some synthesis & 1 \\
    lysosomes & Contain acid hydrolases which digest various substances & 1 \\
    peroxisomes & Metabolize lipids \& toxins using \chem{H_{2}O_{2}} & 1\\
    \bottomrule
  \end{tabularx}
  \label{tab:org}
\end{table}

\subsection{The Nucleus}
The nucleus contains the genome in a compartment surrounded by the
\vocab[]{nuclear envelope}, which is composed of two lipid bilayer membranes.
The membrane of the endoplasmic reticulum is at points continuous with the outer
nuclear membrane, making the interior of the ER (\vocab[]{lumen}) contiguous
with the space between the two nuclear membranes.\par

\vocab[]{Nuclear pores} allow the passage of material into and out of the
nucleus. Molecules larger than 60kDa must contain a sequence of amino acids
called a \vocab[]{nuclear localization sequence}.

\subsection{The Genome}
The genome is a linear molecule of ds-DNA. The large size of the genome requires
it to be split into pieces called \vocab{chromosomes}. Chromosomes have a
\vocab[]{centromere} near the middle to ensure that newly replicated chromosomes
are sorted properly during cell division, one copy to each daughter cell. The
ends of the chromosomes also have regions called \vocab{telomeres}.\par

Genes can be mapped genetically and physically to the chromosome they reside on.
The specific location on the chromosome is termed the \vocab[]{locus}. The locus
of a gene may affect its expression. Regions of densely packed chromatin are
called \vocab[]{heterochromatin} within which genes are inaccessible and turned
off. Loosely packed regions are called \vocab[]{euchromatin} and have active
genes.\par

The nucleus is supported by a mesh of proteins called the \vocab[]{nuclear
matrix} or \vocab[]{nuclear scaffold}. The matrix may also play a role in
regulating gene expression.

\subsection{The Nucleolus}
The \vocab[]{nucleolus} is a region within the nucleus which produces ribosomes.
The nucleolus is the site of transcription of rRNA by RNA pol I. The protein
components of the ribosome are transported into the nucleus from the cytoplasm
and the ribosome is partially assembled. Complete assembly is accomplished in
the cytoplasm.

\subsection{Mitochondria}
\vocab[]{Mitochondria} are the site of oxidative phosphorylation. The interior
of the mitochondria, the \vocab[]{matrix}, is bounded by the inner and outer
mitochondrial membranes. The matrix contains pyruvate dehydrogenase and the
enzymes of the Krebs cycle. The inner membrane is the site of the electron
transport train and is folded into \vocab[]{cristae}. The protons are pumped
into the matrix. There is a lengthy discussion about mitochondria's individual
genome and inheritance.

\subsection{Endoplasmic Reticulum [ER]}
The \vocab[]{endoplasmic reticulum} (ER) is a large system of folded membrane.
There are two types:
\begin{description}
  \item \vocab[endoplasmic reticulum!]{Rough ER} has a large number of ribosomes
    bound to its surface. It is the site of protein synthesis for secreted
    proteins.
  \item \vocab[endoplasmic reticulum!]{Smooth ER} is not actively involved in
    protein processing but can contain enzymes involved in steroid hormone
    biosynthesis (gonads) or in the degradation of environmental toxins (liver).
\end{description}

\subsection{The Rough ER and the Secretory Pathway}
Proteins translated on cytoplasmic ribosomes are headed toward peroxisomes,
mitochondria, the nucleus, or will remain in the cytoplasm. Proteins translated
on the RER will end up either:
\begin{enumerate}
  \item secreted into the extracellular environment
  \item as integral plasma membrane proteins
  \item in the membrane or interior of the ER, Golgi apparatus, or lysosomes.
\end{enumerate}

Whether a protein is translated on the RER is determined by the sequence of the
protein itself. Some proteins have a sequence called a \vocab[]{signal sequence}
at their N-terminus. The sequence is recognized by the \vocab[]{signal
recognition particle} (SRP) which binds to the ribosome. The RER has SRP
receptors and will pick up the complex. The protein is then translated.\par

\vocab[]{Integral membrane proteins} have sections of hydrophobic amino acid
residues called \vocab[]{transmembrane domains} that pass through lipid bilayer
membranes. These domains can be weaved into the ER membrane after
translation.\par

Proteins that are in the secretory pathway need \vocab[]{targeting signals} to
end up elsewhere e.g. the Golgi, ER, lysosome. Proteins that are made in the
cytoplasm that need to be sent to an organelle such as the nucleus need
\vocab[]{localization signals}. A table that summarizes this information is
given below see Table \ref{tab:sumCell}
\begin{table}[H]
  \caption{Summary of Cellular Protein Traffic}
  \centering
  \setlength{\aboverulesep}{0pt}
  \setlength{\belowrulesep}{0pt}
  \rowcolors{1}{fu-blue!10}{white}
  \begin{tabularx}{\textwidth}{p{5.5em}p{5em}p{6em}p{8em}Xp{9em}}
    \toprule
    \rowcolor{fu-blue!20}\textbf{Protein Final\newline Destination} & \textbf{Signal\newline Sequence?} & \textbf{Localization\newline Signal?} & \textbf{Transmembrane\newline Domains?} & \textbf{Targeting\newline Signal?} & \textbf{Example} \\
    \midrule
    Secreted & Yes & No & No & No & Antibodies,\newline Neurotransmitters,\newline Peptide hormones \\
    Plasma Membrane & Yes & No  & Yes & No  & Receptors, channels\\
    Lysosome        & Yes & No  & No  & Yes & Acid hydrolases\\
    Rough ER        & Yes & No  & No  & Yes & Enzymes required for protein modification\\
    Smooth ER       & Yes & No  & No  & Yes & Enzymes required for lipid synthesis\\
    Golgi\newline Apparatus & Yes & No  & No  & Yes & Enzymes required for protein modification\\
    Cytoplasm       & No  & No  & No  & No  & Glycolysis enzymes\\
    Nucleus         & No  & Yes & No  & No  & Histones, DNA|RNA polymerase\\
    Mitochondria    & No  & Yes & No  & No  & PDC/Krebs cycle enzymes\\
    Peroxisome      & No  & Yes & No  & No  & Catalase\\
    \bottomrule 
  \end{tabularx}
  \label{tab:sumCell}
\end{table}

\subsection{The Golgi Apparatus}
The Golgi apparatus has the following functions
\begin{enumerate}
  \item Modification of proteins made in the RER (esp. oligosaccharide chains)
  \item Sorting and sending proteins to their correct destinations
  \item Synthesis of certain macromolecules
\end{enumerate}
The portion of the Golgi nearest the RER is called the \textit{cis} stack and
the part farthest is called the \textit{trans} stack. The \textit{medial} stack
is of course in the middle. Secreted proteins are modified in the
\textit{cis} and \textit{medial} stacks. The route taken by a protein from the
\textit{trans} stack is determined by signals within the protein that determine
which vesicle a protein is sorted into.\par

The vesicles are released from a cell through exocytosis. Any membrane bound
organelles become a part of the cell membrane and any secretory proteins are
secreted. Proteins that are sent in vesicles from the Golgi immediately to the
cell surface follow the \vocab[]{consitutive secretory pathway}. Proteins may be
stored and await for signals to then be secreted. These are said to follow the
\vocab[]{regulated secretory pathway}.

\subsection{Lysosomes}
The \vocab[]{lysosome} is a membrane-bound organelle that is responsible for the
degradation of biological macromolecules by hydrolysis. They degrade particulate
matter through \vocab[]{phagocytosis}. \vocab[]{Crinophagy} refers to lysosomal
digestion of unneeded excess secretory products. The enzymes of the lysosomes,
\vocab[]{acid hydrolases}, only function properly in an acidic environment so
that if the lysosome ruptures the enzymes are inactive in the pH 7.4 of the
cytoplasm.

\subsection{Peroxisomes}
\vocab[]{Peroxisomes} are essential for lipid breakdown and in the liver they
assist in detoxification of drugs and cehmicals. They produce
\chem{H_{2}O_{2}} as a by product but also have an enzyme called
\vocab[]{catalase} which converts it to water and oxygen.

\section{The Plasma Membrane}
\subsection{Membrane Structure}
All membranes of the cell are composed of lipid bilayers. The three most common
lipids in these membranes are \vocab[]{phospholipids}, \vocab[]{glycolipids},
and \vocab[]{cholesterol}. These are empathic molecules thus naturally, the
hydrophobic portions will face each other and the hydrophilic portions will face
away from. This creates the lipid bilayer.\par

There are several types of membrane bound proteins. Some mediate interactions
with other cells, others called \vocab[]{cell-surface receptors} bind
extracellular signaling molecules such as hormones and relay these signals into
the cell. \vocab[]{Channel proteins} selectively allow ions or molecules to
cross the membrane.\par

In general, there are two types of membrane proteins. \vocab[]{Integral membrane
proteins} are embedded into the membrane and held there by hydrophobic
interactions. The membrane crossing regions are the \vocab[]{transmembrane
domains} discussed earlier. \vocab[]{Peripheral membrane proteins} are not bound
to the membrane but are hydrogen bonded to integral membrane proteins. They are
free to move around the membrane surface from site to site.\par

The \vocab[]{fluid mosaic model} seeks to describe the cell membrane. Saturated
fatty acids have a very straight structure and pack tightly with strong van der
Waals forces between side chains and thus decrease membrane fluidity. The
opposite is truce of unsaturated fatty acids. Cholesterol, packed in the
intermembrane space, also plays a role in membrane fluidity.

\section{Transmembrane Transport}
Recall concentration measurements that are discussed in the first chapter of the
general chemistry review; molarity, molality (mass variant of molarity and good
for temperature invariant measures), and mole fraction.

\subsection{Electrolytes}
\vocab[]{Electrolytes} are just free ions in solution.
\vocab[Electrolytes!]{Strong electrolytes} are solutes that dissociate completely
and \vocab[Electrolytes!]{weak electrolytes} only partially dissociate. The
number of ions a substance will produce in a solution is called the
\vocab[]{van't Hoff} factor and is typically denoted by i.

\subsection{Colligative Properties}
\vocab[]{Colligative properties} depend on the \textit{number} of solute
particles in the solution rather than the \textit{type} of particle.
Furthermore, \textit{the identity of the particle is unimportant}. One must
therefore consider the van't Hoff factor. There are four colligative properties
of interest discussed below.

\subsubsection{Vapor-Pressure Depression}
The more solute you add to a solvent, the more the solvent molecules attach to
the solute particles. This increases the energy required for the solvent
molecules to break free of the attractive forces in the liquid and hence
evaporate. Thus, \textit{more solute means higher boiling point}.

\subsubsection{Boiling-Point Elevation}
As discussed in the previous section, the boiling point increases with more
solute and this is given by the following equation:
\begin{align}
  \boxed{\Delta T_{b} = k_{b}im} \\ \eqname{Boiling-Point Elevation}
\end{align}
where $ k_{b} $ is the solvent's boiling-point elevation constant. For water, $
k_{b} \approx 0.5^{\circ}C/m $.


\subsubsection{Freezing-Point Depression}
Physically, adding a solute to the solvent will interfere with the arrangement
of the solvent into a crystal lattice at the solid phase transition. Thus, the
freezing-point is decreased with more solute added. The change in the
freezing-point is given by:
\begin{align}
  \boxed{\Delta T_{f} = -k_{f}im} \\ \eqname{Freezing-Point Depression}
\end{align}

\subsection{Review of Diffusion and Osmosis}
\vocab[]{Diffusion} is the tendency for liquids and gases to fully occupy the
available volume. A solute will diffuse \textit{down its concentration
gradient}. \vocab[]{Osmosis} is a type of diffusion where the solvent diffuses
rather than the solute. This occurs if there is a \vocab[]{semipermeable
membrane} which allows for one type of particle such as water to cross but not
another such as sucrose. So:
\begin{description}
  \item[Diffusion] membrane permeable to both solvent and solute \imp solute
    moves toward equilibrium
  \item[Osmosis] membrane permeable to solvent but not solute \imp solvent moves
    toward equilibrium
\end{description}
In either case the final result is that solute concentrations are the same on
both sides of the membrane.\par

\vocab[]{Tonicity} is used to describe osmotic gradients with reference to the
amount of dissolved solute. Thus, we have the following definitions
\begin{description}
  \item \vocab[Tonicity!]{Hypertonic} solutions have more total dissolved
    solutes than the cell
  \item \vocab[Tonicity!]{Hypotonic} solutions have less total dissolved
    solutes than the cell
  \item \vocab[Tonicity!]{Isotonic} solutions have the same amount of dissolved
    solutes
\end{description} 

\subsection{Osmotic Pressure}
\vocab[]{Osmotic pressure} can be defined as the pressure it would take to stop
osmosis from occurring and is given by the \vocab[]{van't Hoff equation}
\begin{align}
  \boxed{\Pi = MiRT} \\ \eqname{van't Hoff Equation}
\end{align}
where $ \Pi $ is measured in atm and M is the molarity of the solution.

\subsection{Passive Transport}
\vocab[]{Passive transport} is \textit{any thermodynamically favorable movement
of solute across a membrane}. There are two types of passive transport discussed
below.

\subsubsection{Simple Diffusion}
\vocab[Passive transport!]{Simple diffusion} is diffusion of a solute through a
membrane without help from a protein e.g. steroids. \vocab[Passive
transport!]{Facilitated diffusion} is diffusion of a solute through a membrane
that is intrinsically impermeable to that solute with the help of a protein.

\subsubsection{Facilitated Diffusion: Channels}
There are many kinds of \vocab[Passive transport!Facilitated diffusion!]{ion
channels}. They may be \textbf{gated} which is to say that the channel is open
in response to specific environmental stimuli. \vocab[Passive
transport!Facilitated diffusion!ion channels!]{Voltage-gated} ion channels open
in response to a change in the electrical potential across the membrane. One
that opens in response to the binding of a specific molecule is called a
\vocab[Passive transport!Facilitated diffusion!ion channels!]{ligand-gated}
ion channel.

\subsubsection{Facilitated Diffusion: Carriers}
\vocab[Passive transport!Facilitated diffusion!]{Carrier proteins} rely on
conformational change to move molecules across membranes. Carriers that only do
one at a time are called 
\vocab[Passive transport!Facilitated diffusion!Carrier
proteins!]{uniports}, \vocab[Passive transport!Facilitated diffusion!Carrier
proteins!]{Symports} carry two different types of particles across the membrane
in the same direction. \vocab[Passive transport!Facilitated diffusion!Carrier
proteins!]{Antiports} do so in the opposing directions.

\subsubsection{Pores and Porins}
A \vocab[]{pore} is large enough that it is \textit{not selective}. Pores are
formed by polypeptides known as \vocab[]{porins}. Eukaryotic membranes do not
have pores, because it would destroy barrier function of the membrane. Lastly,
you would not expect pores and ion channels to be found in the same membrane as
a pore would make the ion channel useless.

\subsubsection{Kinetic Concerns}
The kinetics of simple diffusion is limited to the surface area of the membrane
and the size of the gradient. Facilitated diffusion, however, is dependent on a
finite number of integral membranes so saturation of these proteins sets the
limit.

\subsection{Active Transport}
\vocab[]{Active Transport} is the movement of molecules through the plasma
membrane against a gradient and therefore requires energy input and proteins.
In \vocab[Active Transport!]{primary active transport}, the transport of a
molecule is coupled to ATP hydrolysis. In \vocab[Active Transport!]{secondary
active transport}, the transport process is indirectly coupled to ATP
hydrolysis. For instance, ATP hydrolysis may create the gradient that drives the
transport. This is found in the sodium, glucose symport of the intestines where
ATPase creates the sodium gradient used to intake the glucose. Note that in this
type of transport, the movement of a molecule down its gradient must be coupled
with the movement of another molecule against its gradient.

\subsubsection{The \chem{Na^{+}/K^{+}} ATPase and the Resting Membrane
Potential}
This pumps out 3 \chem{Na^{+}} and pumps in 2 \chem{K^{+}} hydrolyzing one ATP
to drive the pumping of these ions against their gradients thus it is a primary
active transport protein. The sodium cannot pass back through the membrane, but
the potassium leaks out of the cell through \vocab[]{potassium leak channels}.
The movement of ions allows the cell to maintain osmotic balance. This leak
creates the electric potential across the plasma membrane with a net negative
charge on the interior of the cell. This is the \vocab[]{resting membrane
potential}. Also, the concentration gradient of the high sodium outside of the
cell is the driving force behind some secondary active transport as discussed in
the previous section. To summarize:
\begin{enumerate}
  \item To maintain osmotic balance between cellular interior and exterior
  \item To establish the resting membrane potential
  \item To provide the sodium concentration gradient used to drive secondary
    active transport
\end{enumerate}

\subsection{Endocytosis and Exocytosis}
\vocab[]{Exocytosis} is the method used by the cell to secrete proteins out of
the cell. \vocab[]{Endocytosis} is how the cell intakes things by invaginating
something and forming a vesicle called an \vocab[]{endosome}. There are three
types.\par

\vocab[]{Phagocytosis} is the nonspecific uptake of large particulate matter
where the vesicles later merges with a lysosome to break down the matter. This
is typically done by macrophages. (\textit{Note: This is no an
invagination}).\par

\vocab[]{Pinocytosis} is the nonspecific uptake of small molecules and
extracellular fluid via invagination. Virtually all eukaryotic cells participate
in pinocytosis.\par

\vocab[]{Receptor-mediated endocytosis} is \textit{highly} specific. The site of
endocytosis is marked by pits coated with the molecule \vocab[]{clathrin}
(inside the cell) and with receptors that bind to a specific molecule (outside
of the cell)

\section{Other Structural Elements of the Cell}
\subsection{Cell-Surface Receptors}
The binding of a ligand to a receptor on the cell surface triggers a response
within the cell, this process is called \vocab[]{signal transduction}. There are
three main types of signal-transducing cell-surface receptors.
\begin{description}
  \item \vocab[signal transduction!]{Ligand-gated ion channels} open an ion
      channel upon binding a particular neurotransmitter. An example is the
      ligand-gated sodium channel on the surface of the muscle cell at the
      neuromuscular junction.
  \item \vocab[signal transduction!]{Catalytic receptors} have an enzymatic
      active site on the cytoplasmic side of the membrane. Enzymatic activity is
      initiated by ligand binding at the extracellular surface. Generally, the
      catalytic role is that of a protein \vocab[]{kinase}, which attaches
      phosphate groups to proteins. Recall that proteins can be modified with
      phosphate on the side chain hydroxyl of serine, threonine, or tyrosine.
  \item \vocab[signal transduction!]{G-protein-linked receptor} does not
      directly transduce its signal, but transmits it into the cell with the aid
      of a \vocab[]{second messenger}. The most important of these is
      \vocab[second messenger!]{cyclic AMP}. It is known as the ``universal
      hunger signal'' because it is the second messenger for epinephrine and
      glucagon, which cause glycogen and fat breakdown. The process of
      epinephrine is explained below: (see Figure \ref{fig:cAMP})
      \begin{enumerate}
          \item Epinephrine binds to a specific G-protein-linked receptor
          \item The cytoplasmic portion of the receptor activates G-proteins,
              causing GDP to dissociate and GTP to bind in its place
          \item Activated G-proteins diffuse through the membrane and activate
              adenylyl cyclase
          \item Production of cAMP
          \item cAMP activates cAMP-dependent protein kinases (cAMP-dPK) in the
              cytoplasm
          \item cAMP-dPK phosphorylates certain enzymes resulting in
              mobilization of energy
      \end{enumerate}
      \begin{figure}[H]
          \centering
          \includegraphics[scale=0.2,frame]{cAMP.jpg}
          \caption{G-Protein Mediated Signal Transduction Stimulated by
          Epinephrine}
          \label{fig:cAMP}
      \end{figure}
      G-protein-linked receptors may be stimulatory or inhibitory. The above was
      a stimulatory one. Another second messenger pathway involves the
      activation of \vocab[second messenger!]{phospholipase C} which increases
      cytoplasmic \chem{Ca^{2+}} levels. \textit{Note that G-protein-linked
          receptors induce signal amplification as one ligand binding to the
          cellular surface can induce a cascade of signals to enzymes throughout
      the cell}.
\end{description}

\subsection{The Cytoskeleton}
The \vocab[]{cytoskeleton} provides structural support and allows movement of
the cell and its appgendages (cilia and flagella) and trasport of substances
throughout the cell. The internal cytoskeleton is composed of 
\begin{itemize}
    \item \vocab[cytoskeleton!]{microtubules} (largest)
    \item \vocab[cytoskeleton!]{intermediate filaments}
    \item \vocab[cytoskeleton!]{microfilaments} (smallest)
\end{itemize}

\subsubsection{Microtubules}
A \textbf{microtubule} consists of: \vocab[]{$ \alpha $-tubulin} and
\vocab[]{$\beta$-tubulin}. These two form a dimer which noncovalently bond
others to make a sheet, which rolls into a tube. The microtubule can elongate by
adding dimers to one end, but the other end is attached to the
\vocab[]{microtubule organizing center} (MTOC) so it cannot be elongated from
this end.\par

Withing the MTOC is a pair of \vocab[]{centrioles} composed of a ring of nine
microtubule triplets. During mitosis they separate and move to become antipodal.
then microtubules, called \vocab[]{aster}, extend out. Those that attach to
replicated chromosomes at the \vocab{ kinetochore } are called \vocab[]{polar
fibers}. This whole assembly is called the \vocab[]{mitotic spindle}, and
\vocab[]{kinetochore fibers}. The tiny microtubules that attach the kinetochore
to the spindle.

\subsubsection{Eukaryotic Cilia and Flagella}
The structure of and what drives motion in \vocab{ flagella } of eukaryotes is
different from that of prokaryotes. \vocab{ Cilia } are short and serve to move
fluid past the cell surface. Both flagella and cilia are build with a
``\vocab[]{9 + 2}'' arrangement of microtubules (9 doublets surrounding two lone
microtubules in the center) Each microtubules is bound to its neighbor by a
contractile protein called \vocab[]{dynein}. The cilium or flagellum is anchored
to the plasma membrane by a \vocab[]{basal body} which has the same structure as
a centriole. 

\subsubsection{Microfilaments}
Globular protein \vocab[]{actin} forms rods in the cytoplasm called
\vocab[]{microfilaments}. Microfilaments are dynamic and are responsible for
gross movements of the entire cell, such as pinching the dividing parent cell
into two daughter cells during cell division.

\subsubsection{Intermediate Filaments}
\vocab[]{Intermediate filaments} are in between the other two in terms of size.
They are more permanent than the other two and appear to be involved in
providing strong cell structure.

\subsection{Cell Adhesion and Cell Junctions}
\begin{wrapfigure}{r}{0.3\textwidth}
    \centering
    \vspace{-24pt}
    \includegraphics[width=0.20\textwidth,frame]{CellJunctions.jpg}
    \caption{Cell Junctions}
    \label{fig:junc}
\end{wrapfigure}
In some tissues, cells are tightly bound to each other. For example, the
intestinal wall is lined with \vocab[]{epithelium}. The cells are connected by
\vocab[]{tight junctions} forming a complete seal. However, epithelial cells of
the skin are held together tightly by \vocab[]{desmosomes} but do not form a
complete seal. In the heart the connection has holes called \vocab[]{gap
junctions} to allow for the movement of ions (see Figure \ref{fig:junc}).\par

More detail is given in the book about the different kinds of junctions
described above on page 271. 
\hfil \\
\hfil \\
\hfil \\

\section{The Cell Cycle and Mitosis}
The \vocab{ cell cycle } is separated into four phases. They are as follows:
\begin{description}
    \item \vocab[cell cycle!]{S phase} is when the cell actively replicates its
        genome. By the end it has two copies of the genome.
    \item \vocab[cell cycle!]{M phase} includes \vocab[]{mitosis}, partitioning
        cell components, and \vocab[]{cytokinesis}, physical division.
    \item \vocab[cell cycle!]{G\textsubscript{1} phase}
    \item \vocab[cell cycle!]{G\textsubscript{2} phase}
\end{description}
\begin{figure}[H]
    \centering
    \includegraphics[scale=0.2,frame]{cellcycle.jpg}
    \caption{The Cell Cycle}
\end{figure}

The cell spends most of its time in interphase, in fact some are stuck in
interphase (G\textsubscript{0}). The more specialized the cell the less likely
it is to remain capability of reproducing itself e.g. neurons, blood cells, and
skin cells. They are replenished by reproduction of \vocab[]{stem cells}.\par

During interphase the genome is spread out because replication is going on.
Sister chromatids are identical copies of a chromosome, attached to each other
at the centromere. Homologous chromosomes are equivalent but nonidentical and do
not come anywhere near each other during mitosis. The process of mitosis is
described below:
\begin{description}
    \item \vocab[mitosis!]{ Prophase } The nucleolus disappears, the spindle and
        kinetochore fibers appear, and the centrioles move antipodal. So now the
        cell has two MTOCs called asters. The nuclear envelope converts itself
        into many tiny vesicles.
    \item \vocab[mitosis!]{Metaphase} All the chromosomes line up forming the
        \vocab[Metaphase!]{metaphase plate}. Each sister chromatid is attached
        to spindle fibers that attach to the MTOCs so the chromosome gets pulled
        apart.
    \item \vocab[mitosis!]{Anaphase} The spindle fibers shorten, and the
        centromeres of each sister chromatid pair are pulled apart. The cell
        elongates, and cytokinesis begins with the formation of a
        \vocab[Anaphase!]{cleavage furrow}.
    \item \vocab[mitosis!]{Telophase} A nuclear membrane forms around the bunch
        of chromosomes at each end of the cell, the chromosomes decondense. Each
        daughter nucleus has 2n chromosomes.
\end{description}

\section{Cancer, Oncogenes, and Tumor Suppressors}
Mutated genes that induce cancer are called \vocab[]{oncogenes}. Usually these
are genes that regulate growth of the cell and the cell cycle.
\vocab[]{Protooncogenes} are the normal versions of the genes that allow for
regular growth patterns, but can be converted into oncogenes under the right
circumstances. \vocab[]{Tumor suppressor genes} produce proteins that are
inherent defense system to prevent the conversion of cells into cancer cells.
These genes work to detect damages to the genome and halt growth until it is
repaired or trigger apoptosis if the damage is too severe. \vocab[Tumor
suppressor genes!]{p53} is a example of a product of a common tumor suppressor
gene.

\subsection{Apoptosis}
\vocab[]{Apoptosis} allows a cell to shrink and die while simultaneously
minimizing damage to neighboring cells and limiting the exposure of other cells
to its cytosolic contents. The process goes as:
\begin{enumerate}
    \item Shrinking of the cell and disassembly of the cytoskeleton
    \item Nuclear envelope break down and genome broken into pieces
    \item Cell surface proteins emerge signaling phagocytic cells to clear away
        the dead cell
\end{enumerate}
A family of proteases called \vocab[]{caspases} is responsible for carrying out
apoptosis. There are two types:
\begin{description}
    \item \vocab[]{Initiator caspases} respond to death signals by clustering
        together and activating effector caspases in a cascade of activation
    \item \vocab[]{Effector caspases} cleave a variety of cellular proteins to
        trigger apoptosis
\end{description}

\subsection{Oxidative Stress}
\vocab{ Oxidative stress } is when production of reactive oxygen species (free
oxygen and peroxides) outstrips the cell's ability to detoxify them. These
reactive species can damage DNA leading to cancer. Though the immune system
cells produce these to kill pathogens as well.

\subsection{Senescence}
\vocab[]{Senescence} describes the process of biological aging which occurs at
both the cellular and the organismal level. Telomere length is a measure of the
age of the cell. Of course longer telomeres means a younger cell.


\end{document}
