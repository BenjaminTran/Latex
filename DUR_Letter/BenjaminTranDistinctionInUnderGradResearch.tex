\documentclass[12pt]{article}

\usepackage{geometry} % Required for adjusting page dimensions

%\longindentation=0pt % Un-commenting this line will push the closing "Sincerely," to the left of the page

\geometry{
	paper=letterpaper, % Change to letterpaper for US letter
	top=1.25in, % Top margin
	bottom=1.25in, % Bottom margin
	left=1in, % Left margin
	right=1in, % Right margin
	%showframe, % Uncomment to show how the type block is set on the page
}

\usepackage[T1]{fontenc} % Output font encoding for international characters
\usepackage[utf8]{inputenc} % Required for inputting international characters

\usepackage{stix} % Use the Stix font by default

\usepackage{microtype} % Improve justification

\setlength{\parindent}{0pt}
\setlength{\parskip}{1em}

\begin{document}
\pagenumbering{gobble}
Dear Committee for the Distinction in Undergraduate Research,

My name is Benjamin Tran and I am writing to express my desire to be considered
for the Distinction in Undergraduate Research Award. I have been primarily
working with Professor Wei Li of Bonner Nuclear Labs since my Freshman year, but
have also participated in the joint summer internship in medical physics between
Bonner Labs and MD Anderson Cancer Center culminating in a paper currently in
the beginning stages of writing.\par

During my time working for Professor Li, I have developed a solid foundation in
particle physics research. I have acquired technical skills primarily in writing
C++ and python code but have had exposure to other programming languages as
well. I have also acquired a working knowledge of ROOT, a data analysis
framework used by the particle physics community, as well as CMSSW and CRAB,
which are the frameworks used by the CMS experiment to utilize the CERN
computational Grid.\par

Professor Li has provided me with numerous research opportunities. My first
project was an analysis of particle multiplicity distributions in proton-lead
collisions the summer following my Freshman year. This allowed me to familiarize
myself with the use of ROOT. The following summer, I was given a ten-week
research opportunity on-site at the LHC where I was tasked with writing the
online channel monitoring display system for the new Layer-1 Stage-2 Calorimeter
subdetector. Funding was provided by the Shell Foundation Summer Fellowship
awarded through the Physics and Astronomy Department. Lastly, Professor Li
guided me throughout my senior thesis on the extraction of the $ v_{2} $
anisotropy harmonics of $ \Xi $ baryons produced in proton-lead collisions at $
\sqrt{S_{NN}} = 8 $ TeV; a novel analysis at this energy and collision system
size. This analysis project required the integration of all the skills I had
acquired in my time working with Professor Li.\par

My future plans include taking a gap year during which I will continue working
for Professor Li with the goal of expanding my senior thesis into a publication.
We plan to include the $ \mathrm{K_{S}^{0}} $ and $ \Lambda/\overline{\Lambda} $
particle species, which have been analyzed previously by the Wei Li group but at
lower energies. Then, I will extend the analysis to the $ \Omega $ particle. A
similar analysis on the $ \mathrm{D}\textsuperscript{0} $ meson is also underway
within the group. Ultimately, the analysis of these particle species will allow
us to observe the dependence of the $ v_{2} $ and related characteristics on
particle species. There is a possibility that I may return to CERN to assist in
the development of a new Web Based Monitoring (WBM) system, however this plan is
to be further discussed between Professor Li and the development team.\par
\vspace{0.75cm}
Sincerely,\par
\vspace{0.35cm}
Benjamin Tran


\end{document}
